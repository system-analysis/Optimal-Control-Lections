\documentclass[oneside, final, 14pt, draft]{article}

\usepackage{a4wide}
\usepackage{cmap}
\usepackage[T1]{fontenc}
\usepackage[utf8]{inputenc}
\usepackage[english, russian]{babel}
\usepackage{a4wide}
\usepackage[crop=pdfcrop]{pstool}
\usepackage{amsmath}
\usepackage{amsthm}
\usepackage{amssymb}
\usepackage{graphicx}
\usepackage{subcaption}
\usepackage{float}

\sloppy

\theoremstyle{definition}
\newtheorem{problem}{Задача}[section]
\newtheorem{ex}{Пример}[section]
\theoremstyle{definition}
\newtheorem{defin}{Определение}[section]
\theoremstyle{remark}
\newtheorem{rmk}{Замечание}[section]
\theoremstyle{theorem}
\newtheorem{thm}{Теорема}[section]
\newtheorem{lem}{Лемма}[section]
\newtheorem{nxt}{Следствие}[section]

\DeclareMathOperator{\conv}{conv}
\DeclareMathOperator{\arc}{arctg}
\DeclareMathOperator{\rg}{rg}
\DeclareMathOperator{\opt}{opt}
\DeclareMathOperator{\sign}{sign}


\renewcommand{\epsilon}{\varepsilon}
\renewcommand{\phi}{\varphi}
\renewcommand{\leq}{\leqslant}
\renewcommand{\geq}{\geqslant}
\newcommand{\bigrint}{\hbox to 0.8em{\hss\scalebox{1.2}[1]
    {\rotatebox[origin=c]{17}{$\displaystyle\int$}}\hss}}
\newcommand{\rint}{\mathop{\bigrint}\displaylimits}



\begin{document}
\begin{titlepage}

\thispagestyle{empty}

\begin{center}
\ \vspace{-3cm}

\includegraphics[width=0.5\textwidth]{msu.eps}\\
{\scshape Московский государственный университет имени М.~В.~Ломоносова}\\
Факультет вычислительной математики и кибернетики\\
Кафедра системного анализа

\vfill



\vspace{1cm}

{\Huge\bfseries <<Оптимальное управление: нелинейные системы>>}

\end{center}

\vspace{1cm}


\vfill
\begin{center}
Теоритический минимум, составленный Аксеновым К.~А., Широких И.~Ф. и Любимовым А.~И. на основе лекций, читаемых И.~В.~Рублевым в 6 семестре.
\end{center}
\begin{center}
Москва, 2015
\end{center}

\end{titlepage}
\tableofcontents
\newpage
\section{Постановка задачи оптимального управления}

\begin{equation*}
\begin{cases}
\dot{x}(t) = f(x(t), u(t))\\
J = \int\limits_{t_0}^{t_1} f^0(x(t),u(t)\,dt \longrightarrow \inf\\
u(t) \in \mathcal{P}\\
x(t_0) = x^0,\quad x(t_1) = x^1\\
\end{cases}
\end{equation*}
Считаем, что $x_0,x_1,t_0,t_1$ фиксированны.
\begin{rmk}
В задаче быстродействия $f^0\equiv1$.
\end{rmk}

{\flushleft Приведем данную задачу к задаче с терминальным функционалом. Введем новую переменную:}
\begin{equation*}
    \begin{cases}
        \dot x_0 = f^0(x,u)\\
        x_0(0) = 0
    \end{cases} \Rightarrow x_0(t_1) \equiv J.
\end{equation*}

{\flushleft Тогда, можно перейти к эквивалентной задаче:}

\begin{equation}\label{fsystem}
    \begin{cases}
        \dot{\widetilde{x}}(t) = \widetilde{f}(\widetilde{x}(t),u(t))\\
        x_0(t_1) \longrightarrow \inf\\
        u(t) \in \mathcal{P}\\
        x(t_0) = x^0,\quad x(t_1) = x^1
    \end{cases}
\end{equation}
Где $\widetilde{x} = (x_0,x_1,\ldots,x_n),\; \widetilde{f} = (f_0,f_1,\ldots,f_n)$.

{\flushleft Введем допольнительные обозначения.}

\begin{itemize}
\item $\widetilde{\psi} = (\psi_0,\underbrace{\psi_1,\ldots,\psi_n}_\psi) \text{ --- сопряженная переменная}.$
\item $H = \langle \widetilde{\psi}, \widetilde{f}(x,u)\rangle = \psi_0 f^0(x,u) + \psi_1 f^1(x,u) + ... + \psi_n f^n(x,u)$ --- функция Гамильтона--Понтрягина.
\item $\dot{\widetilde{\psi}}(t) = -\displaystyle \frac{\partial H}{\partial \widetilde{x}}$ --- сопряженная система.
\end{itemize}

Сделаем некоторые выводы:

\begin{align*}
&\dot{\psi_0} = 0 \Rightarrow \psi_0 = 0\\
&\dot{\psi}_j = - \psi_0 \frac{\partial f^0}{\partial x_j} - \sum\limits_{i = 1}^{n} \psi_i \frac{\partial f^i}{\partial x_j}\\
\end{align*}

Из чего получаем вид для сопряженной системы.

$$\dot{\psi} = - \psi_0 \frac{\partial f^0}{\partial x} - \left( \frac{\partial f}{\partial x}\right)^T \psi$$

Где $\frac{\partial f}{\partial x}$ задается следующим образом:

$$\frac{\partial f}{\partial x} =
  \begin{pmatrix}
        \frac{\partial f'}{\partial x_1} & \dots & \frac{\partial f'}{\partial x_n} \\
        \vdots & \ddots & \vdots \\
        \frac{\partial f^{(m)}}{\partial x_1} & \dots & \frac{\partial f^{(n)}}{\partial x_n}
  \end{pmatrix}$$

\begin{thm}[Принцип Максимума Понтрягина] \ \\
    Пусть $\left( x^*(\cdot), u^*(.)\right)$ --- оптимальная пара для системы $\eqref{fsystem}$ $\Rightarrow \exists\:\psi^*\colon\left[t_0^*, t_1^*\right]\rightarrow \mathbb{R}^{n+1}\colon$
    \begin{enumerate}
    \item Условие нетривиальности $$\psi^*(t) \neq 0,\;\dot\forall\,t.$$ 
    \item Сопряженная система
    \begin{equation*}
    \dot{\psi} = \left.-\frac{\partial H}{\partial \widetilde{x}}\middle|_{\substack{x = x^*(t) \\ u = u^*(t)}}\right. .
    \end{equation*}
    \item Условие максимума
    \begin{equation*}
    \left. H \middle |_{\substack{x = x^* \\ \psi = \psi^* \\ u = u^*}}\right. \stackrel{\text{п.в.}}{=}
    \sup\limits_{u \in \mathcal{P}} \left.H\middle|_{\substack{x = x^* \\ \psi = \psi^*}}\right. =
    \left. M \middle |_{\substack{x = x^* \\ \psi = \psi^*}}\right. .
    \end{equation*}
    \item $\psi_0 = const \leq 0$
    \end{enumerate}
\end{thm}

\begin{rmk}
\ \\
\begin{enumerate}
    \item
    Сопряженная переменная определена с точностью умножения на константу. Эту свободу всегда пытаются ликвидировать. Например $\psi_0^* = -1$ --- нормальный случай.\\
    \item $\widetilde{\psi}^* \neq 0$ в формулировке важно. Оно эквивалентно условию $\widetilde{\psi} \neq 0\ \dot \forall\ t$, т.~е. почти всюду (в силу теоремы о единственности решения однородной системы)
\end{enumerate}
\end{rmk}
\section{Формулировка принципа максимума для автономной задачи с нефиксированным временем, задача быстродействия. Вывод из предыдущей формулировки формулировок задач с фиксированным временем, для неавтономной системы. Условия трансверсальности.}
Для начала введем задачу быстродействия:

\begin{equation*}
\begin{cases}
&\dot{x}(t) = f(x(t), u(t)),\quad u \in \mathcal{P}\\
&x(t_0) = x_0\in\mathcal{X}^0,\quad x(t_1) = x_1\in\mathcal{X}^1
\end{cases}
\end{equation*}

{\flushleftГде $ t_0, x_0, x_1$ -- дано, $t_1$ -- свободно. Функционал имеет вид:}

$$J = \int\limits_{t_0}^{t_1} f^0 \left( x(t), u(t)\right) dt \rightarrow \inf.$$

{\flushleft После замен, для приведения к обобщенному виду:}

\begin{align*}
&\widetilde{x} = (x_0, \ldots, x_n)^T, \quad \widetilde{f} = (f^0, \ldots, f^n)^T\\
&H = \langle  \widetilde{\psi}, \widetilde{f}(x, u \rangle = \psi_0 f^0(x, u) + \langle  \psi, f(x, u) \rangle,\\
&M = \sup\limits_{u \in \mathcal{P}}H
\end{align*}

И тогда верен следующий Принцип Максимума.

\begin{thm}[Принцип Максимума Понтрягина] \ \\
Пусть $\left( x^*(t), u^*(t)\right)$ --- оптимальная пара $\Rightarrow \exists\: \psi^*\colon \left[t_0^*, t_1^*\right] \rightarrow \mathbb{R}^{n+1}\colon$
\begin{enumerate}
\item Условие нетривиальности 
$$\psi^*(t) \neq 0,\;\dot\forall\,t$$.
\item Сопряженная система
\begin{equation*}
\dot{\psi} = \left.-\frac{\partial H}{\partial x}\middle|_{\substack{x = x* \\ u = u^*}}\right. .
\end{equation*}
\item Условие максимума
\begin{equation*}
\left. H \middle|_{\substack{x = x^* \\ \psi = \psi^* \\ u = u^*}}\right. =
 \sup\limits_{u \in \mathcal{P}} \left.H\middle|_{\substack{x = x^* \\ \psi = \psi^*}}\right. =
 \left. M \middle|_{\substack{x = x^* \\ \psi = \psi^*}}\right. .
\end{equation*}
\item $\psi_0 = \mathrm{const} \leq 0$
\item Условие трансверсальности
\begin{equation*}
\psi (t_0^*)\: \bot \: T_{x^*(t_0^*)} \mathcal{X}^0;
\end{equation*}
\begin{equation*}
\psi (t_1^*)\: \bot\: T_{x^*(t_1^*)} \mathcal{X}^1.
\end{equation*}
\end{enumerate}
\end{thm}

{\flushleft Рассмотрим неавтономный случай.}

$$ \dot{x}(t) = f(t, x(t), u(t))$$
$$ J = \int\limits_{t_0}^{t_1}f^0(t, x(t), u(t))dt$$

Введем новую переменную $x_{n+1}$ такую, что

\begin{equation*}
\begin{cases}
\dot x_{n+1}(t) =  1\\
x_{n+1}(0) = t_0
\end{cases}
\end{equation*}

То есть $x_{n+1} \equiv t$. После чего можно сделать еще одну замену и перейти к задаче в новых обозначениях:

\begin{align*}
&\overline{x} = (x_0, x_1, ..., x_{n+1})^T\\
&\overline{f} = (f^0, f^1, ..., f^{n+1})^T\\
&\overline{\psi} = (\psi_0, \psi_1, ..., \psi_{n+1})^T\\
&\overline{H} = \psi_0 f^0 + \langle\psi, f\rangle + \psi_{n+1}\cdot 1 = H + \psi_{n+1}\\
&\overline{M} = \sup\limits_{u \in \mathcal{P}} \overline{H} = M + \psi_{n+1}^*
\end{align*}

{\flushleft В итоге мы получили автономную систему. Далее рассмотрим задачу в нормальном случае, когда $\widetilde{\psi} \neq 0$. Принцип Максимума примет вид:} 

\begin{enumerate}

\item \textbf{Сопряженная система.}\\
\begin{align*}
&\dot{\psi}_0^* = 0\\
&\dot{\psi}^* = -\frac{\partial H}{\partial x}\\
&\dot{\psi}_{n+1}^* = - \psi_0^*\frac{\partial f^0}{\partial t} - \left\langle \psi^*, \frac{\partial f}{\partial t}\right\rangle
\end{align*}

\item \textbf{Условие максимума} остается без изменений.

\item \textbf{К условию трансверсальности по $0$-й переменной}.
\begin{align*}
&\psi_0 = \mathrm{const} \leq 0\\
&\overline{x} = (x_0,\widehat{x})^\mathrm{T} = (x_0,x_1,\ldots, x_{n+1})^\mathrm{T}\\
&\widehat{x} (t_0) = (x^0,t_0)\in \mathcal{X}^0\times\mathbb{R} = \widehat{\mathcal{X}}^0\\
&\widehat{x} (t_1) = (x^1,t_1) \in \mathcal{X}^1\times\mathbb{R}=\widehat{\mathcal{X}}^1\\
\end{align*}
Тогда имеем:
$$T_{\widehat{x}} \widehat{\mathcal{X}}^1 = \mathcal{L}\{(0, ..., 0, 1)\}$$

\item \textbf{Условие трансверсальности.} \\
Из предыдущего $T_{\widehat{x}} \widehat{\mathcal{X}}^1 = T_x \mathcal{X}^1 \times T_x \mathbb{R}$, поэтому
\begin{align*}
&\psi_{n+1}^*(t_1^*) = 0\\
&\psi_{n+1}^* (t) = \int\limits_t^{t_1^*}\left[ \psi_0^* + \frac{\partial f^0}{\partial t}(\tau, x^*(\tau), u^*(\tau)) + \left\langle \psi^*(t), \frac{\partial f}{\partial t}\right\rangle\right]d\tau\\
&M \equiv \overline{M} - \int\limits_t^{t_1^*}\left[ \psi_0^* + \frac{\partial f^0}{\partial t}(\tau, x^*(\tau), u^*(\tau)) + \left\langle \psi^*(t), \frac{\partial f}{\partial t}\right\rangle\right]d\tau
\end{align*}

\end{enumerate}

Теперь рассмотрим случай с фиксированным временем. Пусть $t_0,\,t_1$ фиксированы. Фиксация $t_1$ задается следующим образом:

\begin{equation*}
\begin{cases}
&x_{n+1}(t_1) = t_1^*\\
&t_1 = t_1^*
\end{cases}
\end{equation*}

Из условий трансверсальности:

$$ M \equiv \mathrm{const} - \int\limits_t^{t_1^*}\left[ \psi_0^* + \frac{\partial f^0}{\partial t}(\tau, x^*(\tau), u^*(\tau)) +\left\langle \psi^*(t), \frac{\partial f}{\partial t}\right\rangle\right]d\tau$$
%\ \\
%$$\dot{x}(t) = f(\ \cdot\ , x(t), u(t))$$
%$$x(t_0) = x^0\ \rightarrow \ x(t_1) = x^1 \qquad t_0, t_1 \text{ --- фиксированные}$$
%$$J = \int\limits_{t_0}^{t_1} f^0 (x(t), u(t))dt$$
%$$\quad M|_{\substack{x = x^* \\ \widetilde{\psi} = \widetilde{\psi}^*}} \equiv const$$
%\ \\
Есть два существенно различных случая.
\begin{itemize}
\item Анормальный случай. $$\psi_0^* = 0$$
\item Нормальный случай. $$\psi_0^* < 0$$
В силу положительной однородности сопряженной системы можно считать $$\psi_0^* = -1$$
\end{itemize}
   
Далле будут использоваться обозначения:
\begin{align*}
&\psi(t_0) = \psi^0\\
&x^*[t] = x^* (t, t_0, x^0, \psi^0)\\
&\psi^*[t] = \psi^*(t, t_0, x^0, \psi^0)
\end{align*}
\section{Общая задача ОУ с условиями типа равенств, вывод из нее формулировки ПМП для автономной задачи с нефиксированным временем. Вывод формулировки ПМП для задачи со свободным правым концом и интегрально-терминальным функционалом (задача Майера-Больца).}
Для начала сформулируем общую задачу оптимального управления. Введем обозначение $(t_0, x^0, t_1, x^1) = e$, тогда задача формулируется следующим образом:
\begin{equation*}
\begin{cases}
&\dot{x}(t) = f(x(t), u(t))\\
&u(t) \in \mathcal{P}\\
&\varphi_0(e) \rightarrow \inf\\
&\varphi_1(e) = ... = \varphi_k(e) = 0\quad (*)
\end{cases}
\end{equation*}
В данном случае имеется ввиду, что $x(t_0)$, $x(t_1)$ и $t_0$ фиксированны (за счет $\varphi_k(e)$). 
\begin{rmk}
В случае неавтономной системы, задача сводится к автономной одной заменой.
\end{rmk}
{\flushleft Выпишем вывод для автономной задачи с нефиксированным временем:}
$$ t_0, x^0, x^1 \text{ --- фиксированные},\quad t_1 \text{ --- свободная.}$$
Функционал имеет вид:
$$ J = \int\limits_{t_0}^{t_1} f^0(x(t), u(t))dt \rightarrow \inf$$
После, делавшейся уже ранее замены, имеем:
\begin{equation*}
\begin{cases}
\dot{x}_0 = f^0(x, u)\\
x_0(t_0) = 0
\end{cases}
\end{equation*}
$$ x_0(t) = \int\limits_{t_0}^t f^0(x(t), u(t))dt$$
Проведем некоторые переобозначения.
\begin{align*}
&\xi = (x_0,\, x_1,\ldots,\,x_n)^\mathrm{T}\\
&g = (f^0,\,f^1,\ldots,\,f^n)^\mathrm{T}\\
&\xi^0 = (0,\,x_1^0,\ldots,\,x_n^0)^\mathrm{T}\\
&\xi^1 = (J,\,x_1^1,\ldots,\,x_n^1)^\mathrm{T}
\end{align*}
Далее нужно записать ограничения в виде $(*)$:
$$
\begin{aligned}
    &\varphi_0 = \xi_1^1 \rightarrow \inf\\
    &\varphi_1 = t_0 - \tau_0 = 0\\
    &\varphi_2 = \xi_1^0 = 0\\
    &\varphi_3 = \xi_2^0 - a_1^0 = 0\\
    &\ldots \\
    &\varphi_{n+1} = \xi_{n+1}^0 - a_n^0 = 0\\
    &\varphi_{n+2} = \xi_2^1 - a_1^1 = 0\\
    &\ldots \\
    &\varphi_{2n+2} = \xi_{n+1}^1 - a_n^1 = 0   
\end{aligned}
$$
\subsection{Вывод формулировки ПМП для задачи со свободным правым концом и интегрально-терминальным функционалом}
\begin{align*}
&x_0,\ t_0,\ t_1 \text{ --- фиксированные}\\
&J = \int\limits_{t_0}^{t_1} f^0 (x(t), u(t))dt + g(x(t_1))
\end{align*}
Проводим стандартную замену:
\begin{equation*}
\begin{cases}
\dot{x}_0 = f^0\\ 
x_0(t_0) = 0
\end{cases}
\end{equation*}
И получаем
$$  J = x_0(t_1) + g(x(t_1)) = \varphi_0$$
Остальные ограничения имеют вид:
\begin{align*}
&\varphi_1 = t_0 - \tau_0\\
&\varphi_2 = t_1 - \tau_1\\
&\varphi_{3 + 1} = \xi_2^0 - a_1^0\\
&\ldots\\
&\varphi_{3 + n} = \xi_{n + 1}^0 - a_n^0
\end{align*}
На левом конце:
$$ \psi_0^* = -\lambda_3, \quad \psi^*(t_0) = -(\lambda_4,\ldots, \lambda_{n+3})^\mathrm{T}$$
На правом конце:
\begin{align*}
&\left( \frac{\partial\varphi}{\partial \xi_1}\right)^* = \begin{pmatrix}
    1 & \dots & 0 \\
    \frac{\partial g}{\partial x_i}& \ddots & \vdots \\
    \ldots & \dots & 0
\end{pmatrix} \Rightarrow 
\begin{cases}
    \psi_0^*=\lambda_0\\
    \psi^*(t_1^*) = \frac{\partial g}{\partial x} \lambda_0
\end{cases}\\
&H|_{t = t_0} = \lambda_1\\
&H|_{t = t_1} = -\lambda_2\\
&\frac{\partial H}{\partial t} \equiv 0\\
&\psi_0^* \leq 0 \text{ --- const}
\end{align*}
В итоге получаем единственное информативное условие:
$$\psi(t_1^*) = \psi_0^*\frac{\partial g}{\partial x}.$$

\section{Теоремы о зависимости решения обыкновенного ДУ от параметра, непрерывность и дифференцируемость траектории по краевым условиям, уравнение в вариациях.}
Рассмотрим дифференциальное уравнение:
\begin{equation}
 \dot{x} = g(t, x).
 \label{DifEq}
\end{equation}
Пусть функция $g(t, x):$
\begin{enumerate}
\item $g: [t_0 - a, t_0+a] \times B_b (x^0) \rightarrow \mathbb{R}^n;$
\item $g$ измерима по $t \quad \forall x \in B_b (x^0);$
\item $g$ непрерывна по $x \quad \dot{\forall}t \in [t_0 - a, t_0+a];$
\item $\| g(t, x)\| \leq m(t), \quad m(\cdot)$-- интегрируема по Лебегу, $t \in [t_0 - a, t_0+a];$
\end{enumerate}
Либо другой вариант, который мы обозначим $\spadesuit$:
\begin{enumerate}
\item $g: [t_0 - a, t_0+a] \times \mathbb{R}^n \rightarrow \mathbb{R}^n;$
\item $g$ измерима по $t \quad \forall x \in \mathbb{R}^n;$
\item $g$ непрерывна по $x \quad \dot{\forall} t \in [t_0 - a, t_0+a];$
\item $\| g(t, x^1) - g(t, x^2)\| \leq L(t) \|x^1 - x^2\|$, где $L(t)$--- интегрируема по Лебегу.
\end{enumerate}
Далее выпишем условия продолжимости:
\begin{enumerate}
\item $\| g(t, x)\| \leq A(t)\|x(t)\| + B(t);$
\item $\langle x, g(t, x) \rangle \leq \widetilde{A}(t)\|x(t)\|^2 + \widetilde{B}(t).$
\end{enumerate}
Далее в лекциях предложено доказательство того факта, что из $1 \Rightarrow 2$, но мы это доказательство опустим.
\begin{thm} \ \\
Пусть для $\mu \in \mathbb{R}^n$
$$
\left\{
\begin{aligned}
    &\dot{x}(t)=g(t, x, \mu);\\
    &x(t_0) = x^0.\\
\end{aligned}
\right.
$$
Пусть также выполнены условия $\spadesuit$ и $x[t] = x(t, t_0, x^0, \mu)$, тогда $x(t, t_0, x^0, \mu)$ непрерывна по $x^0, \mu$.
\end{thm}

\begin{thm} \ \\
Пусть выполнены условия $\spadesuit$, $\frac{\partial g}{\partial x}, \frac{\partial g}{\partial \mu} \in C \Rightarrow x(t, t_0, x^0, \mu)$ дифференцируема по $x^0, \mu$.
Пусть $\frac{\partial x}{\partial x^0} = Y(t), \quad \frac{\partial x}{\partial \mu} = Z(t):$
\begin{equation}
\dot{Y}(t)  = \frac{\partial g}{\partial x} Y(t) \qquad Y(t_0) = E 
\label{EqVar1}
\end{equation}
\begin{equation}
\dot{Z}(t)  = \frac{\partial g}{\partial x} Z(t) +  \frac{\partial g}{\partial \mu}  \qquad Z(t_0) = 0
\label{EqVar2}
\end{equation}
\end{thm}
{\flushleft В данной теореме (\ref{EqVar1}) обозначает (УВ1), а (\ref{EqVar2}) --- (УВ2).
Далее будет предложено определение $\epsilon$-решения, а также начальные сведения уравнений в вариациях.}
\begin{defin}
$x$ --- $\epsilon$-решение (\ref{DifEq}), если $x \in AC$ и 
$$
\| \dot{x} - g(t, x) \| \leq \epsilon \quad \dot{\forall} t.
$$
\end{defin}
Сильная вариация. Изменим управление следующим образом:\\
$$
u^{\epsilon}(t) = 
\left\{ 
\begin{aligned}
& u^*(t), \quad t \in \left[t_0, t_1\right]\backslash (\tau-\epsilon, \tau]\\
& v, \quad t \in  [\tau-\epsilon, \tau)\\
\end{aligned}
\right.
$$
Тогда получим, что
\begin{align*}
&J(\epsilon) = J(u^{\epsilon}(\cdot)) \geq J(u^{*}(\cdot))\\
&\lim\limits_{\epsilon \rightarrow +0} \frac{J(u^{\epsilon}(\cdot)) -J(u^{*}(\cdot))}{\epsilon} \geq 0\\
\end{align*}
При этом $u^{\epsilon}(\cdot) \mapsto x^{\epsilon}(\cdot)$ ($u^{\varepsilon}$ однозначно порождает $x^{\varepsilon}$).

\begin{lem} При $t \geq \tau$ функция $x^{\varepsilon}$ имеет вид:
$$x^{\epsilon}(t) = x^{*}(t) + \epsilon \delta x + \bar{\bar{o}}(\epsilon)$$
Где
\begin{equation*}
\begin{cases}
\delta x(\tau) = f(\tau, x^*(\tau), v) - f(\tau, x^*(\tau), u^*(\tau))\\
\dfrac{d}{dt} (\delta x(\tau)) = \left. \dfrac{\partial f}{\partial x}\middle|_{\substack{x = x^* \\ u = u^*}} \delta x(t)\right.
\end{cases}
\end{equation*}
\end{lem}
\section{Вариация управления для задачи со свободным правым концом и интегрально-терминальным функционалом (задача Майера-Больца).}
\begin{lem} Имеется система $\dot{x} [t] = f (t, x[t]\ |\ \overline{u}(t)),\;x(t_0) = x^0$. Если проварьировать управление $\overline{u}(t)\colon$
$$\varepsilon > 0 \quad u^\varepsilon [t] = u[t] + \varepsilon \upsilon(t),$$
то оно породит новую траекторию.
$$u^\varepsilon [t] \mapsto x^\varepsilon [t] = x (t, t_0, x^0\ |\ \overline{u}^\varepsilon(t))$$
Где
\begin{align*}
&x^\varepsilon [t] = x[t] + \varepsilon \delta x[t] + \bar{\bar{o}}(\varepsilon)\\
&\begin{cases}
\delta \dot{x} = \dfrac{\partial f}{\partial x} \delta x + \dfrac{\partial f}{\partial u}\upsilon\\
\delta x[t_0]  = 0\\
\end{cases}\\
&\delta x = \left.\frac{\partial}{\partial \varepsilon} x^\varepsilon\right|_{\varepsilon = 0}.
\end{align*}
\end{lem}

\begin{lem}  Имеется система $\dot{x} [t] = f (t, x[t]\ |\ \overline{u}(t)),\;x(t_0) = x^0$. В случае варьирования начального условия система принимает вид: 
\begin{align*}
&x^\varepsilon[t] = x[t] + \varepsilon \delta x[t] + \bar{\bar{o}}(t, \varepsilon)\\
&\begin{cases}
\dot{x}^\varepsilon [t] = f (t, x^\varepsilon[t], u(t))\\
x^\varepsilon[t_0] = x^0 + \varepsilon y^0
\end{cases}\\
&\begin{cases}
\delta \dot{x} = \dfrac{\partial f}{\partial x} \delta x\\
\delta x[t_0] = y^0
\end{cases}
\end{align*}
\end{lem}

\begin{nxt} В случае преставления вариации начального условия в виде $M\rho$ имеем:
\begin{align*}
&x^\varepsilon[t] = x[t] + (\delta x)\rho +  \bar{\bar{o}}(\rho)\\
&x^\varepsilon[t_0] = x^0 + M\rho\\
&\begin{cases}
\delta \dot{x} = \dfrac{\partial f}{\partial x} \delta x\\
\delta x[t_0] = M
\end{cases}
\end{align*}
\end{nxt}

\section{Абстрактная задача нелинейного программирования с условиями типа равенства. Определение конического дифференциала функционала. Теорема об абстрактном правиле множителей Лагранжа для этой задачи нелинейного программирования.}
Рассмотрим абстрактную задачу и запишем для нее Принцип Максимума Понтрягина:
\begin{equation}\label{ssystem}
\begin{cases}
\dot{x} = f(t, x, u)\\
%x(t_0) = x^0\\
u(t) \in \mathcal{P}\\
J = \phi(x(t_1)) \rightarrow \inf\\
\end{cases}
\end{equation}
Функция Гамильтона--Понтрягина:
$$H = \langle \psi,f(t, x, u)\rangle$$
Обозначим ограничения:
$$
 \phi_0 = \phi,\quad \phi_1 = t_0 - \overline{t}_0,\quad \phi_2 = t_1 - \overline{t}_1, \quad \phi_{2+j} = x_j^0 - \overline{x}_j^0, \quad j=1,\ldots, n.
$$
Тогда, пусть:
$$
\Phi = \begin{bmatrix}
            \phi_0\\
            \phi_1\\
            \vdots
   \end{bmatrix}
$$
\begin{thm}[Принцип максимума Понтрягина] \ \\
Пусть $\left( x^*(t), u^*(t)\right)$ --- оптимальная пара для задачи $\eqref{ssystem}$ $\Rightarrow \exists\; \psi: \left[t_0^*, t_1^*\right] \rightarrow \mathbb{R}^n\colon$
\begin{enumerate}
\item Условие невырожденности.
$$\psi(t) \neq 0,\; \dot \forall\,t$$
\item Сопряженная система
\begin{equation*}
\dot{\psi} = -\left(\frac{\partial f}{\partial x}\right)^\mathrm{T} \psi;
\end{equation*}
\item Условие максимума
\begin{equation*}
\langle \psi(t), f(t, x^*(t), u^*(t))\rangle \stackrel{\text{п.в. }}{=} \sup\limits_{u \in \mathcal{P}}\langle \psi(t), f(t, x^*(t), u)\rangle;
\end{equation*}
\item Условие трансверсальности
\begin{equation*}
\psi (t_0) = -\left(\frac{\partial \Phi}{\partial x^0}\right)^\mathrm{T} \lambda = -\sum\limits_{j=1}^n \lambda_{2+j} ;
\end{equation*}
\begin{equation*}
\psi (t_1) = \left(\frac{\partial \Phi}{\partial x^1}\right)^\mathrm{T} \lambda = \lambda_0\frac{\partial \phi}{\partial x};
\end{equation*}
\item Условие трансверсальности по времени
\begin{equation*}
\left.H\right|_{t=t_0} = \left\langle \frac{\partial \Phi}{\partial t^0}, \lambda \right\rangle = \lambda_1;
\end{equation*}
\begin{equation*}
\left.H\right|_{t=t_1} = -\left\langle \frac{\partial \Phi}{\partial t^1}, \lambda \right\rangle = -\lambda_2;
\end{equation*}
\end{enumerate}
\end{thm}
{\flushleft Далее введем понятие конического дифференциала, но для него нам понадобится несколько (не то чтобы совсем мало) дополнительных определений:}
\begin{enumerate}
\item $J = \phi_0(t_0, x(t_0), t_1, x(t_1));$
\item Пусть $e = (t_0, x(t_0), t_1, x(t_1)) \Rightarrow \phi_1(e) = \ldots = \phi_k(e) = 0;$
\item Пусть $S$--- множество четверок вида:
$S = \{(t_0, t_1, x(t_0), u(\cdot))\};$
\item 
$\sigma \in S \rightarrow  \begin{bmatrix}
            \phi_0(e)\\
            \vdots \\
            \phi_k(e)\\
   \end{bmatrix} \in \mathbb{R}^{k+1};$
\item $\hat{S} = S \bigcap \{\phi_1(e) = \ldots = \phi_k(e) = 0\};$
\item $C = \conv \{0, l^1, \ldots, l^{k+1}\}, $ где $ l^j = (0, \ldots, 0, 1, 0, \ldots, 0)^T;$
\item $\rho \in \eta C, \quad \eta>0, \quad \rho \text{--- параметр вариации};$
\item $\zeta: \eta C \rightarrow S;$
\item $\Phi: S \rightarrow \mathbb{R}^{k+1}$ (похоже, что подразумевается перевод из начальных условий в значения ограничивающих функций $\varphi$).
\end{enumerate}
Из предыдущих соображений вытекает, что:

\begin{equation}\label{phivari}
\Phi(\zeta (\rho)) =\Phi(\sigma^*) + M\rho +\bar{\bar{o}}(\rho).
\end{equation}

\begin{defin}[Конический дифференциал]\ \\
$M$ --- конический дифференциал, соответствующий $\zeta^*, \quad M \in \mathbb{R}^{(k+1)\times(k+1)}.$
\end{defin}
\begin{defin}[Конус вариаций]\ \\
$D\subseteq \mathbb{R}^{k+1}$ ---  выпуклый конус вариаций, если $\forall (d^1, \ldots, d^{k+1}) \in D, (d^1, \ldots, d^{k+1})$ --- ЛНЗ $\Rightarrow$
$$\Rightarrow M = [d^1, \ldots, d^{k+1}]\text{--- конический дифференциал в } \sigma^*.$$
\end{defin}
{\flushleft Осталось только выписать теорему об абстрактном правиле множителей Лагранжа, чем мы и займемся:}
\begin{thm}[Абстрактное правило множителей Лагранжа]\ \\
Пусть $\sigma^*$ --- оптимум прайм (наиболее оптимальное: $\phi_0 \rightarrow \min)$, $\phi_1 =\ldots =\phi_k = 0,\quad \forall \sigma \in S.$ Пусть $D$ --- конус вариаций, то есть набор вектором из которых можно составить $M$ в $\eqref{phivari}$. Тогда:
$$\exists \lambda = (\lambda_0, \lambda_1, \ldots, \lambda_k)^\mathrm{T} \neq 0: \quad \lambda_0 < 0, \quad \langle \lambda, d\rangle \leq 0 \quad \forall d \in D.$$
\end{thm}
\section{Выпуклый конус вариаций для общей задачи ОУ с условиями типа равенства, семейства векторов, порождающих конус. Лемма о вариации траектории в произвольный фиксированный момент времени, лемма о вариации траектории в левом и правом конце.}
Ввод конуса вариаций был осуществлен в предыдущем билете, поэтому воздержимся от повторов и позволим читателю переместиться на предыдущую страницу, дабы в этом удостовериться. Всюду далее будут действовать обозначения, введеные ранее.
\begin{thm}\ \\
\begin{enumerate}
\item $$\pm \left[ \frac{\partial \Phi}{\partial t_0} + \frac{\partial \Phi}{\partial x^0} f(t_0, x^0, u(t_0))\right]$$ 
\item $$\pm \left[ \frac{\partial \Phi}{\partial t_1} + \frac{\partial \Phi}{\partial x^1} f(t_1, x^1, u(t_1))\right]$$
\item \begin{align*}
&\frac{\partial \Phi}{\partial x^0}\delta x[t_0] + \frac{\partial \Phi}{\partial x^1} \delta x[t_1], \qquad t\in[t_0, t_1]\\
&\frac{d}{dt}\left(\delta x(t)\right) = \left( \frac{\partial f}{\partial x}\right) \delta x^1 \delta x[t_1]=\delta x^0
\end{align*}
\item
$$\frac{\partial \Phi}{\partial x^1} \delta x[t_1]$$
$$\frac{d}{dt}\left(\delta x(t)\right) = \left( \frac{\partial f}{\partial x}\right) \delta x^1 \delta x[\tau] = f(\tau, x[\tau], v) - f\left(\tau, x[\tau], u^*(\tau)\right), \quad t \in [\tau, t_1]$$
\end{enumerate}
$\Rightarrow D$ порожденный (1) - (4) --- конус вариаций.
\label{conus}
\end{thm}

{\flushleft Далее будут две леммы, связанные с вариацией траектории:}

\begin{lem}[о вариации траектории в произвольный фиксированный момент времени]\ \\
\begin{align*}
&x^{\rho}(t) = x(t) + \delta x(t)\rho + \bar{\bar{0}}(t, \rho)\\
&\delta x(t) = \sum\limits_{j=0}^{S_m} \delta x^j(t), \quad m: \tau{S_m} \leq t\\
&\frac{d}{dt}\delta x^j(t) = \frac{\partial f}{\partial x}\delta x^j(t)\\
&\delta x^0(t_0) = N\\
&\delta x^j(\tau_j) = [f(\tau_j, x(\tau_j), v_j) - f(\tau_j, x(\tau_j), u(\tau_j))]a^j
\end{align*}
\end{lem}

\begin{lem}[о вариации траектории в левом и правом конце]\ \\
\begin{align*}
&x^{\rho}(t_0 + b^0\rho) = x(t_0) +N\rho +f(t_0, x(t_0), u(t_0))b^0\rho + \bar{\bar{o}}(\rho)\\
&x^{\rho}(t_1 + b^1\rho) = x(t_1) +\delta x(t_1)  \rho +f(t_1, x(t_1), u(t_1))b^1\rho + \bar{\bar{o}}(\rho)
\end{align*}
\end{lem}

\section{Основная идея док-ва ПМП для задачи ОУ с условиями типа равенства. Лемма об изменении по времени значения функции максимума по значениям управления в фиксированный момент времени, постоянство значения функции максимума при независимости функции, от которой берется максимум, от времени.}
Для начала сформулируем сам ПМП:
\begin{thm}[ПМП]
\ \\
    Пусть $\{ \xi^*(\cdot), u^*(\cdot)\}$ --- оптимальная пара для задачи ОУ с ограничениями типа равенства ($\dot{x} = f(t, x, u)$), тогда 
    $ \,\exists\; \psi^*: \left[t_0^*, t_1^*\right] \rightarrow \mathbb{R}^{n+1}, \quad \exists \lambda = (\lambda_0, \ldots, \lambda_k)^T \neq 0, \ \lambda_0\leq 0$:
    \begin{enumerate}
        \item Условие нетривиальности
        $$\psi^* \neq 0,\; \dot\forall\, t$$
        \item Сопряженная система
        $$
        \dot{\psi} = - \left. \frac{\partial f}{\partial x} \middle|_{\substack{\xi = \xi^*\\ u = u^*}}\psi\right.
        $$
        \item Условие максимума
        $$
        \sup\limits_{v \in \mathcal{P}} \langle \psi, f(\tau, x(\tau), v)\rangle = \langle \psi, f(\tau, x(\tau), u(\tau))\rangle
        $$
        \item 
            \begin{enumerate}
            \item Условие трансверсальности 1
            $$
            \psi(t_1) = \left(\frac{\partial \Phi}{\partial x^1}\right)^T \lambda;
            $$
            \item Условие трансверсальности 2
            $$
            \psi(t_0) = -\left(\frac{\partial \Phi}{\partial x^0}\right)^T \lambda;
            $$
            \item Условие трансверсальности 3
            $$
            \left. H \right|_{t=t_1} = -\left\langle \frac{\partial \Phi}{\partial t_1}, \lambda \right\rangle;
            $$
            \item Условие трансверсальности 4
            $$
            \left. H \right|_{t=t_0} = \left\langle \frac{\partial \Phi}{\partial t_0}, \lambda \right\rangle.
            $$
            \end{enumerate}
    \end{enumerate}
\end{thm}
{\flushleft Далее в виде алгоритма будет описан ход доказательства:}
\begin{enumerate}
            \item Для начала введем конус вариаций по теореме (\ref{conus}), что позволит нам использовать дополнительные условия.
            \item Определим $\forall d \in D$, так, чтобы $\langle \lambda, d \rangle \leq 0$, исходя из определения конуса: $$d = \frac{\partial \Phi}{\partial x^1} \delta x(t_1). $$
            \item Определим сопряженную систему через условия $\mathit{1}$ и $\mathit{3a}$.
            \item Пользуясь отдельными пунктами (\ref{conus}) выводим (УМ), (УТ1), (УТ2), (УТ3), (УТ4).
            \item Как-то так и доказывается ПМП
\end{enumerate}
Введем лемму:
\begin{lem}
Пусть $h(t, u)$ --- непрерывна, $\frac{\partial h}{\partial t}$ --- непрерывна, $u(t)$ --- кусочно непрерывна(слева); $\max\limits_{v \in \mathcal{P}} h(t, v) = h(t, u(t)) \Rightarrow$
$$
\Rightarrow h(t, u(t)) \text{ --- непрерывна по } t \text{ и кусочно дифференцируема};
$$
$$
\frac{d}{dt} h(t, u(t)) = \frac{\partial h}{\partial t}(t, u(t)) \quad \dot{\forall} t.
$$
\end{lem}
Осталось рассмотреть систему, для которой функция Гамильтона-Понтрягина не зависит от времени:
\begin{equation*}
\begin{cases}
\dot{x} = f(x, u), \quad u \in \mathcal{P}\\
x(t_0) = x^0\\
x(t_0) \rightarrow x^1\\
\mathcal{X}[t] = \mathcal{X}(t, t_0, \mathcal{X}^0)\\
t_1- t_0 \rightarrow \inf\
\end{cases}
\end{equation*}
Введем функцию Гамильтона-Понтрягина:
$$
H(x, \psi, u) = \langle \psi, f(x, u) \rangle;
$$
$$
M(x, \psi) = \sup\limits_{u \in \mathcal{P}}\langle \psi, f(x, u) \rangle;
$$
\begin{thm}[Принцип максимума Понтрягина] \ \\
    Пусть $\left( x^*(\cdot), u^*(.)\right)$ --- оптимальная пара $\Rightarrow \exists\, \psi^*: \left[t_0^*, t_1^*\right] \rightarrow \mathbb{R}^{n+1}$:
    \begin{enumerate}
    \item Условие нетривиальности
    $$\psi^* \neq 0,\;\dot\forall\,t$$
    \item Сопряженная система
    \begin{equation*}
    \dot{\psi} = -\left.\left(\frac{\partial H}{\partial x}\right)\right|_{u=u^*} = -\left(\frac{\partial M}{\partial x}\right)
    \end{equation*}
    \item Условие максимума
    \begin{equation*}
    \left. H \middle|_{\substack{x = x^* \\ \psi = \psi^* \\ u = u^*}}\right. \stackrel{\text{п.в.}}{=}
    \sup\limits_{u \in \mathcal{P}} \left.H\middle|_{\substack{x = x^* \\ \psi = \widetilde{\psi}^*}}\right. =
    \left. M \middle|_{\substack{x = x^* \\ \psi = \widetilde{\psi}^*}}\right. .
    \end{equation*}
    \item $M = const \geq 0$
    \end{enumerate}
\end{thm}

\section{Система ДУ при условиях Каратеодори. Решение Каратеодори. Теорема о существовании, единственности и продолжимости решения для системы ДУ при условиях Каратеодори. Условия на систему управления, гарантирующие существование, единственность и продолжимость траектории системы, отвечающей измеримому управлению.}
Введем систему:
\begin{equation}\label{thirdsystem}
\begin{cases}
\dot{x} = g(t, x)\\
x(t_0) = x^0
\end{cases}
\end{equation}

\begin{defin}
$x(\cdot)$ --- решение по Каратеодори задачи Коши $\eqref{thirdsystem}$, если
    \begin{enumerate}
    \item $x(\cdot) \in A\ C\ [t_0, t_1]$ --- абсолютно непрерывная функция.
    \item $x$ --- удовлетворяет $\eqref{thirdsystem}$ п.в. на $[t_0, t_1]$.
    \end{enumerate}
\end{defin}

\begin{defin} [Абсолютно непрерывная функция]\ \\
$x(t)$ --- абсолютно непрерывна на $[t_0,\,t_1]$, если
   \begin{align*}
    &\forall \varepsilon > 0\ \exists\, \delta (\varepsilon) > 0 : \forall\ a_j, b_j\ (t_0 \leq a_1 < b_1\ldots a_m<b_m\leq t_1) : \sum\limits_{j = 1}^{m}|b_j - a_j| < \delta\\
    &\Rightarrow \sum\limits_{j = 1}^{m}\|x(b_j) - x(a_j)\| < \varepsilon
   \end{align*}
\end{defin}

{\flushleft Введем обозначение:}
$$
\begin{cases}
t \in [t_0 - a, t_0 + a],\; a > 0\\
b > 0\,:\, \| x - x^0\| \leq b
\end{cases}
\Rightarrow \text{Пусть } \mathcal{D} = [t_0 - a, t_0 + a] \times B_b(x^0)
$$

\begin{thm}[Условия Каратеодори]\ 
\begin{enumerate}
    \item $g \text{ --- определена } \forall x \in B_b(x^0)\quad \dot\forall\, t \in [t_0 - a, t_0 + a]$
    \item $g$ --- измеримо по $t\quad \dot\forall x \in B_b(x^0)$
    \item $g \in C_x\quad \dot{\forall} t \in [t_0 - a, t_0 + a]$
    \item $\| g(t, x)\| \leq m(t)\quad \dot{\forall} t,\; \forall x,\; m \in L [t_0 - a, t_0 + a]$
\end{enumerate}
\end{thm}

\begin{thm}
Пусть выполнены условия Каратеодори и 
\begin{align*}
&\exists\,d>0,\ d\leq a\colon \varphi(t_0\pm d) \leq b\\
&\varphi(t) = \left|\int\limits_{t_0}^{t_1} m(s) ds \right|
\end{align*}
Тогда у задачи Коши $\eqref{thirdsystem}$ существует решение по Каратеодори на отрезке $[t_0-d,\,t_0+d]$.
\end{thm}
\begin{thm}
Пусть выполнены условия Каратеодори и
\begin{align*}
\langle x^2 - x^1, g(t, x^2) - g(t, x^1)\rangle \leq L(t) \|x^2 - x^1\|^2\\
\forall x^1, x^2 \in B_b(x^0)\quad \dot{\forall} t,\quad L(t) \in L[t_0 - a, t_0 + a]
\end{align*}
Тогда решение $\eqref{thirdsystem}$ единственно.
\end{thm}
%\includegraphics[scale=0.15]{9.jpg}
\ \\
\begin{rmk}
Если $(t_0 + d, x(t_0 + d)) \in int D \Rightarrow $ можно продолжить.
\end{rmk}
\begin{defin}
\ \\
    $ \overline{\tau} = \sup \left\{ t \in [t_0 - a, t_0 + a]\,\middle|\,x(s) \in D\; \dot{\forall}\,s \in [t_0, t)\right\}.$
\end{defin}
\begin{defin}
\ \\
    $ \underline{\tau} = \inf \left\{ t \in [t_0 - a, t_0 + a]\,\middle|\,x(s) \in D\; \dot{\forall}\,s \in (t, t_0]\right\}.$
\end{defin}
\begin{thm}
$
(\overline{\tau}, x(\overline{\tau})) \in \partial D.
$
\end{thm}
\section{Теорема существования ОУ для общей задачи управления с условиями типа равенства. Пример задачи, для которой ОУ не существует.}
Введем обозначение: $(t_0, x^0, t_1, x^1) = e$.\\
Сформулируем общую задачу оптимального управления c условиями типа равенство:
$$
\begin{cases}
    \dot{x}(t) = f(t, x(t), u(t))\\
    u(t) \in \mathcal{P}\\
    J = \varphi_0(e) \rightarrow inf\\
    \varphi_1(e) = ... = \varphi_k(e) = 0
\end{cases}
$$
Рассмотрим дополнительные условия и обозначим их $\maltese$:
$$
\begin{aligned}
    &f \in C_{x, u, t}\\
    &\mathcal{P} \text{--- компакт}\\
    &\|f(t, x, u)\| \leq A\|x\| +B\\
    &\|f(t, x^1, u)- f(t, x^2, u)\| \leq L \|x^1 - x^2\|\\
    &1) \ \exists M_1: \|x(t)\| \leq M_1(t_0, t_1) \quad \forall u(\cdot) \text{ --- измеримой}\\
    &2)\  \exists M_2: \|x(t)- x(s)\| \leq M_2|t-s| \quad 
\end{aligned}
$$
Введем дифференциальное включение: $$F(t, x) = \bigcup\limits_{u\in \mathcal{P}} \{f(t, x, u)\}\text{ --- вектор Грама}.$$ Тогда $\dot{x} \in F(t, x) .$
Далее сформулируем теорему существования ОУ, но перед тем как ее сформулировать выпишем некоторые обозначения:
\begin{enumerate}
\item $E = \{\left. e \right| \phi_i(e) = 0\quad \forall i\}$.
\item $S$ --- множество траекторий: $e \in E$.
\end{enumerate}
\begin{thm}[Существование ОУ]\ 
\begin{enumerate}
\item Пусть $S \neq \varnothing$;
\item $E$ --- компакт;
\item Выполнено $\maltese$, $\phi_0 \in C(E);$
\item $F \in \conv \mathbb{R}^n$
\end{enumerate}
Тогда $\exists\, (x^*(t), u^*(t)),\; t_0^*,\, t_1^*$ --- решение задачи.
\end{thm}
И, наконец, вашему вниманию будет предложен пример задачи, для которой не существует оптимального управления:
\begin{ex}
$$
\begin{aligned}
    &\dot{x}(t) = u, \quad u \in \mathcal{P};\\
    &J(x(\cdot)) = \int\limits_0^1 x^2(t) dt;\\
    &x(0) = 1, \quad x(1) = 0.\\
\end{aligned}
$$
$$
u^j = 
\left\{ 
\begin{aligned}
& -j, \quad t \in \left[0, \frac{1}{j}\right]; \\
& 0, \quad t \in \left[\frac{1}{j}, 1\right]. \\
\end{aligned}
\right| \Rightarrow \text{решения } \ \nexists
$$
\end{ex}
\begin{ex}[Контрпример на невыпуклость вектора Грама]
\begin{align*}
&\begin{cases}
\dot{x}_1 = -(x_2)^2 + u^2\\
\dot{x}_2 = u
\end{cases},\quad |u| \leq 1\\
&t_0 = 0,\quad t_1 \rightarrow \inf\\
&x^0 = (-1,0)^\mathrm{T} \rightarrow \{x_1^2+x_2^2 = a^2 \,|\, 0<a<1 \}\\
\end{align*}
\end{ex}

\begin{ex}[Контрпример на невыпуклость $E$]
\begin{gather*}
\dot{x} = u,\quad |u| \leq 1\\
t_0 = 0,\quad t_1 > 0\\
x^0 = 0 \rightarrow \frac{1}{t_1},\quad x(t_1) \rightarrow \inf
\end{gather*}
\end{ex}
\newpage
\begin{thebibliography}{0}
  \bibitem{rublev} Рублёв~И.~В. \emph{Лекции по оптимальному управлению (нелинейные системы)}. 2015.
\end{thebibliography}
\end{document}
