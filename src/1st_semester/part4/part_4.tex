%
%
%	Лекции 11-12: задача быстродействия
%
%

%
% 	Лекция 11
%
\section{Задача быстродействия}
\subsection{Постановка задачи}
Преступим к изучению следующего типа задач
оптимального управления~--- \textit{задач быстродействия}~--- задач
перевода системы из начального фиксированного
положения в конечное, также фиксированное,
положение, за минимальное время.

Пусть наша система описывается следующими условиями:
\begin{equation}\label{QT_1}
  \begin{cases}
    \dot{x}(t)= A(t)x(t) + B(t)u(t) + f(t),\\
    x(t_0) = x^0,\\
    x(t_1) = x^1,\\
    u(\tau) \in \mathcal{P}(\tau)\in\conv\mathbb{R}^{m},\\
    t_1 - t_0\to\inf,
  \end{cases}
\end{equation}
где $x_0, x_1, t_0$~--- фиксированы,
$A(t), B(t), f(t)$~--- непрерывны,
а $\mathcal{P}$ непрерывно как многозначное
отображение (это требование гарантирует нам,
что для любого $l$ $\rho(l|\mathcal{P(\tau)})$
по $\tau$ непрерывна
\footnote{В частности, при $m = 1$ множество
$\mathcal{P}$ выглядит как
$\mathcal{P} = [a(\tau); b(\tau)]$;
непрерывность многозначного отображения означает,
что $a(\tau), b(\tau)$~--- непрерывны.}).

Отметим, что отказ от требования
$u(\tau)\in \mathcal{P}(\tau)\in\conv\mathbb{R}^{m}$ возможен;
в этом случае
$\overline{\soa_{\mathcal{P}}[t_1]}=\soa_{\overline{\mathcal{P}}}[t_1]$.
Разумность такого отказа показывает следующий
\ex{
  Пусть уравнения \eqref{QT_1} имеют вид
  \begin{equation*}
    \begin{cases}
      \dot{x} = u,\\ 
      x(0) = 0,\\
      u(\tau) \in [-1,1].
    \end{cases}
  \end{equation*}
  Тогда множеством достижимости $\soa_1$ будет
  бесконечный треугольник в I и IV координатных четвертях,
  лежащий внутри прямых $x=t$ и $x=-t$.
  При этом геометрически ясно, что при замене
  множества допустимых управлений с отрезка $[-1,1]$ на
  двухточечное множество $\{-1,1\}$
  множество достижимости не изменится:
  любую точку, лежащую внутри $\soa_1$,
  можно соединить с началом координат ломанной,
  содержащей звенья, параллельные прямым $x=t$ и $x=-t$.

  Именно этот прием используется при управлении
  парусными судами при отсутствии попутного ветра
  (при этом говорят, что судно \emph{идет галсом}).
}

Введём множество достижимости 
\begin{equation*}
  \soa[t_1] = \soa(t_1, t_0, x^0) = \{x = x(t_1, t_0, x^0 | u(\cdot)),
  \qquad u(\tau) \in \mathcal{P}\}.
\end{equation*}
Введём также \emph{трубку достижимости}
как\footnote{Следует понимать,
что множество достижимости~--- это множество,
а трубка достижимости~--- это функция,
отображающая время на соответствующее
множество достижимости.} $\soa[\cdot]$.
Её графиком будем называть
множество $\soa[\cdot] = \{(t, x) : \ x\in\soa[t]\}$.

Ключевую роль играет следующее очевидное
{
  \stm{Если $t_1^* - t_0$~--- время оптимального взаимодействия,
    $x^*$, $u^*$~--- соответственно траектория и управлениe,
    отвечающие этому времени,
    то $(t_1^*, x^*(t_1^*)) \in \partial\soa[\cdot]$.
  }
}
\begin{proof}
  Достаточно заметить,
  что если $(t_1^*, x^*(t_1^*)) \notin \partial\soa[\cdot]$,
  то достаточно сместится назад во времени к
  некоему моменту $t_2^*$ такому,
  что $(t_2^*, x^*(t_1^*)) \in \partial\soa[\cdot]$
  (такая точка существует в силу выпуклости и непрерывности);
  это приводит к противоречию с тем,
  что $t_1^* - t_0$~--- время оптимального взаимодействия.
\end{proof}

Следующий пример показывает,
что в криволинейных координатах это утверждение,
вообще говоря, неверно.

\ex{
  Пусть уравнения \eqref{QT_1} записаны в
  полярных координатах и имеют вид
  \begin{equation*}
    \begin{cases}
      \dot{\rho} = u_1,\ |u_1|\leqslant 1,\\
      \dot{\varphi} = u_2,\ |u_2|\leqslant 1,\\
      \rho(0) = \rho^0 > 0,\\
      \phi(0) =\phi^0.
    \end{cases}
  \end{equation*}
  Если бы это были декартовы координаты на плоскости,
  то трубкой достижимости был бы \glqq распухающий квадрат \grqq
  $\soa[t_1] = \{|x - x^0| \leqslant t_1,\ |y - y^0| \leqslant t_2\}$.
  В нашем случае это будет \glqq распухающий кольцевой сектор \grqqq,
  и множество достижимости не будет выпуклым.
  Это приведет к тому, что если финальная точка
  будет отвечать углу $\pi$,
  то $(t^*, x^*(t^*)) \notin \partial\soa[t_1^*]$.
}

Введём функцию $\varepsilon[t_1] = d(x^1, \soa[t_1])$.
Тогда очевидно
{
  \stm{$t_1^* - t_0$~--- время оптимального взаимодействия
    $\Leftrightarrow$ $t_1^*$~--- наименьший корень уравнения
    $\varepsilon[t_1] = 0,\ t_1 \geqslant t_0.$
  }
}

При этом стоит иметь ввиду,
что если некое множество $Z$~--- компакт,
то $x \in Z\ \Leftrightarrow\ d(x,Z)=0$.

\subsection{Свойства множества достижимости}
{
  \stm{
    $\soa[t_1] \in \conv\mathbb{R}^{n}.$
  }
}
\begin{proof}
  \begin{enumerate}
    \item
    Докажем выпуклость.
    Пусть $\hat{x}_1,\ \hat{x}_2\ \in\soa[t_1]$,
    $u^1, u^2$~--- отвечающие им управления,
    $u^1,u^2 \in \mathcal{P}$;
    тогда для $j=1, 2$ по формуле Коши имеем
    \begin{equation}\label{QT_2}
      \hat{x}_j = X(t_1, t_0)x^0 +
          \int\limits_{t_0}^{t_1}X(t_1, \tau)
                                 [B(\tau)u^j(\tau) + f(\tau)]\,d\tau
    \end{equation}
    Пусть
    $\hat{x} = \lambda \hat x^1 + (1 - \lambda)\hat{x}^2$,
    $\ u(\tau) = \lambda u^1(\tau) + (1 - \lambda)u^2(\tau)$.
    Домножая первое соотношение в \eqref{QT_2} на $\lambda$,
    а второе~--- на $1 - \lambda$, и, складывая, получаем,
    что траектории $\hat{x}$ отвечает управление
    $u(\tau) \in \mathcal{P(\tau)}$
    (ибо $\mathcal{P}(\tau)$~--- выпукло),
    что и означает выпуклость $\soa[t_1]$.
    \item
    Докажем ограниченность.
    Покажем, что существует такое $c > 0$,
    что $\mathcal{P}(\tau) \subseteq c \cdot B_1(0)$.
    Так как $\rho(l|\mathcal{P}(\tau))$
    непрерывно по $\tau$, то возьмём
    $c = \max\limits_{||l||=1,\tau\in[t_0,t_1]}\rho(l|\mathcal{P}(\tau))$.
    Тогда для любых $l$ и любых $\tau\in[t_0,t_1]$ в силу
    положительной однородности опорной функции
    $\rho(l|\mathcal{P}(\tau)) \leqslant c\norm{l}$.
    Тогда в формуле Коши
    \begin{equation*}
      \hat{x} = X(t_1, t_0)x^0 +
          \int\limits_{t_0}^{t_1}X(t_1, \tau)
                                 [B(\tau)u(\tau) + f(\tau)]\,d\tau
    \end{equation*}
    все компоненты в правой части ограничены,
    что даёт ограниченность и левой части.
    \item
    Докажем замкнутость.
    Пусть $\hat{x}^j \to \hat{x}$.
    Надо доказать, что $\hat{x} \in \soa[t_1]$.
    Пусть траекториям $\hat{x}^j$ отвечают управления
    $u^j(\tau) \in \mathcal{P}(\tau)$.
    Без ограничения общности считаем,
    что\footnote{Т.е., возможно, переходя к подпоследовательностям.}
    $u^j \xrightarrow[j \rightarrow \infty]{\text{слабо в } L_2}u$. 

    Докажем, что $u(\tau) \in \mathcal{P}(\tau)$ для почти всех $\tau$.
    Для произвольных $l(\tau) \in L_2$ и почти всех $\tau$ верно 
    соотношение\footnote{Напоминаем,
    что если $A\subseteq B$,
    то $\rho(l|A) \leqslant \rho(l|B)$ для любого $l$.}:
    \begin{equation*}
      \scalar{l(\tau)}{u^j(\tau)}\leqslant\rho(l(\tau)\ |\ \mathcal{P}(\tau)).
    \end{equation*} 
    Проинтегрируем это соотношение от $t_0$ до $t_1$:
    \begin{equation*}
      \int\limits_{t_0}^{t_1}\scalar{l(\tau)}{u^j(\tau)}\,d\tau \leqslant
      \int\limits_{t_0}^{t_1}\rho(l(\tau)\ |\ \mathcal{P}(\tau))\,d\tau.
    \end{equation*} 
    Учитывая, что
    $u^j \xrightarrow[j \rightarrow \infty]{\text{слабо в } L_2}u$, получаем:
    \begin{equation}\label{QT_3}
      \int\limits_{t_0}^{t_1}\scalar{l(\tau)}{u(\tau)}\,d\tau \leqslant
      \int\limits_{t_0}^{t_1}\rho(l(\tau)\ |\ \mathcal{P}(\tau))\,d\tau.
    \end{equation} 
    Итак, предположим противное.
    Пусть существует подмножество $Z \subseteq[t_0, t_1]$ ненулевой меры,
    где $u(\tau) \notin \mathcal{P}(\tau)$.
    Тогда найдутся такие $l(\tau),\ \varepsilon > 0$ , что 
    \begin{equation*}
      \scalar{l(\tau)}{u(\tau)} >
      \rho(l(\tau)\ |\ \mathcal{P}(\tau)) + \varepsilon.
    \end{equation*}
    Заменим значения $u(\tau)$ вне $Z$ на ноль; тогда 
    \begin{equation*}
      \int\limits_{t_0}^{t_1}\scalar{l(\tau)}{u(\tau)}\,d\tau \geqslant
      \int\limits_{t_0}^{t_1}\rho(l(\tau)\ |\ \mathcal{P}(\tau))\,d\tau -
          \varepsilon\mu{Z},
    \end{equation*} 
    что противоречит \eqref{QT_3}.
    Значит, $u(\tau) \in \mathcal{P}(\tau)$. 

    Запишем формулу Коши:
    \begin{equation*}
      \hat{x}^j = X(t_1, t_0)x^0 +
                  \int\limits_{t_0}^{t_1}X(t_1, \tau)
                      [B(\tau)u^j(\tau) + f(\tau)]\,d\tau. 
    \end{equation*}
    Устремляя $j \to \infty$, получаем: 
    \begin{equation*}
      \hat{x} = X(t_1, t_0)x^0 +
                \int\limits_{t_0}^{t_1}X(t_1, \tau)
                    [B(\tau)u(\tau) + f(\tau)]\,d\tau. 
    \end{equation*}
    Так как $u(\tau) \in \mathcal{P}(\tau)$,
    то $\hat{x} \in \soa[t_1]$,
    что и означает замкнутость $\soa[t_1]$.
  \end{enumerate}
\end{proof}

Найдем опорную функцию множества достижимости:
\begin{multline}
\label{QT_4}
  \rho(l\ |\ \soa[t_1]) =
  \sup\limits_{u(\cdot)}
  \left[
    \scalar{l}{X(t_1, t_0)} +
    \int\limits_{t_0}^{t_1}\scalar{B^T(\tau)X^T(t_1, \tau)l}{u(\tau)}\,d\tau +
    \int\limits_{t_0}^{t_1}\scalar{l}{X(t_1, \tau)f(\tau)}\,d\tau
  \right] =\\
  {}= \scalar{l}{X(t_1,t_0)} +
  \int\limits_{t_0}^{t_1}\scalar{l}{X(t_1, \tau)f(\tau)}\,d\tau +
  \sup\limits_{u(\cdot)}
  \left[
    \int\limits_{t_0}^{t_1}\scalar{B^T(\tau)X^T(t_1, \tau)l}{u(\tau)}\,d\tau
  \right].
\end{multline}
Обозначим для краткости
$s(\tau) = B^T(\tau)X^T(t_1, \tau)l$.
Для дальнейшего продвижения нам потребуется следующая
\begin{lemma}
  $
    \sup\limits_{u(\cdot)}
    \left[
      \int\limits_{t_0}^{t_1}\scalar{s(\tau)}{u(\tau)}\,d\tau
    \right] =
    \int\limits_{t_0}^{t_1}
    \sup\limits_{u \in \mathcal{P}}\scalar{s(\tau)}{u}\,d\tau
  $.
\end{lemma}
\begin{proof}
  Так как $s(\tau)$~--- непрерывная функция, то
  $\sup\limits_{u \in \mathcal{P}(\tau)}\scalar{s(\tau)}{u} =
  \rho(s(\tau)\ |\ \mathcal{P}(t))$ непрерывно по $\tau$, и, следовательно,
  интегрируема.

  Рассмотрим
  $\Argmax\limits_{u(\cdot) \in \mathcal{P}(\tau)}\scalar{s(\tau)}{u} =
  \mathcal{P}^*(\tau)$. Проверим, что это многозначное
  отображение является измеримым.
  Для этого докажем его полунепрерывность
  сверху\footnote{Ибо, как известно,
  полунепрерывность есть достаточное условие измеримости.}.
  Так как полунепрерывность сверху равносильна
  замкнутости графика $\mathcal{P}^*(\tau)$,
  то нам надо показать, что из
  $\tau^j \to \tau,\ u^j \to u, \ u^j \in \mathcal{P}^*(\tau^j)$ следует,
  что $u \in \mathcal{P}^*(\tau)$. Это равносильно соотношениям
  $$
    \scalar{s(\tau^j)}{u^j} = \rho(s(\tau^j)\ |\ \mathcal{P}(\tau^j)),
  $$
  $$
    \scalar{l}{u^j} \leqslant \rho(l\ |\ \mathcal{P}^*(\tau^j)),
  $$
  для любого $l$. Тогда 
  $$
    \scalar{s(\tau)}{u} = \rho(s(\tau)\ |\ \mathcal{P}(\tau)),
  $$
  $$
    \scalar{l}{u} \leqslant \rho(l\ |\ \mathcal{P}^*(\tau)),
  $$
  что верно, и, стало быть, $u \in \mathcal{P}^*(\tau)$,
  что и дает нам замкнутость графика, следовательно, измеримость. 

  Воспользуемся \textit{леммой об измеримом селекторе}
  из курса многозначного анализа:
  \textit{если многозначное отображение $\mathcal{P}^*$ измеримо,
  то существует такая измеримая функция (селектор) $u^*(\cdot)$,
  что $u^*(\tau) \in \mathcal{P}^*(\tau)$ для почти всех $\tau$.} 

  Для этого селектора
  $\scalar{s(\tau)}{u^*(\tau)} = \rho(s(\tau)\ |\ \mathcal{P}(\tau))$,
  интегралы в условии леммы существуют,
  что влечет достижение точной верхней грани на
  $u(\tau) \in \mathcal{P}^*(\tau)$, что и требовалось доказать.
\end{proof}

Таким образом, мы можем выписать окончательный вид опорной функции:
$$
  \rho(l\ |\ \soa[t_1]) =
    \scalar{l}{X(t_1,t_0)} +
    \int\limits_{t_0}^{t_1}
      \scalar{l}{X(t_1, \tau)f(\tau)}\,d\tau +
    \int\limits_{t_0}^{t_1}
      \rho(B^T(\tau)X^T(t_1, \tau)l\ |\ \mathcal{P}(\tau))\,d\tau.
$$
%
%	Лекция 12
%

Итак, оптимальное управление доставляет максимум выражению 
$$
  \max\limits_{u \in \mathcal{P}(\tau)}
    \scalar{B^T(\tau)X^T(t_1, \tau)l}{u}.
$$
Обозначая $\psi(\tau) = X^T(t_1, \tau)l$, получим из \eqref{QT_4}:
$$
  \rho(l\ |\ \soa[t_1]) = \scalar{l}{\psi(t_0),x_0)} +
    \int\limits_{t_0}^{t_1}\scalar{\psi(\tau)}{f(\tau)}\,d\tau +
    \int\limits_{t_0}^{t_1}\rho(B^T(\tau)\psi(\tau)|\ \mathcal{P}(\tau))\,d\tau.
$$
При этом $\psi(\tau)$ называют \textit{сопряженной переменной}.
Из определения фундаментальной матрицы ясно,
что $\psi(\tau)$ удовлетворяет следующим соотношениям:
$$
  \begin{cases}
    \dot{\psi} = - A^T(\tau)\psi,\\
    \psi(t_1) = l.
  \end{cases}
$$
\subsection{Условие максимума}
Перейдем теперь непосредственно к решению задачи быстродействия.  Выпишем в терминах опорных функций условие $x^1\in\soa[t_1]$:
$$
  \scalar{l}{x^1}\leqslant\rho(l\ |\ \soa[t_1])
$$
для любого $l$, или, в терминах расстояний до множества,
$d(x^1, \soa[t_1]) = \varepsilon[t_1] = 0$.
Фиксируем произвольной число $\hat{\varepsilon}$.
Тогда верна следующая цепочка равносильных переходов:
$$
  d(x^1,\soa[t_1]) \leqslant 
  \hat{\varepsilon} \Leftrightarrow
  x^1 \in \soa[t_1] + \hat{\varepsilon}B_1(0) \Leftrightarrow
  \scalar{l}{x^1} \leqslant
  \rho(l\ |\ \soa[t_1]) + \hat{\varepsilon}\norm{l}.
$$
В силу положительной однородности левой и правой части по $l$,
последнее соотношение можно нормировать и записать в виде
$$
  \sup\limits_{\norm{l} = 1}
  \left(
    \scalar{l}{x^1} - \rho(l\ |\ \soa[t_1])
  \right) \leqslant \hat\varepsilon,
$$
откуда следует, что
$
  \varepsilon[t_1] = 
  \sup\limits_{\norm{l} = 1}
  \left(
    \scalar{l}{x^1} - \rho(l\ |\ \soa[t_1])
  \right)
$.
Таким образом, отсюда время быстродействия $t_1^*$
находится как наименьшей корень уравнения $\varepsilon[t_1^*] = 0$. 

Возьмём вектор
$
  l^0 \in \Argmax\limits_{\norm{l} = 1}
  \left(
    \scalar{l}{x^1} - \rho(l\ |\ \soa[t_1])
  \right)
$.
Тогда $\scalar{l^0}{x^1} = \rho(l^0\ |\ \soa[t_1^*])$,
что означает, что $x^1$ лежит на пересечении
опорной гиперплоскости и самого множества.
Отсюда $u^*(\tau) = u^{l_0}(\tau)$.
Таким образом, мы можем записать необходимое условие максимума:

\textit{Если $u^*$ есть управление, доставляющее оптимальное управление, то}
\begin{equation}\label{QT_PMP}
  \scalar{B^T(\tau)\psi(\tau)}{u^*(\tau)} =
  \max\limits_{u \in \mathcal{P}(\tau)}\scalar{B^T(\tau)\psi(\tau)}{u}.
\end{equation}
Естественно встает вопрос: является ли это условие достаточным?
Оказывается, что нет~--- следующий пример показывает,
что условию максимума может удовлетворять вообще любое допустимое управление!
\begin{ex}
  Рассмотрим следующую задачу быстродействия:
  \begin{equation*}
    \begin{cases}
      \dot{x}_1 = u - 1,\\
      \dot{x}_2 = u + 1,\\
      x^0 = [0,0]^T,\\
      x^1= [-1,1]^T,\\
      |u(t)| \leqslant 1.
    \end{cases}
  \end{equation*}
  В этой задаче,
  $\mathcal{P}(t) \equiv \mathcal{P} = [-1,1].$
  Найдем опорную функцию для этой задачи:
  $$
    \rho(l\ |\ \soa[t_1]) =
    \int\limits_{0}^{t_1}\scalar{l}{[-1,1]^T}\,d\tau +
    \int\limits_{0}^{t_1}\rho([1, 1]^Tl\ |\ \mathcal{P(\tau)})\,d\tau =
    t_1(l_2 - l_1) + t_1|l_1 + l_2|.
  $$
  Легко видеть,
  что это сумма опорных функций одноточечного множества и отрезка.
  C геометрической точки зрения,
  множество достижимости есть отрезок,
  соединяющий на плоскости точки $[-1,-1]^T$ и $[1,1]^T$,
  который \glqq ползает\grqq по плоскости.
  Очевидно,
  что для быстрейшего достижения точки $[-1,1]^T$
  надо \glqq ползти\grqq вверх по прямой $y = -x$.
  Тогда в момент $t^*=1$ мы достигнем финальной точки.

  Однако для нахождения оптимального управления
  нам (формально) надо было бы найти вектор-максимизатор $l_0$.
  На эту роль подходят вектора
  $\frac{1}{\sqrt{2}}[-1,1]^T$ и $\frac{1}{\sqrt{2}}[1,-1]^T$.
  Выпишем условие максимума:
  $$
    \scalar{B^Tl^0}{u^*} = \max\limits_{u \in \mathcal{P}}\scalar{B^Tl^0}{u},
  $$
  которое в нашем случае принимает вид $0 = 0$.
\end{ex}

Хотя приведенный пример показывает редкую
для линейных систем ситуацию,
стоит поставить вопрос об условиях,
позволяющих использовать условие максимума
как необходимое и достаточное условие.
\subsubsection{Условие нормальности (общности положения)}
Рассмотрим частный случай задачи \eqref{QT_1}:
пусть $A, B$~--- $\const$,
а $\mathcal{P}$~--- выпуклый многогранник с непустой внутренностью,
построенный на точках $u^1, u_2, \ldots, u_M$,
причем\footnote{Т.е. все $u_j$ \glqq существенно\grqq влияют на вид многогранника.}
$u_j \in \partial\mathcal{P},\ j=\overline{1,M}$.
Пусть $w = w^{k,l} = u^k - u^l$, где $k, l$ соединены ребром.
Потребуем, что бы выполнялось
\textit{условие нормальности} (или \textit{условие общности положения}):
\begin{equation*}
  \text{\textit{Вектора} } Bw, ABw,\ldots, A^{n - 1}Bw
  \text{\textit{ линейно независимы.}}
\end{equation*}
Отметим, что если $\mathcal{P}$ имеет вид
\glqq параллелепипеда\grqq,
$
  \mathcal{P} =
  \set{u \in \real^m}{a_i \leqslant u_i\leqslant b_i,\  i = \overline{1,m}}
$, а матрица $B$ состоит из столбцов $b^1, b^2, \ldots, b^m$,
то условие нормальности требует линейной независимости
векторов $b^i, Ab^i, \ldots, A^{n - 1}b^i$ для всех $i$,
что представляет собой в точности условие полной управляемости.

Роль этого условия раскрывает следующая
\begin{theorem}
  Если выполняется условие нормальности,
  то условию максимума удовлетворяет единственно управление.
\end{theorem}
\begin{proof}
  Покажем, что при $l^0 \neq 0$ существует и при том
  единственное $u^*(\cdot)$, удовлетворяющее \eqref{QT_PMP}.
  Предположим противное,
  пусть $\hat{u}^1, \hat{u}^2$ удовлетворяют \eqref{QT_PMP},
  и на множестве положительной меры $\hat{u}^1 \neq \hat{u}^2$.
  Так как 
  $
    \max\limits_{u\in\mathcal{P}}\scalar{B^T\psi(\tau)}{u} =
    \rho(B^T\psi(\tau)\ |\ \mathcal{P})
  $, то
  $\scalar{B^T\psi(\tau)}{\hat{u}^1 - \hat{u}^2} = 0$
  для почти всякого $\tau$.
  Это можно переписать в виде
  $$
    \scalar{B^Te^{-A^T(\tau - t_1)}l^0}{w} = 0,
  $$ что равносильно условию
  $l^{0T}e^{A(t_1 - \tau)}Bw = 0$ на некотором множестве
  положительной меры. Дифференцируя это тождество, получаем
  $$
    l^{0T}e^{A(t_1 - \tau)}Bw = 0,
  $$
  $$
    -l^{0T}e^{A(t_1 - \tau)}ABw = 0,
  $$
  $$
    \ldots \ldots
  $$
  $$
    (-1)^{n-1}l^{0T}e^{A(t_1 - \tau)}A^{n - 1}Bw = 0.
  $$
  Но, ибо $l^0 \neq 0$,
  получаем противоречие с условием нормальности.

  Покажем теперь, что, если управление $u$ удовлетворяет \eqref{QT_PMP},
  то $u \in \mathcal{P}$ почти всюду.
  Предположим противное: пусть существует интервал времени,
  на котором $B^T\psi(\tau)$ ортогонален ребру;
  но это невозможно: дифференцируя,
  как в первой части доказательства,
  соотношение $\scalar{B^T\psi(\tau)}{w} = 0$,
  мы получим противоречие с условием нормальности.
  Что и требовалось доказать.
\end{proof}
\begin{note}
  На самом деле, мы доказали,
  что условие нормальности гарантирует
  строгую выпуклость множества достижимости.
\end{note}
\begin{ex}
  Рассмотрим задачу
  $$
    \begin{cases}
      \dot{x}_1 = u_1,\ |u_1|\leqslant 1,\\
      \dot{x}_2 = u_2,\ |u_2|\leqslant1.
    \end{cases}
  $$
  Эта система вполне управляема, но не сильно вполне управляема.
  Множество достижимости в данном
  случае~--- квадрат (т.е. не строго выпуклое).
  Случай, в котором условие
  максимума выделяет единственное управление,
  бывает тогда, когда финальная точка
  оказывается на углу квадрата (проверьте!).
\end{ex}
\subsubsection{Условие управляемости при выпуклости множества $\mathcal{P}$}
\begin{theorem}
  Пусть $\mathcal{P}$ строго выпукло и имеет непустую внутренность,
  и выполнено условие полной управляемости,
  $$
    \rg[B |AB| \cdots |A^{n - 1}B] = n.
  $$
  Тогда условие максимума определяет
  оптимальное управление единственным образом.
\end{theorem}
\begin{proof}
  Максимум, очевидно, достигается в единственной точке
  в силу строгой выпуклости; осталось показать,
  что он ненулевой, т.е. что $B^T\psi(\tau) \neq 0$ на любом интервале.
  Предположим противное,
  пусть $B^T\psi(t) \equiv 0$ для
  любого\footnote{В силу аналитичности.} $t$.
  Дифференцируя это тождество и полагая $t = t_1$,
  получим противоречие с условием полной управляемости.
\end{proof}
