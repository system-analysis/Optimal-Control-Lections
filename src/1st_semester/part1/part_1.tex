\section{Введение}

Рассмотрим некоторый объект,
состояние которого в каждый момент времени описывает набор чисел
$ x =\left(x_1,x_2,\ldots,x_n\right)^\tran$;
пусть мы можем управлять этим объектом,
т.\,е. выбирать некоторый параметр,
так или иначе влияющий на состояние объекта.
Пусть поведение объекта описывается некоторыми уравнениями, например:
\begin{equation*}
  x^{k + 1} = f\left(k, x^k, u^k\right),
  \quad k \in \mathbb{Z}, \; u^k\text{~--- управление.}
\end{equation*}

Другим примером может служить система дифференциальных уравнений:
\begin{equation*}
  \dot{x} = f\left(t, x(t), u(t)\right),
  \quad t \in \mathbb{R}, \; u(t)\text{~--- управление.}
\end{equation*}

Наложим геометрические ограничения на множества допустимых управлений:
\begin{gather*}
  u^k \in \mathcal P(k) \text{ в случае дискретного времени},\\
  u(t)\in \mathcal P(t) \text{ в случае непрерывного времени}.
\end{gather*}

Рассмотрим функционал качества $J\left(u(\cdot)\right)$,
который будет оценивать <<качество>>
выбранного управления по некоторому критерию.
Нас в дальнейшем будут интересовать задачи
нахождения \emph{оптимального управления},
т.\,е. такого управления, на котором функционал
качества достигает экстремума.

Например, в \emph{задаче Майера--Больца} функционал имеет вид 
\begin{equation}
\label{IntroMayerBolz}
  J\left(u(\cdot); t_0, x^0\right) =
    \int\limits_{t_0}^{t_1} L\left(t, x(t), u(t)\right)\,dt +
    \varphi(t_1, x(t_1)).
\end{equation}

В дискретном случае функционал \eqref{IntroMayerBolz} принимает вид
\begin{equation*}
  J\left(\{u_k\}_{k = k_0}^{k_1}; k_0, x^{k_0}\right) =
    \sum\limits_{k = k_0}^{k_1 - 1} L\left(k, x^k, u^k\right) +
    \varphi(k_1, x^{k_1}).
\end{equation*}

\subsection{Сведения из дифференциальных уравнений}

\subsubsection{Условия существования решений задачи Коши}
Рассмотрим линейную систему дифференциальных уравнений
\begin{equation}
\label{IntroLinSys1}
  \begin{cases}
    \dot{x}= A(t)x(t) + B(t)u(t) + f(t),\\
    x(t_0) = x^0. 
  \end{cases}
\end{equation}

Как известно, для локального существования и единственности решения
этой системы достаточно непрерывности и
липшицевости правой части по переменной $x$.
Кроме того, справедлива следующая
\begin{theorem}
Рассмотрим уравнение
\begin{equation}
\label{IntroSample1}
  \dot{x} = f(x,t), 
\end{equation}
где $f(x,t)$ непрерывна в некоторой области $A$.
Тогда для любой компактной области из $A$ существует решение уравнения
\eqref{IntroSample1}, доходящее до её границы. 
\end{theorem}

\begin{note}
Если управление в правой части \eqref{IntroLinSys1} лишь
кусочно-непрерывная функция,
то полученное решение лишь кусочно-дифференцируемая функция.
В таком случае, задача сначала решается
на одном промежутке непрерывности правой части,
и значение решения в точке разрыва
полагается начальным условием для задачи нахождения решения
на примыкающем промежутке непрерывности.

В самом широком случае,
когда правая часть \eqref{IntroLinSys1} всего лишь измерима,
решение понимается в смысле решения по Каратеодори.
\end{note}

Следующий пример показывает,
что в условиях теоремы продолжимость решения невозможна.
\begin{ex}
Решением системы
\begin{equation*}
  \begin{cases}
	  \dot{x} = x^2,\\
	  x(0) = 1,
  \end{cases}
\end{equation*}
является функция
$x(t) = \frac{1}{1-t}$, которая является разрывной в точке $t = 1$.
\end{ex}

Понятно, что единственной причиной непродолжимости
может быть уход траекторий решений за конечное время на бесконечность.
Чтобы исключить подобные ситуации из рассмотрения потребуем,
чтобы $\norm{x(t)}$ как функция от $t$ росла
не быстрее известной ограниченной функции.
Для этого оценим скорость роста $\norm{x(t)}^2$:
\begin{equation*}
  \frac{d}{dt}\left(\norm{x}^2\right) = 
  \frac{d}{dt}\left(\vphantom{\norm{x}^2}\scalar{x(t)}{x(t)}\right) = 
  2\scalar{x(t)}{f(t,x(t))}.
\end{equation*}

Для продолжимости решений вправо потребуем, чтобы 
\begin{equation}
\label{IntroExtend}
  \scalar{x}{f(t, x)} \leqslant
  C_1 \norm{x(t)}^2 + C_2, \quad C_1, C_2 > 0. 
\end{equation}

Обозначая $y = y(t) = \norm{x(t)}^2$, получаем:
\begin{gather*}
  \dot{y} \leqslant 2C_1y + 2C_2 \Longleftrightarrow
  \dot{y} - 2C_1y \leqslant 2C_2 \quad
    \left|\text{ домножим на $e^{-2C_1t}$:} \vphantom{\frac{d}{dt}}\right.\\
  \dot{y} e^{-2C_1t} - 2 C_1 y e^{-2C_1t} \vphantom{\frac{d}{dt}}
    \leqslant 2 C_2 e^{-2C_1t},\\
  \frac{d}{dt}\left(y e^{-2C_1t}\right)
    \leqslant 2 C_2 e^{-2C_1t} \quad
    \left|\text{ интегрируем от $t_0$ до $t$:}\vphantom{\frac{d}{dt}}\right.\\
  y(t) e^{-2C_1t} - y(t_0) e^{-2C_1t_0} \vphantom{\frac{d}{dt}}
    \leqslant
    \tfrac{C_2}{C_1}\left(e^{-2 C_1 t_0} - e^{-2 C_1 t}\right),\\
  y(t) = \norm{x(t)}^2 \vphantom{\frac{d}{dt}}
    \leqslant
    e^{2 C_1 t}\left(
      \tfrac{C_2}{C_1}\left(e^{-2 C_1 t_0} - e^{-2 C_1 t}\right) +
      y(t_0) e^{-2 C_1 t_0}
               \right),
\end{gather*}
что гарантирует продолжимость решения.

Применяя неравенство Коши--Буняковского,
условие \eqref{IntroExtend} можно усилить:
\begin{equation}
\label{IntroSubLin}
\norm{f(t,x)} \leqslant C_3\norm{x} + C_4, \quad C_3, C_4 > 0.
\end{equation}

Условие \eqref{IntroExtend} называется \emph{условием продолжимости},
а \eqref{IntroSubLin}~--- \emph{условием сублинейного роста},
оно гарантирует продолжимость решений в обе стороны.

\subsubsection{Формула Коши векторного дифференциального уравнения}
Вернемся к рассмотрению задачи \eqref{IntroLinSys1}.
Рассмотрим однородное уравнение
\begin{equation}
\label{IntroLinSys2}
  \begin{cases}
    \dot{x} = A(t)x(t),\\
    x(t_{0}) = x^{0}.
  \end{cases}
\end{equation}

\begin{df}
  Матрица $X(t, \tau) \in \real^{n \times n}$ называется
  фундаментальной матрицей для уравнения \eqref{IntroLinSys2}, если
  \begin{equation*}
    \begin{cases}
      \dfrac{\partial{X}(t, \tau)}{\partial{t}} = A(t)X(t, \tau),\\
      X(\tau,\tau) = E.
    \end{cases}
  \end{equation*}
\end{df}

Пусть вектор-столбцы $x^1, x^2, \ldots, x^n$~--- фундаментальная
система решений уравнения \eqref{IntroLinSys2},
составим из них матрицу $\Phi(t)$.
При этом
\begin{equation*}
  \dfrac{d\Phi(t)}{dt} = A(t)\Phi(t).
\end{equation*}

Несложно показать, что $X(t, \tau) = \Phi(t)\Phi^{-1}(\tau)$.
Поэтому решение задачи Коши \eqref{IntroLinSys2}
представимо в виде $x(t) = X(t, t_{0})x^0$.
Это выражение так же называется
\emph{формулой Коши} для однородного уравнения.

Отметим, что для матрицы $X(t, \tau)$ выполнено
важное свойство, которым мы будем пользоваться
в дальнейшем:
\begin{equation}
\label{IntroInvariant}
  X(t + T, \tau + T) = X(t, \tau) \quad \forall T \in \mathbb{R}.
\end{equation}

Далее, рассмотрим неоднородное уравнение
\begin{equation}
\label{IntroLinSys3}
  \begin{cases}
    \dot{x} = A(t)x(t) + f(t),\\
    x(t_0)= x^0.
  \end{cases}
\end{equation}

Будем искать решения в виде $x(t) = F(t)y(t)$,
где $F(t)$~--- некоторая матрица.\\
Дифференцируя и подставляя в \eqref{IntroLinSys3}, получим:
\begin{gather*}
  \dot{x} = \dot{F}(t)y(t) + F(t) \dot{y}(t) = A(t)F(t)y(t) + f(t),\\
  \dot{y} = F^{-1}(t)(A(t)F(t) - \dot{F}(t))y(t) + F^{-1}(t)f(t).
\end{gather*}

Выбирая $F(t) = X(t, t_0)$,
мы сводим систему \eqref{IntroLinSys3} к системе
\begin{equation*}
  \begin{cases}
    \dot{y} = X^{-1}(t, t_0) f(t),\\
    y(t_0)= x^0.
  \end{cases}
\end{equation*}

Таким образом, мы можем выписать в явном виде ответ:
\begin{gather}
  y(t) = x^0 +
         \int\limits_{t_0}^{t} X^{-1}(\tau, t_0) f(\tau)\,d\tau, \notag\\
\label{IntroVecAnswer}
  x(t) = X(t, t_0) x^0 +
         \int\limits_{t_0}^{t} X(t, t_0) X^{-1}(\tau, t_0) f(\tau)\,d\tau.
\end{gather}

Отметим, что $X(\tau, t_0)$ отображает $x^0$ в $x(\tau)$,
а $X(t, \tau)$ отображает $x(\tau)$ в $x(t)$.
В таком случае $x(t) = X(t, \tau) X(\tau, t_0) x^0$.
Но в силу единственности решения верно,
что $x(t) = X(t, t_0)x^0$.
Из этих двух равенств получаем:
\begin{equation}
\label{IntroHalfGroup}
  X(t, \tau) X(\tau, t_0) = X(t, t_0).
\end{equation}

Соотношение \eqref{IntroHalfGroup} называется
\emph{полугрупповым свойством} фундаментальной матрицы.
В частности, при $t = t_0$,
получаем $X(t_0, \tau) X(\tau, t_0) = E$.
Полугрупповое свойство позволяет записать
соотношение \eqref{IntroVecAnswer} в виде
\begin{equation*}
  x(t) = X(t, t_0) x^0 + \int\limits_{t_0}^{t} X(t, \tau) f(\tau)\,d\tau,
\end{equation*}
называемом \emph{формулой Коши} для неоднородного уравнения.

Конкретно для задачи \eqref{IntroLinSys1}, формула Коши имеет вид
\begin{equation*}
  x(t) = X(t, t_0)x ^0 +
         \int\limits_{t_0}^{t} X(t, \tau) B(\tau) u(\tau)\,d\tau +
         \int\limits_{t_0}^{t} X(t, \tau) f(\tau)\,d\tau.
\end{equation*}

Зададимся теперь вопросом,
как найти $\frac{\partial{X}(t, \tau)}{\partial{\tau}}$?
Для этого запишем:
\begin{gather*}
  E = X(t, \tau) \cdot X(\tau, t),\text{ продифференцируем по } \tau:\\
  0 = \frac{\partial{X}(t, \tau)}{\partial{\tau}} X(\tau, t) +
      X(t, \tau) A(\tau) X(\tau, t),\text{ откуда получаем:}\\
  \begin{cases}
    \frac{\partial{X}(t, \tau)}{\partial{\tau}} = -X(t, \tau) A(\tau),\\
    X(t, t) = E.
  \end{cases}
\end{gather*}

Если обозначить $S(t, \tau) = X^{T}(\tau, t)$, то получим:
\begin{equation*}
  \begin{cases}
    \frac{\partial{S}(t, \tau)}{\partial{t}} = -A^{T}(t) S(t, \tau),\\
    S(t, t) = E,
  \end{cases}
\end{equation*}
если $X(t, \tau)$~--- фундаментальная матрица
для задачи Коши \eqref{IntroLinSys2},
то $S(t, \tau)$~--- фундаментальная матрица для задачи Коши
\begin{equation}
\label{IntroLinSys4}
  \begin{cases}
    \dot{s}(t) = -A^{T}(t) s(t),\\
    s(t_{0}) = s^{0}.
  \end{cases}
\end{equation}

Системы \eqref{IntroLinSys2} и \eqref{IntroLinSys4}
называются сопряженными.

\begin{problem}
  Пусть задано уравнение
  \begin{equation*}
    \dot{x} = A(t) x(t) + f(t).
  \end{equation*}
  
  Какая должна быть матрица $F$ в линейной
  замене переменных, чтобы уравнение приняло вид
  \begin{equation*}
    \dot{y} = B(t)y(t) + g(t),
  \end{equation*}
  где $B(t)$~--- заданная матрица?
\end{problem}

