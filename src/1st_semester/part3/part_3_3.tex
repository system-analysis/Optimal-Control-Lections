\subsection{Принцип максимума Понтрягина}

Рассмотрим $\psi(\tau) = X^\tran(t_1, \tau) \ell$, рассмотрим сопряжённую систему:
\begin{equation*}
  \begin{cases}
    \dot{\psi} = -A^\tran(t) \psi,\\
    \psi(t_1) = \ell.
  \end{cases}
\end{equation*}

\begin{theorem}[Принцип максимума Понтрягина]
	Если $u^*$ решает нашу задачу, то найдется ненулевой вектор $\ell^0$ и
	функция $\psi^0 \ncong 0$, что:
	\begin{equation*}
	  \begin{cases}
	    \dot{\psi^0} = -A^\tran \psi^0,\\
	    \psi^0(t_1) = \ell^0,
	  \end{cases}
	\end{equation*}
	и
	\begin{equation*}
	  \scalar{B^\tran(\tau) \psi^0(\tau)}{u^*(\tau)} \stackrel{\text{п.\,в.}}{=}
	    \underset{\norm{u} \leqslant \mu}{\max} \scalar{B^\tran(\tau) \psi^0(\tau)}{u}.
	\end{equation*}
\end{theorem}

Когда можно утверждать, что $s^0(\tau) \neq 0$ всюду?

$B$, $X$~--- хорошие, непрерывные; рассмотрим, когда $s^0(\tau) = B^\tran(\tau) X^\tran(t_1, \tau) \ell^0 = 0$.
% TODO: где-то здесь картинка
Если $s^0(\tau) = 0$ при $\tau \in \left(\Tilde{t}_0, \Tilde{t}_1\right)$,
то наша система не является вполне управляемой на $\left(\Tilde{t}_0, \Tilde{t}_1\right)$.
Тогда ПМП превращается в достаточное условие путём требования полной управляемости на любом интервале.
\begin{stm}
	Пусть матрицы $A$ и $B$ не зависят от времени, пара $(A, B)$~--- управляема.
	В таком случае $u^*$ решает нашу задачу тогда и только тогда, когда
	$\scalar{s^0(\tau)}{u^*(\tau)} \stackrel{\text{п.\,в.}}{=}
	  \underset{\norm{u} \leqslant \mu}{\max} \scalar{s^0(\tau)}{u}$.
\end{stm}

Рассмотрим два примера:
\begin{ex}[Когда всё плохо (а может, несмотря ни на что, очень даже хорошо)]
	\begin{equation*}
	  \begin{cases}
	    \dot{x}_1 = u,\\
			\dot{x}_2 = u,
	  \end{cases}
		\quad \abs{u} \leqslant 1.
	\end{equation*}
	
	\begin{equation*}
	  x^0 =
	  \begin{pmatrix}
	    0\\
	    0
	  \end{pmatrix},
	  \quad
		x^1 =
		\begin{pmatrix}
		  1\\
		  1
		\end{pmatrix},
		\quad
		A = E, \quad
		B =
		\begin{pmatrix}
		  1\\
		  1
		\end{pmatrix}
	\end{equation*}
	
	Задача моментов для данной системы: 
	$\int\limits_{t_0}^{t_1}
	  \begin{pmatrix}
	    1\\
	    1
	  \end{pmatrix}
	  u(\tau) \, d\tau =
	  \begin{pmatrix}
	    1\\
	    1
	  \end{pmatrix} =
	  \begin{pmatrix}
	    1\\
	    1
	  \end{pmatrix},
	  t_0 = 0, t_1 = 2$.

	Скалярно умножим на $\ell(\ell_1, \ell_2)$ и максимизируем:
	\begin{gather*}
	  \underset{u(\cdot)}{\max} \int\limits_{t_0}^{t_1} (\ell_1 + \ell_2) u(\tau) \, d\tau = 2 \mu (\ell_1 + \ell_2),\\
	  \sup\limits_{\ell_1 + \ell_2 \neq 0} \frac{\ell_1 + \ell_2}{2 \abs{\ell_1 + \ell_2}} = \frac{1}{2} = u.
	\end{gather*}
	
	То есть сидим и ждём момента. % FIXME: прояснить
	
	Множество достижимости является отрезком.
	% TODO: можно вставить картинку множества достижимости
\end{ex}

\begin{ex}[А вот здесь уже кажется и плохо бывает, и хорошо]
	\begin{equation*}
		\left\{
		\begin{array}{rclcl}
			\dot x_1 &=& u - 1,\\
			\dot x_2 &=& u + 1,
		\end{array}
		\right.
		\quad \abs{u} \leq 1.
	\end{equation*}
	\begin{equation*}
		x^0 = \left(
				\begin{array}{c}
					0\\
					0
				\end{array}
			  \right)
		, \quad
		x^1 = \left(
				\begin{array}{c}
					-1\\
					1
				\end{array}
			  \right).
	\end{equation*}
	\begin{multline*}
	\displaystyle\int\limits_{t_0}^{t_1} 
	\left(
		\begin{array}{c}
			1\\
			1
		\end{array}
	\right)
	u d \tau = 
	\left(
		\begin{array}{c}
			-1\\
			1
		\end{array}
	\right)
	- (t_1 - t_0)
	\left(
		\begin{array}{c}
			-1\\
			1
		\end{array}
	\right)
	=
	(1 + t_0 - t_1)
	\left(
		\begin{array}{c}
			-1\\
			1
		\end{array}
	\right)
	={}\\
	{}=\left\{ t_0 = 0; \text{Тогда при $t_1 \ne 1$ задача неразрешима; далее $t_1 = 1$}\right\} = 
	\left(
		\begin{array}{c}
			0\\
			0
		\end{array}
	\right).
	\end{multline*}
	То есть при $t_1 = 1$ задача разрешима.\\
	Если же 
	$x^1 = \left(
		\begin{array}{c}
			1\\
			0
		\end{array}
	\right),$
	то
	$
		\displaystyle\int\limits_{t_0}^{t_1} 
	\left(
		\begin{array}{c}
			1\\
			1
		\end{array}
	\right)
	u d \tau = 
	\left(
		\begin{array}{c}
			t_1 - t_0\\
			1 + t_0 - t_1
		\end{array}
	\right).
	$\\
	Считаем, что $t_0 = 0$. Система разрешима только при $t_1 = 1 - t_1$, т.\,е. при $t_1 = \dfrac{1}{2}$.
	Тогда интеграл равен 
	$
	\left(
		\begin{array}{c}
			1/2\\
			1/2
		\end{array}
	\right)
	$,
	$
		\mu^0 = \sup \dfrac{\frac{1}{2} (l_1 + l_2)}{\frac{1}{2} \abs{l_1 + l_2}}.
	$
\end{ex}