\subsection{Пространство $\Lp$, $1 < p < \infty$}

Для начала рассмотрим функциональное пространство $\Lp$ при $p > 1$ и $p < \infty$.
Норма (в случае отождествления функций с точностью до меры $0$) функции $f$ в данном
пространстве задается следующим образом:
\begin{equation}
\label{Lp_norm}
  \norm{f}_{\Lp} =
    \left(\int\limits_{t_0}^{t_1} \norm{f(t)}^p dt\right)^{\frac{1}{p}}.
\end{equation}
Ограничение на управление введем аналогично задаче в пространстве $\LTwo$:
$\norm{u}_\Lp \leqslant \mu$.

Аналогично задаче моментов \eqref{L2_moment_task}, получаем задачу моментов в данном случае.
Введем множество достижимости для данного случая:
\begin{equation}
\label{Lp_moment_task_soa}
	\soa_\mu^0 (t_1, t_0) = \soa_\mu^0 [t_1] =
	\left\lbrace
	  \alpha \in \mathbb{R}^n \colon \exists u(\cdot), \norm{u}_{\Lp} \leqslant \mu,	  
	  \int\limits_{t_0}^{t_1} H(t_1, \tau) u(\tau)\,d\tau = \alpha
	\right\rbrace.
\end{equation}

Сформулируем и докажем утверждение относительно множества достижимости.

\begin{stm}
	$\soa_\mu^0 \in \conv \mathbb{R}^n$, где $\conv \mathbb{R}^n$ есть множество
	непустых выпуклых компактов в $\mathbb{R}^n$.
\end{stm}

\begin{proof}
  Для доказательства исходного утверждения необходимо доказать
  три свойства: выпуклость, ограниченность и замкнутость.
	
	Все три свойства доказываются аналогично случаю пространства $\LTwo$.
	Отметим, что в силу рефлексивности пространства $\Lp$, единичный ша является слабым
	компактом, откуда следует замкнутость введенного множества $\soa_\mu^0$.
\end{proof}

Найдём опорную функцию множества $\soa_\mu^0$:
\begin{gather*}
  \sufu{\ell}{\soa_\mu^0[t_1]} =
    \sup_{c \in \soa_\mu^0[t_1]} \scalar{\ell}{c} =
    \sup_{u} \int\limits_{t_0}^{t_1} \scalar{\ell}{H(t_1, \tau) u(\tau)} d\tau = {}\\
  {} = \sup_{u} \int\limits_{t_0}^{t_1} \scalar{H^\tran(t_1, \tau) \ell}{u(\tau)} d\tau \leqslant
    \sup_{u} \int\limits_{t_0}^{t_1} \norm{h(\tau)} \cdot \norm{u(\tau)} d\tau \leqslant{}\\
  {} \leqslant \sup_{u}
    \left(
      \left[\int\limits_{t_0}^{t_1} \norm{h(t_1, \tau)}^q d\tau \right]^{\tfrac{1}{q}} \cdot
      \left[\int\limits_{t_0}^{t_1} \norm{u(\tau)}^p d\tau \right]^{\tfrac{1}{p}}
    \right) = {}\\
  {} = \sup_{u} \left( \norm{h}_{\Lq} \cdot \norm{u}_{\Lp} \right) =
    \mu \cdot \norm{h}_{\Lq},
\end{gather*}
где $h(\tau) = H^\tran(t_1, \tau) \ell$, значение $q$ определяется из соотношения $\frac{1}{p} + \frac{1}{q} = 1$.

\begin{note}
  При вычислении опорной функции были использованы неравенства Гёльдера
  и Коши--Буняковского.
\end{note}

%Неравенство Коши--Буняковского превращается в равенство ровно в том случае, когда
%$\norm{u(\tau)} = \lambda(\tau) \norm{H^T(t_1, \tau) l}$, где $\lambda(\tau) \ne \const$. 
%может лучше написать "не обязательно является постоянной"?
%Неравенство Гёльдера превращается в равенство ровно в том случае, когда
%$\norm{u(\tau)}^p = \Tilde \lambda \norm{H^T(t_1, \tau) l}^q, \Tilde \lambda = \const, \Tilde \lambda \geq 0$,
%т.\,е. налицо линейная зависимость двух величин. Осталось выявить зависимость
%между $\lambda(\tau), \Tilde \lambda$ и $\mu$ (а она должна быть,
%поскольку все неравенства можно превратить в равенства и наша
%цель по вычислению опорной функции будет достигнута, т.\,к. мы получим точную верхнюю грань):
%\begin{gather*}
%	\norm{u(\tau)} = \lambda(\tau) \norm{H^T(t_1, \tau) l},\\
%	\norm{u(\tau)}^p = \lambda^p(\tau) \norm{H^T(t_1, \tau) l}^p = \Tilde \lambda \norm{H^T(t_1, \tau) l}^q,\\
%	\lambda^p = \Tilde \lambda \norm{H^T(t_1, \tau) l}^{q-p} =
%	  \Tilde \lambda \norm{H^T(t_1, \tau) l}^{\frac{2p - p^2}{p - 1}},\\
%	\displaystyle\int\limits_{t_0}^{t_1} \norm{u(\tau)}^p d \tau =
%	  \Tilde \lambda \displaystyle\int\limits_{t_0}^{t_1} \norm{H^T(t_1, \tau) l}^q d \tau,\\
%	\mu^p = \Tilde \lambda \norm{H^T(t_1, \cdot) l}^q_\Lq,\\
%	\lambda^p = \mu^p \left[\dfrac{\norm{H^T(t_1, \tau) l}}{\norm{H^T(t_1, \cdot) l}_\Lq} \right]^{\frac{2p - p^2}{p - 1}}.
%\end{gather*}
%
%Тогда управление, на котором достигается максимум опорной функции в направлении $l$ вычисляется как
%$$
%	u^l(\tau) =
%  \mu \left[\dfrac{\norm{H^T(t_1, \tau) l}}{\norm{H^T(t_1, \cdot) l}_\Lq} \right]^%
%  {\frac{2p - p^2}{p - 1}} %H^T(t_1, \tau) l.
%$$

Используя условия, при которых в промежуточных неравенствах достигаются равенства, получаем
функцию $u_\ell^*(t)$, реализующую супремум:
\begin{equation*}
  u_{\ell}^{*}(t) = \mu \frac{H^\tran(t_1, t) \ell}{\norm{H^\tran(t_1, \cdot) \ell}_{\Lp}^{\frac{q}{p}}} \cdot
    \norm{H^\tran(t_1, t) \ell}^{\frac{q - p}{p}}.
\end{equation*}

Отметим, что множество $\soa_\mu^0$ строго выпукло, так как максимизатор $u_\ell^*(t)$ единственный,
правда, данное множество не является эллипсоидом, в отличие от случая пространства $\LTwo$.

Минимальное значение параметра $\mu$, при котором задача моментов разрешима,
находится аналогично случаю пространства $\LTwo$:
\begin{gather*}
	\scalar{\ell}{c} \leqslant \mu \norm{H^\tran(t_1, \cdot) \ell}_{\Lq},\\
	\mu^0 = \sup\limits_{\ell \neq 0}
	  \frac{\scalar{\ell}{c}}{\lefteqn{\norm{H^\tran(t_1, \cdot) \ell}_\Lq}\phantom{\norm{H^\tran(t_1, \cdot) \ell}_0}} =
	\sup \left\lbrace \scalar{\ell}{c} \left| \norm{H^\tran(t_1, \cdot) \ell}_{\Lq} = 1 \right. \right\rbrace.
\end{gather*}
%
% !
%
Если система вполне управляема, то $\mu^0$~--- сопряжённая норма; если нет, то есть условие разрешимости
(заметим, что сопряжённая норма в $\Lq$~--- это норма в $\Lp$). % может лучше сопряжённое пространство?

\begin{ex}[Пример разрешимой системы]
	\begin{equation*}
		A = 0, B = \const, \abs{B} \neq 0; \mu^0 = \norm{B^{-1} c}_\Lp.
	\end{equation*}
	Найдём $\ell^0$ такое, что $\ell^0 \in \Argmax \set{\scalar{\ell}{c}}{\norm{H^\tran \ell}_\Lq = 1}$.
	Тогда $u^0(\tau) = u^{\ell^0}(\tau)$.
\end{ex}