\section{Задача моментов в $\Lp$ и $\LInf$}

Рассмотрим следующую задачу:
\begin{gather*}
	\dot{x}(t) = A(t)x(t) + B(t)u(t) + f(t),\\
	x(t_0) = x^0 \longrightarrow x(t_1) = x^1,
\end{gather*}
где $t_0$, $t_1$, $x^0$, $x^1$~--- фиксированные константы.

В предыдущей главе рассматривалось ограничение на управление по норме в пространстве $\LTwo[t_0; t_1]$,
задаваемое соотношением \eqref{L2_control_conditions}. Однако, минимизирование данного функционала
не имеет особого физического смысла. В данной главе будут рассмотрены другие ограничения на управление.

\begin{df}
  Существенным супремумом функции $f(t)$ на множестве $T$ называется следующее выражение:
  \begin{equation*}
    \esssup\limits_{T} f =
    \left\lbrace\left.
      \iinf_{Z \subseteq T \vphantom{T \backslash Z}} \sup_{t \in T \backslash Z} f(t) \right|
      \mu(Z) = 0
    \right\rbrace,\quad
    \mu(\cdot)\text{~--- мера Лебега на множестве $T$.}
  \end{equation*}
\end{df}

\begin{note}
  Для непрерывных функций выполнено равенство $\sup f = \esssup f$.
\end{note}

Например, вместо функционала \eqref{L2_control_conditions} в некоторых задачах имеет смысл рассматривать
следующее ограничение на управление:
\begin{equation}
  \label{Linf_control_conditions}
  \esssup\limits_{[t_0; t_1]} \abs{u} \rightarrow \inf.
\end{equation}

В данной главе будут рассмотрены другие два функционала,
один из которых задается соотношением \eqref{Linf_control_conditions}.
Везде далее будем рассматривать функциональные пространства на отрезке $[t_0; t_1]$,
отождествляя функции, имеющие различия на множестве меры $0$.

\subsection{Пространство $\Lp$, $1 < p < \infty$}

Для начала рассмотрим функциональное пространство $\Lp$ при $p > 1$ и $p < \infty$.
Норма (в случае отождествления функций с точностью до меры $0$) функции $f$ в данном
пространстве задается следующим образом:
\begin{equation}
\label{Lp_norm}
  \norm{f}_{\Lp} =
    \left(\int\limits_{t_0}^{t_1} \norm{f(t)}^p dt\right)^{\frac{1}{p}}.
\end{equation}
Ограничение на управление введем аналогично задаче в пространстве $\LTwo$:
$\norm{u}_\Lp \leqslant \mu$.

Аналогично задаче моментов \eqref{L2_moment_task}, получаем задачу моментов в данном случае.
Введем множество достижимости для данного случая:
\begin{equation}
\label{Lp_moment_task_soa}
	\soa_\mu^0 (t_1, t_0) = \soa_\mu^0 [t_1] =
	\left\lbrace
	  \alpha \in \mathbb{R}^n \colon \exists u(\cdot), \norm{u}_{\Lp} \leqslant \mu,	  
	  \int\limits_{t_0}^{t_1} H(t_1, \tau) u(\tau)\,d\tau = \alpha
	\right\rbrace.
\end{equation}

Сформулируем и докажем утверждение относительно множества достижимости.

\begin{stm}
	$\soa_\mu^0 \in \conv \mathbb{R}^n$, где $\conv \mathbb{R}^n$ есть множество
	непустых выпуклых компактов в $\mathbb{R}^n$.
\end{stm}

\begin{proof}
  Для доказательства исходного утверждения необходимо доказать
  три свойства: выпуклость, ограниченность и замкнутость.
	
	Все три свойства доказываются аналогично случаю пространства $\LTwo$.
	Отметим, что в силу рефлексивности пространства $\Lp$, единичный ша является слабым
	компактом, откуда следует замкнутость введенного множества $\soa_\mu^0$.
\end{proof}

Найдём опорную функцию множества $\soa_\mu^0$:
\begin{gather*}
  \sufu{\ell}{\soa_\mu^0[t_1]} =
    \sup_{c \in \soa_\mu^0[t_1]} \scalar{\ell}{c} =
    \sup_{u} \int\limits_{t_0}^{t_1} \scalar{\ell}{H(t_1, \tau) u(\tau)} d\tau = {}\\
  {} = \sup_{u} \int\limits_{t_0}^{t_1} \scalar{H^\tran(t_1, \tau) \ell}{u(\tau)} d\tau \leqslant
    \sup_{u} \int\limits_{t_0}^{t_1} \norm{h(\tau)} \cdot \norm{u(\tau)} d\tau \leqslant{}\\
  {} \leqslant \sup_{u}
    \left(
      \left[\int\limits_{t_0}^{t_1} \norm{h(t_1, \tau)}^q d\tau \right]^{\tfrac{1}{q}} \cdot
      \left[\int\limits_{t_0}^{t_1} \norm{u(\tau)}^p d\tau \right]^{\tfrac{1}{p}}
    \right) = {}\\
  {} = \sup_{u} \left( \norm{h}_{\Lq} \cdot \norm{u}_{\Lp} \right) =
    \mu \cdot \norm{h}_{\Lq},
\end{gather*}
где $h(\tau) = H^\tran(t_1, \tau) \ell$, значение $q$ определяется из соотношения $\frac{1}{p} + \frac{1}{q} = 1$.

\begin{note}
  При вычислении опорной функции были использованы неравенства Гёльдера
  и Коши--Буняковского.
\end{note}

%Неравенство Коши--Буняковского превращается в равенство ровно в том случае, когда
%$\norm{u(\tau)} = \lambda(\tau) \norm{H^T(t_1, \tau) l}$, где $\lambda(\tau) \ne \const$. 
%может лучше написать "не обязательно является постоянной"?
%Неравенство Гёльдера превращается в равенство ровно в том случае, когда
%$\norm{u(\tau)}^p = \Tilde \lambda \norm{H^T(t_1, \tau) l}^q, \Tilde \lambda = \const, \Tilde \lambda \geq 0$,
%т.\,е. налицо линейная зависимость двух величин. Осталось выявить зависимость
%между $\lambda(\tau), \Tilde \lambda$ и $\mu$ (а она должна быть,
%поскольку все неравенства можно превратить в равенства и наша
%цель по вычислению опорной функции будет достигнута, т.\,к. мы получим точную верхнюю грань):
%\begin{gather*}
%	\norm{u(\tau)} = \lambda(\tau) \norm{H^T(t_1, \tau) l},\\
%	\norm{u(\tau)}^p = \lambda^p(\tau) \norm{H^T(t_1, \tau) l}^p = \Tilde \lambda \norm{H^T(t_1, \tau) l}^q,\\
%	\lambda^p = \Tilde \lambda \norm{H^T(t_1, \tau) l}^{q-p} =
%	  \Tilde \lambda \norm{H^T(t_1, \tau) l}^{\frac{2p - p^2}{p - 1}},\\
%	\displaystyle\int\limits_{t_0}^{t_1} \norm{u(\tau)}^p d \tau =
%	  \Tilde \lambda \displaystyle\int\limits_{t_0}^{t_1} \norm{H^T(t_1, \tau) l}^q d \tau,\\
%	\mu^p = \Tilde \lambda \norm{H^T(t_1, \cdot) l}^q_\Lq,\\
%	\lambda^p = \mu^p \left[\dfrac{\norm{H^T(t_1, \tau) l}}{\norm{H^T(t_1, \cdot) l}_\Lq} \right]^{\frac{2p - p^2}{p - 1}}.
%\end{gather*}
%
%Тогда управление, на котором достигается максимум опорной функции в направлении $l$ вычисляется как
%$$
%	u^l(\tau) =
%  \mu \left[\dfrac{\norm{H^T(t_1, \tau) l}}{\norm{H^T(t_1, \cdot) l}_\Lq} \right]^%
%  {\frac{2p - p^2}{p - 1}} %H^T(t_1, \tau) l.
%$$

Используя условия, при которых в промежуточных неравенствах достигаются равенства, получаем
функцию $u_\ell^*(t)$, реализующую супремум:
\begin{equation*}
  u_{\ell}^{*}(t) = \mu \frac{H^\tran(t_1, t) \ell}{\norm{H^\tran(t_1, \cdot) \ell}_{\Lp}^{\frac{q}{p}}} \cdot
    \norm{H^\tran(t_1, t) \ell}^{\frac{q - p}{p}}.
\end{equation*}

Отметим, что множество $\soa_\mu^0$ строго выпукло, так как максимизатор $u_\ell^*(t)$ единственный,
правда, данное множество не является эллипсоидом, в отличие от случая пространства $\LTwo$.

Минимальное значение параметра $\mu$, при котором задача моментов разрешима,
находится аналогично случаю пространства $\LTwo$:
\begin{gather*}
	\scalar{\ell}{c} \leqslant \mu \norm{H^\tran(t_1, \cdot) \ell}_{\Lq},\\
	\mu^0 = \sup\limits_{\ell \neq 0}
	  \frac{\scalar{\ell}{c}}{\lefteqn{\norm{H^\tran(t_1, \cdot) \ell}_\Lq}\phantom{\norm{H^\tran(t_1, \cdot) \ell}_0}} =
	\sup \left\lbrace \scalar{\ell}{c} \left| \norm{H^\tran(t_1, \cdot) \ell}_{\Lq} = 1 \right. \right\rbrace.
\end{gather*}
%
% !
%
Если система вполне управляема, то $\mu^0$~--- сопряжённая норма; если нет, то есть условие разрешимости
(заметим, что сопряжённая норма в $\Lq$~--- это норма в $\Lp$). % может лучше сопряжённое пространство?

\begin{ex}[Пример разрешимой системы]
	\begin{equation*}
		A = 0, B = \const, \abs{B} \neq 0; \mu^0 = \norm{B^{-1} c}_\Lp.
	\end{equation*}
	Найдём $\ell^0$ такое, что $\ell^0 \in \Argmax \set{\scalar{\ell}{c}}{\norm{H^\tran \ell}_\Lq = 1}$.
	Тогда $u^0(\tau) = u^{\ell^0}(\tau)$.
\end{ex}
\subsection{Пространство $\LInf$}

В зависимости от значения величины $p$, минимизация нормы управления
следующим образом соотносится с физическими характеристиками:
\begin{itemize}
	\item $p = 1$~--- минимизируется топливо (данный случай не рассматривается в курсе,
	потому что пространство $\LOne$ не рефлексивно, следовательно, задача может не иметь решения).
	
	\item $p = 2$~--- минимизируется энергия.
	
	\item $p = \infty$~--- минимизируется сила.
\end{itemize}

В данном разделе рассматривается последний случай.
Норма в пространстве $\LInf$ определяется следующим образом:
\begin{equation*}
	\norm{u}_\LInf = \esssup_{[t_0; t_1]} \abs{u}.
\end{equation*}

По аналогии с введенными ранее множествами, рассмотрим множество достижимости
$\soa_\mu^0$ для данного случая:
\begin{equation}
\label{LInf_moment_task_soa}
	\soa_\mu^0 (t_1, t_0) = \soa_\mu^0 [t_1] =
	\left\lbrace
	  \alpha \in \mathbb{R}^n \colon \exists u(\cdot), \norm{u}_{\LInf} \leqslant \mu,	  
	  \int\limits_{t_0}^{t_1} H(t_1, \tau) u(\tau)\,d\tau = \alpha
	\right\rbrace.
\end{equation}

Сформулируем и докажем утверждение относительно множества достижимости.

\begin{stm}
	$\soa_\mu^0 \in \conv \mathbb{R}^n$, где $\conv \mathbb{R}^n$ есть множество
	непустых выпуклых компактов в $\mathbb{R}^n$.
\end{stm}

\begin{proof}
  Для доказательства исходного утверждения необходимо доказать
  три свойства: выпуклость, ограниченность и замкнутость.
  
	\emph{Выпуклость}: Доказывается аналогично случаю $\Lp$.
	
	\emph{Ограниченность}: Доказывается аналогично случаю $\Lp$.
	
	\emph{Замкнутость}: Единичный шар в $\LInf$, вообще говоря, не является слабо компактным.
	Рассмотрим последовательность функций $u^j$:
	\begin{equation*}
	  u^j \in \LInf, \quad \norm{u^j} \leqslant \mu, \quad
	  c^j = \int\limits_{t_0}^{t_1} X(t_1, \tau) B(\tau) u^j(\tau) \, d\tau.
	\end{equation*}

	Тогда $u^j \in \LTwo$ и $\norm{u^j}_\LTwo \leqslant \mu \abs{t_1 - t_0}$.
	В пространстве $\LTwo$ последовательность $u^j$ имеет слабый предел:
	$u^j \xrightarrow[\LTwo, j \rightarrow \infty]{\text{слабо}}u^0$.
	По теореме Лебега предел $u^0$ тоже ограничен: $\norm{u^0} \leqslant \mu$.
	Ещё заметим, что произведение $X(t_1, \tau) B(\tau)$ непрерывно, если функция $B(\tau)$ непрерывна.
	В итоге:
	\begin{equation*}
	  u^0 \in \LInf, \quad
	  c^j \xrightarrow[j \rightarrow \infty]{} c = \displaystyle\int\limits_{t_0}^{t_1}
		  X(t_1, \tau) B(\tau) u^0(\tau) \, d\tau.
	\end{equation*}
	
	Таким образом, утверждение полностью доказано.
\end{proof}

Найдём опорную функцию множества достижимости (считаем, что внутренняя норма $\norm{u(\tau)}$ евклидова):
\begin{gather*}
	\sufu{\ell}{\soa^0_\mu[t_1]} =
	  \sup\limits_{u} \int\limits_{t_0}^{t_1}
	    \scalar{\underbrace{B^\tran(\tau) X^\tran(t_1, \tau) \ell}_{s(\tau)}}{u(\tau)} d \tau
	  \stackrel{\text{К.--Б.}}{\leqslant} \sup_{u}
	    \int\limits_{t_0}^{t_1} \norm{s(\tau)} \cdot \norm{u(\tau)} d\tau \leqslant{}\\
	\phantom{\sufu{\ell}{\soa^0_\mu[t_1]}}
	{}\leqslant \mu \int\limits_{t_0}^{t_1} \norm{s(\tau)} d\tau = \mu \norm{s(\cdot)}_\LOne.
\end{gather*}

Найдём максимизатор:
\begin{equation*}
	u^\ell(\tau) = \lambda(\tau) B^\tran(\tau) X^\tran(t_1, \tau) \ell, \quad \lambda(\tau) \geqslant 0.
\end{equation*}

Хотим показать, что для п.\,в. $\tau \in \set{t}{\norm{s(t)} \neq 0}$ верно,
что $\norm{u(\tau)} = \mu$. Предположим обратное.
Тогда $\exists A \subseteq \set{t}{\norm{s(t)} \neq 0} \colon \mu(A) \neq 0,
\forall \tau \in A \norm{u(\tau)} \leqslant \mu - \varepsilon$.
Разбиваем исходный интеграл на два: на множестве $A$ и на дополнении $A$.
Тогда он на множестве меры $\mu(A)$ больше максимума.
Получили противоречие.

Итак, $u^\ell(\tau) = \mu \dfrac{B^\tran(\tau) X^\tran(t_1, \tau) \ell}{\norm{B^\tran(\tau) X^\tran(t_1, \tau) \ell}}$.

В конечном счёте,
\begin{equation*}
	u^\ell(\tau) =
	\begin{cases}
	  \mu \dfrac{s(\tau)}{\norm{s(\tau)}}, & s(\tau) \neq 0,\\
		\text{любое}, & s(\tau) = 0.
	\end{cases}
\end{equation*}

Потенциально, максимум может достигаться не в одной точке.

Тогда
\begin{equation}
  \mu^0 = \sup\limits_{\ell \neq 0}
    \dfrac{\scalar{\ell}{c}}{\int\limits_{t_0}^{t_1} \norm{s(\tau)} d\tau}, \quad
    \ell^0 \in \underset{\text{не $(\norm{s(\tau)} \stackrel{\text{п.\,в.}}{=} 0)$}}{\Argmax}
    \dfrac{\scalar{l}{c}}{\int\limits_{t_0}^{t_1} \norm{s(\tau)} d \tau}.  
\end{equation}
% FIXME: если взглянуть на то, что выше в pdf-файле, то будет видна чёткая вертикаль, что не очень хорошо смотрится

Необходимое условие оптимальности:
$s^0(\tau) = B^\tran(\tau) X^\tran(t_1, \tau) \ell^0$.
Если $u^*$ решает задачу, то $\forall \tau \colon s^0(\tau) \neq 0,
u^*(\tau) = \dfrac{s^0(\tau)}{\mu \norm{s^0(\tau)}}$ (утверждение в обратную сторону, вообще говоря, неверно).
Или же $u^*(\tau) \in \underset{\norm{u} \leqslant \mu}{\Argmax} \scalar{s^0(\tau)}{u}$,
т.\,е. $\scalar{s^0(\tau)}{u^*(\tau)} = \max\limits_{\norm{u} \leqslant \mu} \scalar{s^0(\tau)}{u}$. 
% почти что п.м.Понтрягина!!!
\subsection{Принцип максимума Понтрягина}

Рассмотрим $\psi(\tau) = X^\tran(t_1, \tau) \ell$, рассмотрим сопряжённую систему:
\begin{equation*}
  \begin{cases}
    \dot{\psi} = -A^\tran(t) \psi,\\
    \psi(t_1) = \ell.
  \end{cases}
\end{equation*}

\begin{theorem}[Принцип максимума Понтрягина]
	Если $u^*$ решает нашу задачу, то найдется ненулевой вектор $\ell^0$ и
	функция $\psi^0 \ncong 0$, что:
	\begin{equation*}
	  \begin{cases}
	    \dot{\psi^0} = -A^\tran \psi^0,\\
	    \psi^0(t_1) = \ell^0,
	  \end{cases}
	\end{equation*}
	и
	\begin{equation*}
	  \scalar{B^\tran(\tau) \psi^0(\tau)}{u^*(\tau)} \stackrel{\text{п.\,в.}}{=}
	    \underset{\norm{u} \leqslant \mu}{\max} \scalar{B^\tran(\tau) \psi^0(\tau)}{u}.
	\end{equation*}
\end{theorem}

Когда можно утверждать, что $s^0(\tau) \neq 0$ всюду?

$B$, $X$~--- хорошие, непрерывные; рассмотрим, когда $s^0(\tau) = B^\tran(\tau) X^\tran(t_1, \tau) \ell^0 = 0$.
% TODO: где-то здесь картинка
Если $s^0(\tau) = 0$ при $\tau \in \left(\Tilde{t}_0, \Tilde{t}_1\right)$,
то наша система не является вполне управляемой на $\left(\Tilde{t}_0, \Tilde{t}_1\right)$.
Тогда ПМП превращается в достаточное условие путём требования полной управляемости на любом интервале.
\begin{stm}
	Пусть матрицы $A$ и $B$ не зависят от времени, пара $(A, B)$~--- управляема.
	В таком случае $u^*$ решает нашу задачу тогда и только тогда, когда
	$\scalar{s^0(\tau)}{u^*(\tau)} \stackrel{\text{п.\,в.}}{=}
	  \underset{\norm{u} \leqslant \mu}{\max} \scalar{s^0(\tau)}{u}$.
\end{stm}

Рассмотрим два примера:
\begin{ex}[Когда всё плохо (а может, несмотря ни на что, очень даже хорошо)]
	\begin{equation*}
	  \begin{cases}
	    \dot{x}_1 = u,\\
			\dot{x}_2 = u,
	  \end{cases}
		\quad \abs{u} \leqslant 1.
	\end{equation*}
	
	\begin{equation*}
	  x^0 =
	  \begin{pmatrix}
	    0\\
	    0
	  \end{pmatrix},
	  \quad
		x^1 =
		\begin{pmatrix}
		  1\\
		  1
		\end{pmatrix},
		\quad
		A = E, \quad
		B =
		\begin{pmatrix}
		  1\\
		  1
		\end{pmatrix}
	\end{equation*}
	
	Задача моментов для данной системы: 
	$\int\limits_{t_0}^{t_1}
	  \begin{pmatrix}
	    1\\
	    1
	  \end{pmatrix}
	  u(\tau) \, d\tau =
	  \begin{pmatrix}
	    1\\
	    1
	  \end{pmatrix} =
	  \begin{pmatrix}
	    1\\
	    1
	  \end{pmatrix},
	  t_0 = 0, t_1 = 2$.

	Скалярно умножим на $\ell(\ell_1, \ell_2)$ и максимизируем:
	\begin{gather*}
	  \underset{u(\cdot)}{\max} \int\limits_{t_0}^{t_1} (\ell_1 + \ell_2) u(\tau) \, d\tau = 2 \mu (\ell_1 + \ell_2),\\
	  \sup\limits_{\ell_1 + \ell_2 \neq 0} \frac{\ell_1 + \ell_2}{2 \abs{\ell_1 + \ell_2}} = \frac{1}{2} = u.
	\end{gather*}
	
	То есть сидим и ждём момента. % FIXME: прояснить
	
	Множество достижимости является отрезком.
	% TODO: можно вставить картинку множества достижимости
\end{ex}

\begin{ex}[А вот здесь уже кажется и плохо бывает, и хорошо]
	\begin{equation*}
		\left\{
		\begin{array}{rclcl}
			\dot x_1 &=& u - 1,\\
			\dot x_2 &=& u + 1,
		\end{array}
		\right.
		\quad \abs{u} \leq 1.
	\end{equation*}
	\begin{equation*}
		x^0 = \left(
				\begin{array}{c}
					0\\
					0
				\end{array}
			  \right)
		, \quad
		x^1 = \left(
				\begin{array}{c}
					-1\\
					1
				\end{array}
			  \right).
	\end{equation*}
	\begin{multline*}
	\displaystyle\int\limits_{t_0}^{t_1} 
	\left(
		\begin{array}{c}
			1\\
			1
		\end{array}
	\right)
	u d \tau = 
	\left(
		\begin{array}{c}
			-1\\
			1
		\end{array}
	\right)
	- (t_1 - t_0)
	\left(
		\begin{array}{c}
			-1\\
			1
		\end{array}
	\right)
	=
	(1 + t_0 - t_1)
	\left(
		\begin{array}{c}
			-1\\
			1
		\end{array}
	\right)
	={}\\
	{}=\left\{ t_0 = 0; \text{Тогда при $t_1 \ne 1$ задача неразрешима; далее $t_1 = 1$}\right\} = 
	\left(
		\begin{array}{c}
			0\\
			0
		\end{array}
	\right).
	\end{multline*}
	То есть при $t_1 = 1$ задача разрешима.\\
	Если же 
	$x^1 = \left(
		\begin{array}{c}
			1\\
			0
		\end{array}
	\right),$
	то
	$
		\displaystyle\int\limits_{t_0}^{t_1} 
	\left(
		\begin{array}{c}
			1\\
			1
		\end{array}
	\right)
	u d \tau = 
	\left(
		\begin{array}{c}
			t_1 - t_0\\
			1 + t_0 - t_1
		\end{array}
	\right).
	$\\
	Считаем, что $t_0 = 0$. Система разрешима только при $t_1 = 1 - t_1$, т.\,е. при $t_1 = \dfrac{1}{2}$.
	Тогда интеграл равен 
	$
	\left(
		\begin{array}{c}
			1/2\\
			1/2
		\end{array}
	\right)
	$,
	$
		\mu^0 = \sup \dfrac{\frac{1}{2} (l_1 + l_2)}{\frac{1}{2} \abs{l_1 + l_2}}.
	$
\end{ex}