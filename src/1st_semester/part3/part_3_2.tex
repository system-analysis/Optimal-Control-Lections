\subsection{Пространство $\LInf$}

В зависимости от значения величины $p$, минимизация нормы управления
следующим образом соотносится с физическими характеристиками:
\begin{itemize}
	\item $p = 1$~--- минимизируется топливо (данный случай не рассматривается в курсе,
	потому что пространство $\LOne$ не рефлексивно, следовательно, задача может не иметь решения).
	
	\item $p = 2$~--- минимизируется энергия.
	
	\item $p = \infty$~--- минимизируется сила.
\end{itemize}

В данном разделе рассматривается последний случай.
Норма в пространстве $\LInf$ определяется следующим образом:
\begin{equation*}
	\norm{u}_\LInf = \esssup_{[t_0; t_1]} \abs{u}.
\end{equation*}

По аналогии с введенными ранее множествами, рассмотрим множество достижимости
$\soa_\mu^0$ для данного случая:
\begin{equation}
\label{LInf_moment_task_soa}
	\soa_\mu^0 (t_1, t_0) = \soa_\mu^0 [t_1] =
	\left\lbrace
	  \alpha \in \mathbb{R}^n \colon \exists u(\cdot), \norm{u}_{\LInf} \leqslant \mu,	  
	  \int\limits_{t_0}^{t_1} H(t_1, \tau) u(\tau)\,d\tau = \alpha
	\right\rbrace.
\end{equation}

Сформулируем и докажем утверждение относительно множества достижимости.

\begin{stm}
	$\soa_\mu^0 \in \conv \mathbb{R}^n$, где $\conv \mathbb{R}^n$ есть множество
	непустых выпуклых компактов в $\mathbb{R}^n$.
\end{stm}

\begin{proof}
  Для доказательства исходного утверждения необходимо доказать
  три свойства: выпуклость, ограниченность и замкнутость.
  
	\emph{Выпуклость}: Доказывается аналогично случаю $\Lp$.
	
	\emph{Ограниченность}: Доказывается аналогично случаю $\Lp$.
	
	\emph{Замкнутость}: Единичный шар в $\LInf$, вообще говоря, не является слабо компактным.
	Рассмотрим последовательность функций $u^j$:
	\begin{equation*}
	  u^j \in \LInf, \quad \norm{u^j} \leqslant \mu, \quad
	  c^j = \int\limits_{t_0}^{t_1} X(t_1, \tau) B(\tau) u^j(\tau) \, d\tau.
	\end{equation*}

	Тогда $u^j \in \LTwo$ и $\norm{u^j}_\LTwo \leqslant \mu \abs{t_1 - t_0}$.
	В пространстве $\LTwo$ последовательность $u^j$ имеет слабый предел:
	$u^j \xrightarrow[\LTwo, j \rightarrow \infty]{\text{слабо}}u^0$.
	По теореме Лебега предел $u^0$ тоже ограничен: $\norm{u^0} \leqslant \mu$.
	Ещё заметим, что произведение $X(t_1, \tau) B(\tau)$ непрерывно, если функция $B(\tau)$ непрерывна.
	В итоге:
	\begin{equation*}
	  u^0 \in \LInf, \quad
	  c^j \xrightarrow[j \rightarrow \infty]{} c = \displaystyle\int\limits_{t_0}^{t_1}
		  X(t_1, \tau) B(\tau) u^0(\tau) \, d\tau.
	\end{equation*}
	
	Таким образом, утверждение полностью доказано.
\end{proof}

Найдём опорную функцию множества достижимости (считаем, что внутренняя норма $\norm{u(\tau)}$ евклидова):
\begin{gather*}
	\sufu{\ell}{\soa^0_\mu[t_1]} =
	  \sup\limits_{u} \int\limits_{t_0}^{t_1}
	    \scalar{\underbrace{B^\tran(\tau) X^\tran(t_1, \tau) \ell}_{s(\tau)}}{u(\tau)} d \tau
	  \stackrel{\text{К.--Б.}}{\leqslant} \sup_{u}
	    \int\limits_{t_0}^{t_1} \norm{s(\tau)} \cdot \norm{u(\tau)} d\tau \leqslant{}\\
	\phantom{\sufu{\ell}{\soa^0_\mu[t_1]}}
	{}\leqslant \mu \int\limits_{t_0}^{t_1} \norm{s(\tau)} d\tau = \mu \norm{s(\cdot)}_\LOne.
\end{gather*}

Найдём максимизатор:
\begin{equation*}
	u^\ell(\tau) = \lambda(\tau) B^\tran(\tau) X^\tran(t_1, \tau) \ell, \quad \lambda(\tau) \geqslant 0.
\end{equation*}

Хотим показать, что для п.\,в. $\tau \in \set{t}{\norm{s(t)} \neq 0}$ верно,
что $\norm{u(\tau)} = \mu$. Предположим обратное.
Тогда $\exists A \subseteq \set{t}{\norm{s(t)} \neq 0} \colon \mu(A) \neq 0,
\forall \tau \in A \norm{u(\tau)} \leqslant \mu - \varepsilon$.
Разбиваем исходный интеграл на два: на множестве $A$ и на дополнении $A$.
Тогда он на множестве меры $\mu(A)$ больше максимума.
Получили противоречие.

Итак, $u^\ell(\tau) = \mu \dfrac{B^\tran(\tau) X^\tran(t_1, \tau) \ell}{\norm{B^\tran(\tau) X^\tran(t_1, \tau) \ell}}$.

В конечном счёте,
\begin{equation*}
	u^\ell(\tau) =
	\begin{cases}
	  \mu \dfrac{s(\tau)}{\norm{s(\tau)}}, & s(\tau) \neq 0,\\
		\text{любое}, & s(\tau) = 0.
	\end{cases}
\end{equation*}

Потенциально, максимум может достигаться не в одной точке.

Тогда
\begin{equation}
  \mu^0 = \sup\limits_{\ell \neq 0}
    \dfrac{\scalar{\ell}{c}}{\int\limits_{t_0}^{t_1} \norm{s(\tau)} d\tau}, \quad
    \ell^0 \in \underset{\text{не $(\norm{s(\tau)} \stackrel{\text{п.\,в.}}{=} 0)$}}{\Argmax}
    \dfrac{\scalar{l}{c}}{\int\limits_{t_0}^{t_1} \norm{s(\tau)} d \tau}.  
\end{equation}
% FIXME: если взглянуть на то, что выше в pdf-файле, то будет видна чёткая вертикаль, что не очень хорошо смотрится

Необходимое условие оптимальности:
$s^0(\tau) = B^\tran(\tau) X^\tran(t_1, \tau) \ell^0$.
Если $u^*$ решает задачу, то $\forall \tau \colon s^0(\tau) \neq 0,
u^*(\tau) = \dfrac{s^0(\tau)}{\mu \norm{s^0(\tau)}}$ (утверждение в обратную сторону, вообще говоря, неверно).
Или же $u^*(\tau) \in \underset{\norm{u} \leqslant \mu}{\Argmax} \scalar{s^0(\tau)}{u}$,
т.\,е. $\scalar{s^0(\tau)}{u^*(\tau)} = \max\limits_{\norm{u} \leqslant \mu} \scalar{s^0(\tau)}{u}$. 
% почти что п.м.Понтрягина!!!