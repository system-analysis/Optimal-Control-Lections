\subsection{Периодические матрицы}

\begin{df}
  Матрица $X(T, 0)$ называется \emph{матрицей монодромии},
  а её собственные значения~--- \emph{мультипликаторами}.
\end{df}

\begin{stm}
  Для того, чтобы $\rho$ являлся мультипликатором
  системы \eqref{IntroPeriodicDiffeq},
  необходимо и достаточно, чтобы нашлось
  ненулевое решение $x(t)$ системы
  \eqref{IntroPeriodicDiffeq},
  удовлетворяющее соотношению $x(t + T) = \rho x(t)\;\forall t$.
\end{stm}

\begin{proof}
\emph{Необходимость}:
  Если $\rho$~--- собственное значение матрицы $X(T, 0)$,
  то $\exists \upsilon \neq 0$~--- собственный вектор $X(T, 0)$: 
  \begin{equation*}
	  \Phi(T) \upsilon = \rho \upsilon.
  \end{equation*}

  Пусть $x(t)$~--- решение \eqref{IntroPeriodicDiffeq}
  при условии $x(0) = \upsilon$. Тогда
  \begin{equation*}
	  x(t + T) = \Phi(t + T) \upsilon =
	    \Phi(t)\Phi(T) \upsilon = \rho \Phi(t) \upsilon = \rho x(t).
  \end{equation*}
  
\emph{Достаточность}:
  Пусть $x(t + T) = \rho x(t)\;\forall t$. Тогда
  \begin{gather*}
	  x(t + T) = \Phi(t + T) x(0) = \Phi(t) \Phi(T) x(0),\\
	  x(t + T) = \rho x(t) = \Phi(t) \rho x(0).
  \end{gather*}
  
  $\Phi(t)$ невырождена, следовательно,
  $x(0)$~--- собственный вектор $\Phi(T)$,
  а $\rho$~--- собственное значение $\Phi(T)$.
\end{proof}

Отсюда следует, что периодическое решение
системы \eqref{IntroPeriodicDiffeq} существует
тогда и только тогда, когда у соответствующей
матрицы монодромии существует единичный мультипликатор.

\begin{theorem}[Флоке]
  Для всякой системы \eqref{IntroPeriodicDiffeq}
  с периодической матрицей найдутся такие
  матрицы $\Psi(t)$ и $\bar{A} = \const$,
  что
  \begin{gather}
    \Psi(t + T) = \Psi(t) \; \forall t,\notag\\
    \abs{\Psi} \neq 0,\notag\\
  \label{IntroFlockeEq} 
	  \Phi(t) = \Psi(t) e^{\bar{A} t}.
  \end{gather}
\end{theorem}

\begin{proof}
  Конструктивно построим такие $\Psi(t)$ и $\bar{A}$.

  Так как $\Phi(t) = X(t, 0)$,
  то $\dot{\Phi} = A(t) \Phi$,
  и, если равенство \eqref{IntroFlockeEq} выполняется, то
  \begin{gather*}
	  \dot{\Phi} = \dot{\Psi} e^{\bar{A} t} +
	    \Psi(t) \bar{A} e^{\bar{A} t} =
	    A(t) \Psi(t) e^{\bar{A} t},\text{ откуда}\\
	  \dot{\Psi} = A(t) \Psi(t) - \Psi(t) \bar{A}.
  \end{gather*}
  
  Потребуем, чтобы выполнялись соотношения
  \begin{equation*}
    \Phi(T) = \Psi(T) e^{\bar{A} T} = \Psi(0) e^{\bar{A}T}\qquad
      \text{и} \qquad
    \Psi(0) = E.
  \end{equation*}

  Тогда $\bar{A}$ найдётся из условия
  $\Phi(T) = e^{\bar{A} T}$.

  Таким образом, для того чтобы найти
  матрицу $\bar{A}$ нам необходимо
  \glqq прологарифмировать\grqq{} матрицу.
  В курсе линейной алгебры такая операция не рассматривалась,
  однако мы можем ввести её в полной аналогии с вещественными числами.
  Например, логарифм вещественных чисел можно вводить как сумму ряда:
  \begin{equation*}
    \ln(1+z) = \sum_{k=1}^\infty \dfrac{(-1)^{k-1} z^k}{k}.
  \end{equation*}
  Соответственно для матриц положим по определению
  \begin{equation*}
	  \ln \Phi(t) \eqdef \sum_{k=1}^\infty \dfrac{(-1)^{k-1}}{k} (\Phi(t) - E).
  \end{equation*}
  Ниже будет показано, что этот ряд сходится.
  Тот факт, что $e^{\ln z} = z$ легко проверяется
  с использованием свойств вещественных рядов.

  Итак, $\bar{A} = \dfrac{\ln{\Phi(T)}}{T}$.
  Положим $\Psi(t) = \Phi(t) e^{-\bar{A}t}$.
  Проверим периодичность матрицы $\Psi(t)$:
  учитывая, что $\Phi(T)e^{-\bar{A}T} = \Psi(0) = E$, получим
  \begin{equation*}
	  \Psi(t+T) =
	    \Phi(t+T) e^{-\bar{A} (t+T)} =
	    \Phi(t) \Phi(T) e^{-\bar{A}T} e^{-\bar{A}t} =
	    \Phi(t) e^{-\bar{A}t} = \Psi(t).
  \end{equation*}
  Требуемое в условии равенство тоже, очевидно, выполняется:
  \begin{equation*}
	  \Psi(t) e^{\bar{A}t} = \Phi(t)e^{-\bar{A}t}e^{\bar{A}t} = \Phi(t).
  \end{equation*}
  Таким образом, все утверждения теоремы справедливы, и теорема доказана.
\end{proof}

\subparagraph*{Сходимость матричного логарифма.}
Покажем, что ряд в определении матричного логарифма всегда сходится:
%Список смотрится не очень... Чего бы сюда запихнуть?
%\begin{itemize}
%\item 

Если $A$~--- матрица простой структуры,
т.\,е. $A = T^{-1}\Lambda T$, то $\ln A = T^{-1} \cdot \ln\Lambda \cdot T$, а 
\begin{equation*}
	\ln \left[
	\begin{array}{rcl}
		\lambda_1	& 		& 		\\
 					&\ddots & 		\\
  					& 		& \lambda_n
	\end{array}
	\right] = \left[
	\begin{array}{rcl}
	\ln\lambda_1	& 		& 		\\
					&\ddots	& 		\\
  					& 		& \ln\lambda_n
	\end{array}
	\right],
\end{equation*}
где $\ln\lambda_k$, вообще говоря, комплексные числа.

%\item 
Если $A$~--- произвольная, то 
\begin{equation*}
	A = T^{-1} \left[
	\begin{array}{rcl}
		L_1(\lambda_1)	&		& 		\\
						&\ddots & 		\\
	  					& 		& L_n\lambda_n
	\end{array}
	\right] T,
\end{equation*}
где $L_j$~--- жордановы ящики; прологарифмируем поблочно: учитывая 
\begin{equation*}
	\ln (\lambda_j + x) = 
	\ln\lambda_j + \sum_{k=1}^\infty {\dfrac{(-1)^{k-1}}{k\lambda_j^k} x^k},
\end{equation*} 
получим 
\begin{equation*}
	\ln L_j(\lambda_j) = 
	\ln \left( \lambda_j E + L_j(\lambda_j) - \lambda_j E \right) = 
	\ln\lambda_j E +
	  \sum_{k=1}^\infty{\dfrac{(-1)^{k-1}}{k\lambda_j^k}
	  \left[L_j(\lambda_j) - \lambda_j E \right]^k}.
\end{equation*}
Т.\,к $\left[L_j(\lambda_j) - \lambda_j E \right]$~--- нильпотентная матрица,
то ряд в правой части обращается в~конечную сумму,
а, следовательно, исходный ряд сходится, что и требовалось.
%\end{itemize}

\subparagraph{Сингулярное разложение матрицы.}	%% В этом абзаце всё друг на друга наезжает... надо как-то переделать. Это раз. 
%я не понимаю, что этот абзац хочет нам сказать --- это два.
Это представление нам нужно для выявления качественных свойств системы.
Проведём сингулярное разложение матрицы $A$ из \eqref{IntroPeriodicDiffeq}:
$A=U\Lambda V$. Тогда
$$
  e^A = \sum_{k=0}^\infty {\dfrac{A^k}{k!}} =
  \sum_{k=0}^\infty {\dfrac{1}{k!} U\Lambda V U\Lambda V \dots U\Lambda V}.
$$

Если $A=A^T$, то $U = V^T$, т.\,к. $U^{-1} = U^T$,
то $e^A = Ue^{\Lambda} U^T$ и о собственных
значениях $e^A$ можно судить по $e^\Lambda$.
В теории устойчивости и стабилизации это позволяет
судить о поведении системы по собственным
значениям $\Phi(t)$ при отсутствии необходимости
знания самой $\Phi(t)$ в явном виде.