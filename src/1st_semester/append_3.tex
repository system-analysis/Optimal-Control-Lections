\subsubsection{Сведения из выпуклого анализа}

\begin{df}
  Пусть $X$~--- пространство с введённым скалярным произведением,
  $\ell \in X$, $A \subset X$.
  Тогда \emph{опорной функцией множества $A$}
  называется функция
  \begin{equation*}
    \sufu{\ell}{A} = \sup_{x \in A}\scalar{\ell}{x}.
  \end{equation*}
\end{df}

Геометрический смысл опорной функции достаточно прост:
при фиксированном $\ell$ множество
$\left\lbrace \ell \mid \scalar{\ell}{z} = c = \const \right\rbrace$
есть гиперплоскости, ортогональные $\ell$,
сдвинутые от начала координат вдоль $\ell$ на $\frac{c}{\norm{\ell}}$.

Если $\norm{\ell} = 1$, то $c = \scalar{\ell}{z}$ есть
расстояние от начала координат до гиперплоскости,
ортогональной $\ell$ и проходящей через $z$.

Получается, что опорная функция множества показывает
максимальное расстояние от начала координат до гиперплоскости
заданной ориентации, ещё имеющей какие-то общие точки с нашим множеством.
Эта наиболее удалённая гиперплоскость
называется \emph{опорной гиперплоскостью}
$\pi_\ell = \left\lbrace z \mid \scalar{\ell}{z} = \sufu{\ell}{Z} \right\rbrace,
(\ell \not = 0)$.

Опорная функция обладает следующими свойствами:

\begin{enumerate}
	\item
	Она \emph{положительно-однородна}:
	$\sufu{\alpha \ell}{Z} = \alpha \sufu{\ell}{Z}$, $\alpha \geqslant 0$.

	\item
	Она \emph{полуаддитивна}:
	$\sufu{\ell^1 + \ell^2}{Z} \leqslant \sufu{\ell^1}{Z} + \sufu{\ell^2}{Z}$
	(неравенство треугольника).

	\item Из первого и второго пунктов следует, что она \emph{выпукла}.

	\item Между выпуклыми компактами и $\sufu{\ell}{Z}$
	существует взаимнооднозначное соответствие.

Действительно, прямо из определения следует, что $\forall z\in Z$ имеет место неравенство 
\begin{equation}
  \label{supportFuncIneq}
	\scalar{\ell}{z} \leqslant \sufu{\ell}{Z},\quad \forall \ell \in \real^n.
\end{equation}

Если же $Z \in \conv \real^n$ (является выпуклым компактом),
то справедливо и обратное утверждение, то есть из \eqref{supportFuncIneq}
следует, что $z \in Z$. Тогда
\begin{equation*}
  Z = \bigcap\limits_{\ell \in real^n}{\pi^{-}_\ell},\quad
    \pi^{-}_\ell = \left\lbrace z \mid  z \leqslant \sufu{\ell}{X} \right\rbrace.
\end{equation*}
\end{enumerate}