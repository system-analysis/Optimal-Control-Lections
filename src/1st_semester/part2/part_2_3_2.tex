\subsubsection{Система с переменными коэффициентами}

Важное отличие непрерывных систем с переменными коэффициентами от непрерывных систем
с постоянными коэффициентами в том, что в этом случае управляемость может зависеть от
рассматриваемого промежутка времени. В то время как в случае постоянных коэффициентов,
если система управляема на некотором промежутке, то она управляема и на любом меньшем.

Рассмотрим однородную систему с переменными коэффициентами:
\begin{equation}
\label{linear_nonconst}
  \begin{cases}
    \dot{x}(t) = A(t) x(t) + B(t) u(t),\\
    A(t) \in \mathbb{R}^{n \times n}, B(t) \in \mathbb{R}^{n \times m}.
  \end{cases}
\end{equation}

Согласно полученному ранее результату, система \eqref{linear_nonconst}
является вполне управляемой тогда и только тогда, когда выполнено соотношение
\begin{equation*}
  \abs{W} \neq 0 \Leftrightarrow
  \forall \ell \neq 0 \Rightarrow H^\tran(t_1, \tau) \ell \neq 0 \Leftrightarrow
  \forall \ell \neq 0 \Rightarrow \ell^\tran X(t_1, \tau) B(\tau) \neq 0.
\end{equation*}

\begin{theorem}
  Пусть $t^* \in [t_0; t_1]$, матрицы $A(t)$ и $B(t)$ дифференцируемы $(n - 1)$ раз
  в окрестности $t^*$. Рассмотрим следующий набор матриц:
  \begin{equation*}
    \begin{cases}
      L_1(t) = B(t),\\
      L_{k}(t) = A(t) L_{k - 1}(t) - \frac{d L_{k - 1}(t)}{d t}, k = 2, \ldots, n.
    \end{cases}
  \end{equation*}
  
  Если $\rg\left[L_{1}(t^*) | L_{2}(t^*) | \ldots | L_{n}(t^*) \right] = n$,
  то система \eqref{linear_nonconst} вполне управляема.
\end{theorem}

\begin{note}
  Если матрицы $A(t)$ и $B(t)$ не зависят от времени, то $L_{k} = A^{k - 1} B$.
\end{note}

\begin{proof}
	Предположим обратное: существует такой ненулевой вектор $\ell$, что
	\begin{equation*}
	  \ell^\tran X(t_1, \tau) B(\tau) = \ell^\tran X(t_1, \tau) L_{1}(\tau) \equiv 0.
	\end{equation*}
	Продифференцируем данное равенство по $\tau$, учитывая, что
	$\frac{\partial X(t_1, \tau)}{\partial \tau} = -X(t_1, \tau) A(\tau)$:
	\begin{equation*}
	  -\ell^\tran X(t_1, \tau) A(\tau) B(\tau) + \ell^\tran X(t_1, \tau) \frac{\partial B(\tau)}{\partial \tau} \equiv 0
	  \Leftrightarrow
	  -\ell^\tran X(t_1, \tau) L_2(\tau) \equiv 0.
	\end{equation*}
	
	Используя метод математической индукции, получаем, что $\ell^\tran X(t_1, \tau) L_n(\tau) \equiv 0$
	в окрестности точки $t^*$. Вектор $\widetilde{\ell} = X^\tran(t_1, \tau) \ell$ при $\tau = t^*$ ортогонален
	всем $L_1, \ldots, L_n$, что противоречит условию полного ранга.
\end{proof}