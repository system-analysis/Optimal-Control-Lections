\subsection{О декомпозиции состояния нелинейной системы (отступление)}
Рассмотрим более общую систему, линейную по управлению.
\begin{equation*}
	\dot{x}(t) = f(x(t)) + g_1 (x(t))u_1 + \ldots + g_m (x(t))u_m
\end{equation*}

В линейном случае
\begin{gather*}
	f(x) = Ax	\\
	g_j(x) = B^j % Это столбец!
\end{gather*}

\begin{df}
	Будем называть {\it скобкой Ли} двух векторных полей следующее ?поле?:
	\begin{equation*}
		[f_1(\cdot), f_2(\cdot)] = \frac {\partial f_1}{\partial x} f_2 - 
								   \frac {\partial f_2}{\partial x} f_1. 
	\end{equation*}
	Если $[f_1, f_2] = 0$, то говорят, что поля $f_1$ и $f_2$ коммутируют.

\end{df}

В линейном случае
\begin{gather*}
	[f, g_j] = AB_j \\
	[f, [f, g_j]] = A_2 B_j
\end{gather*}
и коммутируемость полей равносильна коммутируемости (перестановочности) матриц.

Прикладной смысл скобки Ли:
% Про графичек с последовательным применением и разность путей / эпсилон при стремлении к нулю

% Про возможность сдвинуться туда, сюда и смысл - определение ещё движений. Так, на машине можно парковаться (двигаться вбок) путем повторения чередующихся шагов различных управлений (вперёд-назад) и поворота.