\subsubsection{Исследование разрешимости задачи моментов}

Задача моментов \eqref{L2_moment_task} разрешима тогда и только тогда, когда
\begin{equation}
\label{L2_moment_task_solvable_1}
  \forall \ell \neq 0 \Rightarrow
  \scalar{\ell}{c} \leqslant \sufu{\ell}{\soa_\mu^0[t_1]} =
  \mu \norm{h}_\LTwo,
\end{equation}
что эквивалентно неравенству
\begin{equation}
\label{L2_moment_task_solvable_2}
  \mu \geqslant
    \frac{\scalar{\ell}{c}}{\lefteqn{\norm{h}_\LTwo}\phantom{\norm{u}}} \Longleftrightarrow
  \mu \geqslant \mu^0 = \sup\limits_{\ell \neq 0}
    \frac{\scalar{\ell}{c}}{\lefteqn{\norm{h}_\LTwo}\phantom{\norm{u}}}.
\end{equation}

Имея в виду, что данный супремум конечен, распишем $\norm{h}_\LTwo$:
\begin{equation*}
	\norm{h}_\LTwo =
	\left[
	  \int\limits_{t_0}^{t_1} \scalar{H^\tran \ell}{H^\tran \ell} d\tau
	\right]^{\tfrac{1}{2}} =
	\left[
	  \scalar{\ell}{\left(
	    \int\limits_{t_0}^{t_1}H(t_1, \tau) H^\tran(t_1, \tau) \, d\tau
	  \right) \ell}
	\right]^{\tfrac{1}{2}}.
\end{equation*}

Обозначим через $W(t_1, t_0)$ следующее выражение:
\begin{equation*}
	W(t_1, t_0) = \int\limits_{t_0}^{t_1} H(t_1, \tau) H^\tran(t_1, \tau) \, d\tau.
\end{equation*}

\begin{note}
  Для краткости обозначений, везде далее $W = W(t_1, t_0)$.
\end{note}

В новых обозначениях опорная функция множества \eqref{L2_moment_task_soa} принимает вид:
\begin{equation}
\label{L2_moment_sufu_soa}
  \sufu{\ell}{\soa_\mu^0} = \mu \sqrt{\scalar{\ell}{W \ell}}.
\end{equation}

Следовательно, условие \eqref{L2_moment_task_solvable_2} можно записать следующим образом:
\begin{equation}
\label{L2_moment_task_solvable_3}
  \mu \geqslant \mu^0 =
    \sup_{\scriptscriptstyle\scalar{\ell}{W \ell} \neq 0}
      \frac{\scalar{\ell}{c}}{\sqrt{\scalar{\ell}{W \ell}}} =
    \sup_{\scriptscriptstyle\scalar{\ell}{W \ell} = 1} \scalar{\ell}{c} \Longleftrightarrow
  \frac{1}{\mu_0} = \inf_{\scalar{\ell}{c} = 1} \sqrt{\scalar{\ell}{W \ell}}.
\end{equation}

Матрица $W$ называется матрицей управляемости.
Рассмотрим различные случаи:

\begin{enumerate}
	\item $\abs{W} \neq 0$.

	Заметим, что $W$~--- матрица Грамма строк матрицы $H$, а т.\,к. $\abs{W} \neq 0$,
	то строки $H(t_1, \cdot)$ линейно независимы.

	Для любого $\ell$ верно, что $\scalar{\ell}{W \ell} \neq 0$,
	где $\sqrt{\scalar{\ell}{W \ell}}$~--- норма.

	$\mu_0$~--- норма от $c$, сопряженная к $\sqrt{\scalar{\ell}{W \ell}}$, выпишем это явно:
	\begin{equation*}
		\mu_0 = \sup \set{\scalar{\ell}{c}}{\scalar{\ell}{W \ell} = 1} = \sqrt{\scalar{c}{W^{-1} c}}.
	\end{equation*}

	Максимум достигается на
	$\ell^0 = \frac{W^{-1} c}{\sqrt{\scalar{c}{W^{-1} c}}}$,
	тогда $h^0(t_1, \tau) = H^\tran(t_1, \tau) \ell^0$.

	Используя то, что $\scalar{\ell^0}{W \ell^0} = 1$, найдём управление:
	\begin{equation*}
		u^0(\tau) = \mu^0 \frac{H^\tran(t_1, \tau) \ell^0}{\sqrt{\scalar{\ell^0}{W \ell^0}}} =
		\sqrt{\scalar{c}{W^{-1} c}} H^\tran(t_1, \tau) \frac{W^{-1} c}{\sqrt{\scalar{c}{W^{-1} c}}} =
		H^\tran(t_1, \tau) W^{-1}(t_1, \tau) c.
	\end{equation*}

	Для задачи моментов $\displaystyle\int\limits_{t_0}^{t_1} H(t_1, \tau) u(\tau) \, d\tau = c$
	имеем $\varGamma u = c$, тогда $u = \varGamma^\tran \ell$
	($\varGamma^\tran$~--- сопряженный оператор), отсюда $\varGamma \varGamma^\tran \ell = c$.

	Множество достижимости для этого случая~--- невырожденный эллипсоид $\soa_\mu$.
	Это случай полной управляемости,
	то есть если мы решили задачу для $[a, b]$, то можем решить и на
	$[c, d] \supset [a, b]$ (просто берём управление на $[c, a]$ и $[b, d]$ как угодно,
	а на $[a, b]$ уже решаем).

	\item $\abs{W} = 0$.

  Задача в этом случае является не всегда разрешимой.
  \begin{gather}
	  \notag \sufu{\ell}{\soa_\mu^0} = \mu \sqrt{\scalar{\ell}{W \ell}},\\
	  \notag \ell \in \ker W \Longleftrightarrow \sufu{\ell}{\soa_\mu^0} = 0,\\
	  \scalar{\ell}{c} \leqslant \sufu{\ell}{\soa_\mu^0} \label{nerm}.
  \end{gather}

  Тогда, если левая часть неравенства \eqref{nerm} положительна, а правая равна нулю,
  то \eqref{nerm} не выполнено, поэтому при любом $\mu \geqslant 0$ вектор
  $c \notin \soa_\mu^0$.
\end{enumerate}

На самом деле, $c \in \soa_\mu^0 \Longleftrightarrow c \in (\ker W)^\perp$.
% TODO: отредактировать то, что ниже
Если мы покажем, что из соотношения $c \in (\ker W)^\perp$ следует
$c \in \soa_\mu^0$,
то $c \notin (\ker W)^\perp \Leftarrow c \notin \soa_\mu^0$.

Если $c \in (\ker W)^\perp$,
то $\scalar{\ell}{c} \leqslant \sufu{\ell}{\soa_\mu^0}$, надо проверить лишь,
что $\ell \in (\ker W)^\perp$ (т.\,к. $\ell = \ell^1 + \ell^2$,
где $\ell \in W$, $\ell^1 \in \ker W$, $\ell^2 \in (\ker W)^\perp$).
\begin{gather*}
	\scalar{\ell}{c} = \scalar{\ell^2}{c},\\
	\sufu{\ell}{\soa_\mu^0} =
	\mu \sqrt{\scalar{\ell^1 + \ell^2}{W \ell^2}} =
	\mu \sqrt{\scalar{\ell^1}{W \ell^2} + \scalar{\ell^2}{W \ell^2}} ={}\\
	\phantom{\sufu{\ell}{\soa_\mu^0}}{}= \mu \sqrt{\scalar{\ell^2}{W \ell^2}} =
	\sufu{\ell^2}{\soa_\mu^0}.
\end{gather*}

Теперь находим $\mu_0$:
\begin{equation*}
	\mu_0 = \sup_{\ell \in (\ker W)^\perp}
	  \dfrac{\scalar{\ell}{c}}{\sqrt{\scalar{\ell}{W \ell}}} =
	\sup\set{\scalar{\ell}{c}}{\scalar{\ell}{W \ell} = 1, \ell \in (\ker W)^\perp}.
\end{equation*}