\subsection{Доказательство теоремы для случая постоянных коэффициентов}
Рассмотрим пространство $L = \Im (B|AB| \ldots A^{n-1}B)$. Докажем

\begin{lemma}
	L инвариантно относительно нашей линейной системы с постоянными коэффициентами ?Ссылка?
\end{lemma}
\begin{proof}
	Пусть $x^0 \in L$. Это верно, если $x^0$ имеет вид
	\begin{equation}
		x^0 = B v^0 + AB v^1 + \ldots + A^{n-1} B v^{n-1} \label{elemL}
	\end{equation}

	Выпишем формулу Коши и разложим матричную экспоненту в ряд.
	%Тут формула коши + разложение в ряд
	
	Из теоремы Гамильтона--Кэли следует, что любую степень матрицы можно представить в виде линейной комбинации
	%Тут линейная комбинация для A^k
	
	Подставим, соберём коэффициенты и получим, что $x(t)$ также представим в виде \eqref{elemL}, следовательно $x(t) \in L$, что и означает инвариантность L относительно нашей системы.
\end{proof}

Далее приведем достаточные условия управляемости для различных систем.