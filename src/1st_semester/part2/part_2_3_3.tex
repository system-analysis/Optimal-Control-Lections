\subsubsection{Система с периодическими коэффициентами}

Рассмотрим систему \eqref{linear_nonconst} (для простоты считаем $t_0 = 0$) с условием периодичности:
\begin{equation}
\label{period_cond}
  \exists T > 0 \colon A(t + T) = A(t), B(t + T) = B(t) \quad \forall t.
\end{equation}

\begin{theorem}
  Пусть выполнено условие \eqref{period_cond}. Пусть дополнительно $A(t)$ и $B(t)$
  являются аналитическими функциями по $t$, то есть в окрестности каждой точки представляются
  в виде сходящегося ряда Тейлора. Тогда, если
	\begin{equation*} 
		\rg \left[B(0) | X(T, 0) B(0) | \ldots | X^{n - 1}(T, 0) B(0)\right] = n, 
	\end{equation*}
	то система вполне управляема на любом промежутке времени.
\end{theorem}

\begin{proof}
	Предположим обратное: существует такой ненулевой вектор $\ell$, что
	\begin{equation*}
	  \ell^\tran X(t_1, \tau) B(\tau) \equiv 0, \tau \in [0; t_1] \Rightarrow
	  \ell^\tran X(t_1, \tau) B(\tau) \equiv 0 \quad \forall \tau \in \mathbb{R}.
  \end{equation*}
  Рассмотрим следующие значения по времени: $t_1 = T$, $\tau = -k T$, где $k = 0, 1, \ldots$, в таком случае
  \begin{gather*}
    X(T, -k T) = X^{k + 1}(T, 0),\\
    0 = \ell^\tran X(T, -k T) B(-k T) = \ell^\tran X^{k + 1}(T, 0) B(0),
  \end{gather*}
  что противоречит условию полного ранга.
\end{proof}