\subsection{Решение}

Запишем формулу Коши для уравнения \eqref{L2_task}:
\begin{equation*}
	x^1 = X(t_1, t_0) x^0 + \int\limits_{t_0}^{t_1} X(t_1, \tau) B(\tau) u(\tau)\,d\tau +
	\int\limits_{t_0}^{t_1} X(t_1, \tau) f(\tau)\,d\tau.
\end{equation*}

Данное равенство можно переписать в следующем виде:
\begin{gather}
\label{L2_moment_task}
%\label{ourTask}
	\begin{cases}
	  %\displaystyle
	  c = x^1 - X(t_1, t_0) x^0 - \int\limits_{t_0}^{t_1} X(t_1, \tau) f(\tau)\,d\tau,\\
	  H(t, \tau) = X(t, \tau) B(\tau),\\
	  %\displaystyle
	  \int\limits_{t_0}^{t_1} H(t_1, \tau) u(\tau)\,d\tau = c.
	\end{cases}
\end{gather}

Полученная система уравнений \eqref{L2_moment_task} называется задачей моментов.
Рассмотрим множество достижимости данной задачи:
\begin{equation}
\label{L2_moment_task_soa}
	\soa_\mu^0 (t_1, t_0) = \soa_\mu^0 [t_1] =
	\left\lbrace
	  \alpha \in \mathbb{R}^n \colon \exists u(\cdot), \norm{u}_{\LTwo} \leqslant \mu,	  
	  \int\limits_{t_0}^{t_1} H(t_1, \tau) u(\tau)\,d\tau = \alpha
	\right\rbrace.
\end{equation}

\begin{note}
  Если взять $x^0 = 0$, $f \equiv 0$, то $\soa_\mu^0 (t_1, t_0)$ есть множество концов траекторий
  системы \eqref{L2_task} в момент времени $t_1$.
\end{note}

Требуется найти минимальное положительное значение $\mu$, при котором $c \in \soa_\mu^0$.
Сформулируем и докажем утверждение относительно введенного множества \eqref{L2_moment_task_soa}.

\begin{stm}
	$\soa_\mu^0 \in \conv \mathbb{R}^n$, где $\conv \mathbb{R}^n$ есть множество
	непустых выпуклых компактов в $\mathbb{R}^n$.
\end{stm}

\begin{proof}
	Для доказательства исходного утверждения необходимо доказать
	три свойства: выпуклость, ограниченность и замкнутость.
	
	\emph{Выпуклость}:
	
	Пусть $c^1, c^2 \in \soa_\mu^0$, тогда найдутся соответствующие им допустимые
	управления $u^1(\cdot)$ и $u^2(\cdot)$ такие, что выполнены равенства
	\begin{equation}
	\label{L2_soa_proof_convex}
	  c^j = \int\limits_{t_0}^{t_1} H(t_1, \tau) u^j(\tau)\,d\tau.
	\end{equation}
	
	Пусть $c = \lambda c^1 + (1 - \lambda) c^2$ для некоторого $\lambda \in [0, 1]$.
	Требуется выяснить, принадлежит ли вектор $c$ множеству достижимости $\soa_\mu^0$.
	
	Рассмотрим управление $u(t) = \lambda u^1(t) + (1 - \lambda) u^2(t)$.
	В силу линейности по управлению интегральных соотношений \eqref{L2_soa_proof_convex},
	соответствующее интегральное равенство выполнено также для вектора $c$. Осталось проверить, что
	управление $u(t)$ является допустимым, а именно, что выполнено неравенство
	$\norm{u}_\LTwo \leqslant \mu$:
	\begin{equation*}
		\norm{u}_\LTwo \leqslant \lambda \norm{u^1}_\LTwo + (1 - \lambda) \norm{u^2}_\LTwo \leqslant \mu.
	\end{equation*}

	\emph{Ограниченность}:
	
	Рассмотрим следующую цепочку неравенств:
	\begin{gather*}
		\norm{\int\limits_{t_0}^{t_1} H(t_1, \tau) u(\tau)\,d\tau} \leqslant
		\int\limits_{t_0}^{t_1} \norm{H(t_1, \tau) u(\tau)} d\tau \leqslant
		\int\limits_{t_0}^{t_1} \norm{H(t_1, \tau)} \cdot \norm{u(\tau)} d\tau \leqslant{}\\
		\intertext{по неравенству Коши--Буняковского:}
		{}\leqslant
		\left(\int\limits_{t_0}^{t_1} \norm{H(t_1, \tau)}^2 d\tau\right)^{\frac{1}{2}} \cdot \norm{u}_\LTwo =
		\norm{H(t_1, \cdot)}_\LTwo \cdot \norm{u}_\LTwo \leqslant
		\mu \cdot \norm{H(t_1, \cdot)}_\LTwo \leqslant M,
	\end{gather*}
	где $M$~--- некоторая константа. Таким образом, множество $\soa_\mu^0$ лежит в шаре с радиусом
	$M$, следовательно, ограничено.

	\emph{Замкнутость}:
	
	Пусть имеется некоторая сходящаяся последовательность $\left\lbrace c^j \right\rbrace \in \soa_\mu^0$:
	$c^j \to c$. Требуется доказать замкнутость множества $\soa_\mu^0$, то есть доказать принадлежность
	вектора $c$ данному множеству. Для каждого вектора $c^j$ определено соответствующее ему допустимое управление:
	\begin{equation*}
	  c^j = \int\limits_{t_0}^{t_1} H(t_1, \tau) u^j(\tau)\,d\tau.
	\end{equation*}
	
	Множество допустимых управлений является шаром в функциональном пространстве $\LTwo$,
	следовательно, не является компактным множеством. Поэтому, управление $u(t)$,
	соответствующее вектору $c$, не может являться пределом (по норме) функций $u^j(t)$.
	Введем понятие слабой сходимости.
	
	\begin{df}
	  Последовательность функций $u^j(t)$ называется слабо сходящейся к $u(t)$
	  в пространстве $\LTwo$,
	  если $\forall g(t) \in \LTwo \Rightarrow
	  \int\limits_{t_0}^{t_1} g(t) u^j(t)\,dt \to
	  \int\limits_{t_0}^{t_1} g(t) u(t)\,dt$.
	\end{df}
	
	Отметим, что множество допустимых управлений является слабым компактом в пространстве $\LTwo$.
	Следовательно, без ограничения общности, будем считать, что последовательность $u^j(t)$ имеет слабый
	предел $u(t)$. В таком случае получаем:
	\begin{equation*}
		c^j = \int\limits_{t_0}^{t_1} H(t_1, \tau) u^j(\tau)\,d\tau
		\xrightarrow[j \rightarrow \infty]{}
		\int\limits_{t_0}^{t_1} H(t_1, \tau) u(\tau)\,d\tau = c \in \soa_\mu^0.
	\end{equation*}
	
	Таким образом, утверждение полностью доказано.
\end{proof}
	
Найдём опорную функцию множества \eqref{L2_moment_task_soa}:
\begin{gather*}
  \sufu{\ell}{\soa_\mu^0[t_1]} =
    \sup_{c \in \soa_\mu^0[t_1]} \scalar{\ell}{c} =
    \sup_{u} \int\limits_{t_0}^{t_1} \scalar{\ell}{H(t_1, \tau) u(\tau)} d\tau = {}\\
  {} = \sup_{u} \int\limits_{t_0}^{t_1} \scalar{H^\tran(t_1, \tau) \ell}{u(\tau)} d\tau \leqslant
    \sup_{u} \int\limits_{t_0}^{t_1} \norm{h(\tau)} \cdot \norm{u(\tau)} d\tau \leqslant{}\\
  {} \leqslant \sup_{u}
    \left(
      \left[\int\limits_{t_0}^{t_1} \norm{h(t_1, \tau)}^2 d\tau \right]^{\tfrac{1}{2}} \cdot
      \left[\int\limits_{t_0}^{t_1} \norm{u(\tau)}^2 d\tau \right]^{\tfrac{1}{2}}
    \right) = {}\\
  {} = \sup_{u} \left( \norm{h}_{\LTwo} \cdot \norm{u}_{\LTwo} \right) =
    \mu \cdot \norm{h}_{\LTwo},
\end{gather*}
где $h(\tau) = H^\tran(t_1, \tau) \ell$.

Покажем, что данная верхняя оценка опорной функции достижима при некотором управлении.
Зная, при каких условиях в приведенных неравенствах достигаются равенства,
получаем вид функции управления:
\begin{equation*}
  u(t) = \lambda H^\tran(t_1, t) \ell,\quad \lambda = \const \geqslant 0,
\end{equation*}
при этом константа $\lambda$ такова, что $\norm{u}_{\LTwo} = \mu$.
Если данная норма ненулевая, то параметр $\lambda$ определяется однозначно.
В таком случае функцию $u(t)$ можно записать в следующем виде:
\begin{equation*}
  u(t) = u_{\ell}^{*}(t) = \mu \frac{H^\tran(t_1, t) \ell}{\norm{H^\tran(t_1, \cdot) \ell}_{\LTwo}}.
\end{equation*}

\subsubsection{Исследование разрешимости задачи моментов}

Задача моментов \eqref{L2_moment_task} разрешима тогда и только тогда, когда
\begin{equation}
\label{L2_moment_task_solvable_1}
  \forall \ell \neq 0 \Rightarrow
  \scalar{\ell}{c} \leqslant \sufu{\ell}{\soa_\mu^0[t_1]} =
  \mu \norm{h}_\LTwo,
\end{equation}
что эквивалентно неравенству
\begin{equation}
\label{L2_moment_task_solvable_2}
  \mu \geqslant
    \frac{\scalar{\ell}{c}}{\lefteqn{\norm{h}_\LTwo}\phantom{\norm{u}}} \Longleftrightarrow
  \mu \geqslant \mu^0 = \sup\limits_{\ell \neq 0}
    \frac{\scalar{\ell}{c}}{\lefteqn{\norm{h}_\LTwo}\phantom{\norm{u}}}.
\end{equation}

Имея в виду, что данный супремум конечен, распишем $\norm{h}_\LTwo$:
\begin{equation*}
	\norm{h}_\LTwo =
	\left[
	  \int\limits_{t_0}^{t_1} \scalar{H^\tran \ell}{H^\tran \ell} d\tau
	\right]^{\tfrac{1}{2}} =
	\left[
	  \scalar{\ell}{\left(
	    \int\limits_{t_0}^{t_1}H(t_1, \tau) H^\tran(t_1, \tau) \, d\tau
	  \right) \ell}
	\right]^{\tfrac{1}{2}}.
\end{equation*}

Обозначим через $W(t_1, t_0)$ следующее выражение:
\begin{equation*}
	W(t_1, t_0) = \int\limits_{t_0}^{t_1} H(t_1, \tau) H^\tran(t_1, \tau) \, d\tau.
\end{equation*}

\begin{note}
  Для краткости обозначений, везде далее $W = W(t_1, t_0)$.
\end{note}

В новых обозначениях опорная функция множества \eqref{L2_moment_task_soa} принимает вид:
\begin{equation}
\label{L2_moment_sufu_soa}
  \sufu{\ell}{\soa_\mu^0} = \mu \sqrt{\scalar{\ell}{W \ell}}.
\end{equation}

Следовательно, условие \eqref{L2_moment_task_solvable_2} можно записать следующим образом:
\begin{equation}
\label{L2_moment_task_solvable_3}
  \mu \geqslant \mu^0 =
    \sup_{\scriptscriptstyle\scalar{\ell}{W \ell} \neq 0}
      \frac{\scalar{\ell}{c}}{\sqrt{\scalar{\ell}{W \ell}}} =
    \sup_{\scriptscriptstyle\scalar{\ell}{W \ell} = 1} \scalar{\ell}{c} \Longleftrightarrow
  \frac{1}{\mu_0} = \inf_{\scalar{\ell}{c} = 1} \sqrt{\scalar{\ell}{W \ell}}.
\end{equation}

Матрица $W$ называется матрицей управляемости.
Рассмотрим различные случаи:

\begin{enumerate}
	\item $\abs{W} \neq 0$.

	Заметим, что $W$~--- матрица Грамма строк матрицы $H$, а т.\,к. $\abs{W} \neq 0$,
	то строки $H(t_1, \cdot)$ линейно независимы.

	Для любого $\ell$ верно, что $\scalar{\ell}{W \ell} \neq 0$,
	где $\sqrt{\scalar{\ell}{W \ell}}$~--- норма.

	$\mu_0$~--- норма от $c$, сопряженная к $\sqrt{\scalar{\ell}{W \ell}}$, выпишем это явно:
	\begin{equation*}
		\mu_0 = \sup \set{\scalar{\ell}{c}}{\scalar{\ell}{W \ell} = 1} = \sqrt{\scalar{c}{W^{-1} c}}.
	\end{equation*}

	Максимум достигается на
	$\ell^0 = \frac{W^{-1} c}{\sqrt{\scalar{c}{W^{-1} c}}}$,
	тогда $h^0(t_1, \tau) = H^\tran(t_1, \tau) \ell^0$.

	Используя то, что $\scalar{\ell^0}{W \ell^0} = 1$, найдём управление:
	\begin{equation*}
		u^0(\tau) = \mu^0 \frac{H^\tran(t_1, \tau) \ell^0}{\sqrt{\scalar{\ell^0}{W \ell^0}}} =
		\sqrt{\scalar{c}{W^{-1} c}} H^\tran(t_1, \tau) \frac{W^{-1} c}{\sqrt{\scalar{c}{W^{-1} c}}} =
		H^\tran(t_1, \tau) W^{-1}(t_1, \tau) c.
	\end{equation*}

	Для задачи моментов $\displaystyle\int\limits_{t_0}^{t_1} H(t_1, \tau) u(\tau) \, d\tau = c$
	имеем $\varGamma u = c$, тогда $u = \varGamma^\tran \ell$
	($\varGamma^\tran$~--- сопряженный оператор), отсюда $\varGamma \varGamma^\tran \ell = c$.

	Множество достижимости для этого случая~--- невырожденный эллипсоид $\soa_\mu$.
	Это случай полной управляемости,
	то есть если мы решили задачу для $[a, b]$, то можем решить и на
	$[c, d] \supset [a, b]$ (просто берём управление на $[c, a]$ и $[b, d]$ как угодно,
	а на $[a, b]$ уже решаем).

	\item $\abs{W} = 0$.

  Задача в этом случае является не всегда разрешимой.
  \begin{gather}
	  \notag \sufu{\ell}{\soa_\mu^0} = \mu \sqrt{\scalar{\ell}{W \ell}},\\
	  \notag \ell \in \ker W \Longleftrightarrow \sufu{\ell}{\soa_\mu^0} = 0,\\
	  \scalar{\ell}{c} \leqslant \sufu{\ell}{\soa_\mu^0} \label{nerm}.
  \end{gather}

  Тогда, если левая часть неравенства \eqref{nerm} положительна, а правая равна нулю,
  то \eqref{nerm} не выполнено, поэтому при любом $\mu \geqslant 0$ вектор
  $c \notin \soa_\mu^0$.
\end{enumerate}

На самом деле, $c \in \soa_\mu^0 \Longleftrightarrow c \in (\ker W)^\perp$.
% TODO: отредактировать то, что ниже
Если мы покажем, что из соотношения $c \in (\ker W)^\perp$ следует
$c \in \soa_\mu^0$,
то $c \notin (\ker W)^\perp \Leftarrow c \notin \soa_\mu^0$.

Если $c \in (\ker W)^\perp$,
то $\scalar{\ell}{c} \leqslant \sufu{\ell}{\soa_\mu^0}$, надо проверить лишь,
что $\ell \in (\ker W)^\perp$ (т.\,к. $\ell = \ell^1 + \ell^2$,
где $\ell \in W$, $\ell^1 \in \ker W$, $\ell^2 \in (\ker W)^\perp$).
\begin{gather*}
	\scalar{\ell}{c} = \scalar{\ell^2}{c},\\
	\sufu{\ell}{\soa_\mu^0} =
	\mu \sqrt{\scalar{\ell^1 + \ell^2}{W \ell^2}} =
	\mu \sqrt{\scalar{\ell^1}{W \ell^2} + \scalar{\ell^2}{W \ell^2}} ={}\\
	\phantom{\sufu{\ell}{\soa_\mu^0}}{}= \mu \sqrt{\scalar{\ell^2}{W \ell^2}} =
	\sufu{\ell^2}{\soa_\mu^0}.
\end{gather*}

Теперь находим $\mu_0$:
\begin{equation*}
	\mu_0 = \sup_{\ell \in (\ker W)^\perp}
	  \dfrac{\scalar{\ell}{c}}{\sqrt{\scalar{\ell}{W \ell}}} =
	\sup\set{\scalar{\ell}{c}}{\scalar{\ell}{W \ell} = 1, \ell \in (\ker W)^\perp}.
\end{equation*}