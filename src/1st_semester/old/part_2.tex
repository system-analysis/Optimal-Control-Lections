\section{Непрерывная задача моментов}

\subsection{Постановка задачи}

Рассмотрим следующую задачу
\begin{gather}
\label{tMoment}
	\begin{cases}
	  \dot{x}(t) = A(t) x(t) + B(t) u(t) + f(t),\\
	  A(t) \in \real^{n \times n}, B(t) \in \real^{n \times m}, f(t) \in \real^{n \times 1},\\
	  x \in \real^{n \times 1}, u \in \real^{m \times 1}.
	\end{cases}
\end{gather}

Будем полагать, что $A$, $B$ и $f$~--- непрерывные функции.
Если же они измеримы, то задачу стоит понимать <<почти всюду>>, а решение~--- решением по Каратеодори.
Систему, находящуюся в начальном состоянии, необходимо привести в конечное состояние:
\begin{equation}
\label{tMoment2}
	x(t_0) = x^0 \longrightarrow x(t_1)= x^1.
\end{equation}

Задача слишком <<свободная>>, поэтому поставим дополнительное условие:
\begin{equation}
\label{ogr}
	\int\limits_{t_0}^{t_1} \norm{u(t)}^2 dt \leqslant \mu^2.
\end{equation}

Это ограничение также можно записать как $\norm{u}_\LTwo^2 \leqslant \mu^2$.

\begin{problem}
  Найти минимальное $\mu$, при котором задача \eqref{tMoment},
  \eqref{tMoment2} разрешима.
\end{problem}

\subsection{Решение}

По формуле Коши получим:
\begin{equation*}
	x^1 = X(t_1, t_0) x^0 + \int\limits_{t_0}^{t_1} X(t_1, \tau) B(\tau) u(\tau)\,d\tau +
	\int\limits_{t_0}^{t_1} X(t_1, \tau) f(\tau)\,d\tau.
\end{equation*}

Это утверждение эквивалентно следующей задаче, называемой задачей моментов:
\begin{gather}
\label{ourTask}
	\begin{cases}
	  \displaystyle
	  c = x^1 - X(t_1, t_0) x^0 - \int\limits_{t_0}^{t_1} X(t_1, \tau) f(\tau)\,d\tau,\\
	  H(t_1, \tau) = X(t_1, \tau) B(\tau),\\
	  \displaystyle
	  \int\limits_{t_0}^{t_1} H(t_1, \tau) u(\tau)\,d\tau = c.
	\end{cases}
\end{gather}

Введём множество достижимости:
\begin{gather*}
	\soa_\mu(t_1, t_0) = \soa_\mu[t_1] =
	\left\lbrace\left.
	  \int\limits_{t_0}^{t_1} H(t_1, \tau) u(\tau)\,d\tau \right| \norm{u}_\LTwo^2 \leqslant \mu^2
	\right\rbrace.
\end{gather*}

\begin{stm}
	$\soa_\mu \in \conv \mathbb{R}^n$.
\end{stm}

\begin{proof}
	Для доказательства исходного утверждения необходимо доказать
	три свойства: выпуклость, ограниченность и замкнутость.
	
	\emph{Выпуклость}:
	\begin{gather*}
	  c^1, c^2 \in \soa_\mu \Rightarrow
	  c^j = \int\limits_{t_0}^{t_1} H(t_1, \tau) u^j(\tau)\,d\tau,
	  \; \norm{u^j}_\LTwo \leqslant \mu.
	\end{gather*}
	Требуется показать, что:
	\begin{equation*}
		c = \lambda c^1 + (1 - \lambda) c^2 \in \soa_\mu,\quad \forall \lambda \in [0, 1].
	\end{equation*}
	
	Если мы возьмем
	\begin{equation*}
	  u(t) = \lambda u^1(t) + (1 - \lambda) u^2(t),
	\end{equation*}
	то, в силу выпуклости нормы, имеем:
	\begin{equation*}
		\norm{u}_\LTwo \leqslant \lambda \norm{u^1}_\LTwo + (1 - \lambda) \norm{u^2}_\LTwo \leqslant \mu.
	\end{equation*}
	
	Домножаем $c^1$ на $\lambda$, $c^2$ на $(1 - \lambda)$, складываем и получаем, что:
	\begin{equation*}
		c = \int\limits_{t_0}^{t_1} H(t_1, \tau) u(\tau)\,d\tau,\;
		\norm{u}_\LTwo \leqslant \mu.
	\end{equation*}
	Таким образом, $c \in \soa_\mu$, что и требовалось доказать.

	\emph{Ограниченность}: Воспользуемся неравенством Коши--Буняковского:
	\begin{equation*}
		\norm{\int\limits_{t_0}^{t_1} H(t_1,\tau) u(\tau)\,d\tau} \leqslant
		\int\limits_{t_0}^{t_1}\norm{H(t_1, \tau)} \norm{u(\tau)}d\tau \stackrel{\text{К.--Б.}}{\leqslant}
		\left(\int\limits_{t_0}^{t_1} \norm{H(t_1, \tau)}^2 d\tau\right)^{\tfrac{1}{2}} \mu \equiv \const.
	\end{equation*}

	\emph{Замкнутость}: По-человечески доказать это мы пока не сможем. Этот шар не компактен.
	В~частности, это связано с~различием сходимости по~норме и покоординатной сходимостью.
	Но всё-таки приведем доказательство замкнутости:
	\begin{equation*}
	  u^j\leftrightarrow c^j\in\soa_\mu.
	\end{equation*}

	\begin{df}
	  $u^j \xrightarrow[j \rightarrow \infty]{\text{слабо}} u^0$,
	  если $\forall g \in \LTwo \colon
	  \displaystyle\int\limits_{t_0}^{t_1} g(t) u^j(t)\,dt \to
	  \displaystyle\int\limits_{t_0}^{t_1} g(t) u^0(t)\,dt$.
	\end{df}
	
	Так~как это слабый компакт, то без ограничения общности будем считать, что
	\begin{gather*}
		u^j \xrightarrow[j \rightarrow \infty]{\text{слабо}} u,\\
		c^j = \int\limits_{t_0}^{t_1} H(t_1, \tau) u^j(\tau)\,d\tau,\\
		H(t_1, \tau) =
		  \begin{bmatrix}
		    H_1(t_1, \tau)\\
				\vdots\\
				H_n(t_1, \tau)
			\end{bmatrix}
		\in \mathbb R^{n \times m}.
	\end{gather*}
	
	Тогда получим:
	\begin{equation*}
		c_i^j = \int\limits_{t_0}^{t_1} H_i(t_1, \tau) u^j(\tau)\,d\tau
		\xrightarrow[j \rightarrow\ infty]{}
		\int\limits_{t_0}^{t_1} H_i(t_1, \tau) u(\tau)\,d\tau = c \in \soa_\mu.
	\end{equation*}
	
	Найдём опорную функцию:
	\begin{align*}
	  \sufu{\ell}{\soa_\mu} & =
	    \sup_C \set{\scalar{\ell}{c}}{c = \int\limits_{t_0}^{t_1} H(t_1, \tau) u(\tau)\,d\tau} =
	    \sup_{u(\cdot) \colon \eqref{ogr}}
	    \int\limits_{t_0}^{t_1} \scalar{\ell}{H(t_1, \tau) u(\tau)}\,d\tau = {}\\
	  {} & = \sup_{u(\cdot) \colon \eqref{ogr}}
	    \int\limits_{t_0}^{t_1} \scalar{H^\tran(t_1, \tau) \ell}{u(\tau)}\,d\tau
	    \stackrel{\text{К.--Б.}}{\leqslant}
	    \sup_{u(\cdot) \colon \eqref{ogr}}
	    \left(
	      \left[\int\limits_{t_0}^{t_1} \norm{h(t_1, \tau)}^2 d\tau \right]^{\tfrac{1}{2}} \norm{u}_\LTwo
	    \right) = {}\\
	  {} & = \mu \norm{h(t_1, \tau)}_\LTwo, \text{ где } h(t_1, \tau) = H^\tran(t_1, \tau).
	\end{align*}
	
	Поэтому, $u(t) = \lambda h(t_1, \tau)$,
	$\lambda = \const \geqslant 0$; супремум достигается на
	$u^*(\tau) = \dfrac{h(t_1, \tau)}{\norm{h(t_1, \tau)}}$, если $h(t_1, \tau) \neq 0$.
\end{proof}

\subsubsection{Исследование разрешимости задачи моментов}

\begin{align*}
  \text{\eqref{ourTask} разрешима} &
    \Leftrightarrow \forall \ell \neq 0 \Rightarrow
    \scalar{\ell}{c} \leqslant \sufu{\ell}{\soa_\mu} =
    \mu \norm{h}_\LTwo \Leftrightarrow
    \mu \geqslant \frac{\scalar{\ell}{c}}{\norm{h}_\LTwo} \Leftrightarrow{}\\
  {} & \Leftrightarrow \mu \geqslant \mu^0 =
    \sup\limits_{\ell \neq 0} \frac{\scalar{\ell}{c}}{\norm{h}_\LTwo}.
\end{align*}

Имея в виду, что супремум конечен, распишем $\norm{h}_\LTwo$:
\begin{equation*}
	\norm{h}_\LTwo =
	\left[
	  \int\limits_{t_0}^{t_1} \scalar{H^\tran \ell}{H^\tran \ell} d\tau
	\right]^{\tfrac{1}{2}} =
	\left[
	  \scalar{\ell}{\left(
	    \int\limits_{t_0}^{t_1}H(t_1, \tau) H^\tran(t_1, \tau) \, d\tau
	  \right) \ell}
	\right]^{\tfrac{1}{2}}.
\end{equation*}

Обозначим через $W(t_1, \tau)$ следующее выражение:
\begin{equation*}
	W(t_1, \tau) = \int\limits_{t_0}^{t_1} H(t_1, \tau) H^\tran(t_1, \tau) \, d\tau.
\end{equation*}

Рассмотрим различные случаи:

\begin{enumerate}
	\item $\abs{W(t_1, t_0)} \neq 0$.

	Заметим, что $W$~--- матрица Грамма строк матрицы $H$, а т.\,к. $\abs{W} \neq 0$,
	то строки $H(t_1, \cdot)$ линейно независимы.

	Для любого $\ell$ верно, что $\scalar{\ell}{W \ell} \neq 0$,
	где $\sqrt{\scalar{\ell}{W \ell}}$~--- норма.

	$\mu_0$~--- норма от $c$, сопряженная к $\sqrt{\scalar{\ell}{W \ell}}$, выпишем это явно:
	\begin{equation*}
		\mu_0 = \sup \set{\scalar{\ell}{c}}{\scalar{\ell}{W \ell} = 1} = \sqrt{\scalar{c}{W^{-1} c}}.
	\end{equation*}

	Максимум достигается на
	$\ell^0 = \frac{W^{-1} c}{\sqrt{\scalar{c}{W^{-1} c}}}$,
	тогда $h^0(t_1, \tau) = H^\tran(t_1, \tau) \ell^0$.

	Используя то, что $\scalar{\ell^0}{W \ell^0} = 1$, найдём управление:
	\begin{equation*}
		u^0(\tau) = \mu^0 \frac{H^\tran(t_1, \tau) \ell^0}{\sqrt{\scalar{\ell^0}{W \ell^0}}} =
		\sqrt{\scalar{c}{W^{-1} c}} H^\tran(t_1, \tau) \frac{W^{-1} c}{\sqrt{\scalar{c}{W^{-1} c}}} =
		H^\tran(t_1, \tau) W^{-1}(t_1, \tau) c.
	\end{equation*}

	Для задачи моментов $\displaystyle\int\limits_{t_0}^{t_1} H(t_1, \tau) u(\tau) \, d\tau = c$
	имеем $\varGamma u = c$, тогда $u = \varGamma^\tran \ell$
	($\varGamma^\tran$~--- сопряженный оператор), отсюда $\varGamma \varGamma^\tran \ell = c$.

	Множество достижимости для этого случая~--- невырожденный эллипсоид $\soa_\mu$.
	$W$ называют матрицей управляемости. Это случай полной управляемости,
	то есть если мы решили задачу для $[a, b]$, то можем решить и на
	$[c, d] \supset [a, b]$ (просто берём управление на $[c, a]$ и $[b, d]$ как угодно,
	а на $[a, b]$ уже решаем).

	\item $\abs{W(t_1, t_0)} = 0$.

  Задача в этом случае является не всегда разрешимой.
  \begin{gather}
	  \notag \sufu{\ell}{\soa_\mu} = \mu \sqrt{\scalar{\ell}{W \ell}},\\
	  \notag \ell \in \ker W \Leftrightarrow \sufu{\ell}{\soa_\mu} = 0,\\
	  \scalar{\ell}{c} \leqslant \sufu{\ell}{\soa_\mu} \label{nerm}.
  \end{gather}

  Тогда если левая часть неравенства \eqref{nerm} положительна, а правая равна нулю,
  то \eqref{nerm} не выполнено, поэтому:
  \begin{equation*}
	  c \notin (\ker W)^\perp \Rightarrow c \notin \soa_\mu \forall \mu.
  \end{equation*}

\end{enumerate}

На самом деле, $c \in \soa_\mu \Leftrightarrow c \in (\ker W)^\perp$.
% TODO: отредактировать то, что ниже
Если мы покажем, что
$c \in \soa_\mu \Leftarrow c \in (\ker W)^\perp$,
то $c \notin (\ker W)^\perp \Leftarrow c \notin \soa_\mu \forall\mu$.

Если $c \in (\ker W)^\perp$,
то $\scalar{\ell}{c} \leqslant \sufu{\ell}{\soa_\mu}$, надо проверить лишь,
что $\ell \in (\ker W)^\perp$ (т.\,к. $\ell = \ell^1 + \ell^2$,
где $\ell \in W$, $\ell^1 \in \ker W$, $\ell^2 \in (\ker W)^\perp$).
\begin{gather*}
	\scalar{\ell}{c} = \scalar{\ell^2}{c},\\
	\sufu{\ell}{\soa_\mu} =
	\mu \sqrt{\scalar{\ell^1 + \ell^2}{W \ell^2}} =
	\sqrt{\scalar{\ell^1}{W \ell^2} + \scalar{\ell^2}{W \ell^2}} =
	\sqrt{\scalar{\ell^2}{W \ell^2}} =
	\sufu{\ell^2}{\soa_\mu}.
\end{gather*}

Теперь находим $\mu_0$:
\begin{equation*}
	\mu_0 = \sup_{\ell \in (\ker W)^\perp}
	  \dfrac{\scalar{\ell}{c}}{\sqrt{\scalar{\ell}{W \ell}}} =
	\sup\set{\scalar{\ell}{c}}{\scalar{ell}{W \ell} = 1, \ell \in (\ker W)^\perp}.
\end{equation*}

{\newcommand{\rang}{\mathop{\mathrm{rang}}}

%Заменить | на что-нить более адекватное.. На \mid?

%Как лучше - \begin{gather*}
%	M=[B| AB| \dots| A^{n-1}B] \\
%\intertext{или}
%	M= \left[B\mid AB\mid \dots\mid A^{n-1}B\right] \ ?
%\end{gather*}

%Убрать матрицу M = [B|\dots|A^{n-1}B] совсем, или  наоборот именно её везде поставить?
% $$->begin/end{equation}

\subsection{Непрерывная задача оптимального управления}

\subsubsection{Система с постоянными коэффициентами}
Рассмотрим систему
\begin{equation}
\label{ConstKoefContSys}
  \begin{cases}
    \dot{x}(t) = A x(t) + B u(t),\\
    A = \const,\\
    B = \const.
  \end{cases}
\end{equation}

Для неё наша матрица $W$ имеет вид
\begin{equation*}
	W(t_1, t_0) = \int_{t_0}^{t_1} X(t_1, \tau) B B^\tran X^\tran(t_1, \tau) \, d\tau =
	\int_{t_0}^{t_1} e^{A(t_1 - \tau)} B B^\tran e^{A^\tran(t_1 - \tau)} \, d\tau.
\end{equation*}

Изучим $\Ker W$:
	%$l\in \Ker W \Leftrightarrow B^T e^{A^T(t_1-\tau)}l = 0 \ \forall \tau$
т.\,к. $W = W^\tran \geqslant 0$, то
\begin{multline*}
	\ell \in \Ker W \Leftrightarrow 
	\ell^\tran W \ell = 0 =
	\int_{t_0}^{t_1} (B^\tran e^{A^\tran(t_1 - \tau)} \ell)^\tran B^\tran e^{A^\tran(t_1 - \tau)} \ell \, d\tau =\\
	\int_{t_0}^{t_1} \norm{B^\tran e^{A^\tran(t_1 - \tau)} \ell} \, d\tau
	\Leftrightarrow \ell^\tran e^{A(t_1 - \tau)} B \equiv 0.
\end{multline*}
	
Последнее равенство следует, вообще говоря, понимать как равенство почти всюду,
но мы будем считать, что у нас все функции достаточно хорошие,
и оно выполняется вообще везде.

%переформулировать, заменить стрелочки на Необходимо и Достаточно...
\begin{stm}
  $\abs{W} \neq = 0 \Leftrightarrow \rang M = n$,
  где $M = [B | AB | \ldots | A^{n - 1} B]$~--- матрица,
  составленная из матриц $B$, $AB$, $\ldots$, $A^{n - 1} B$,
  поставленных в плотную друг к другу.
  $M \in \real^{n \times m n}$.
\end{stm}

\begin{proof}
  $\Longleftarrow$ (\emph{Достаточность}). Предположим обратное:
  пусть $M$~--- матрица полного ранга, но найдётся ненулевой вектор $\ell \in \Ker W$.
  Тогда $\ell^\tran e^{A(t_1 - \tau)} B \equiv 0$.
  Продифференцируем это равенство $n - 1$ раз по переменной $\tau$: 
  %Криво написана система?
  \begin{equation}
	  \begin{array}{r@{}cl}
		  -l^T e^{A(t_1-\tau)} &AB &\equiv 0,\\
		  \dots\\
		  (-1)^{n-1} l^T e^{A(t_1-\tau)} &A^{n-1}B &\equiv 0.
	  \end{array}
  \end{equation}
  Но тогда $l^Te^{A(t_1-\tau)} \perp B, AB, \ldots, A^{n-1}B$, что противоречит предположению.
  %тут нужно указать, что l^Te^{A(t_1-\tau)} \not= 0, но откуда это следует???
  
  $\Longrightarrow$ (\emph{Необходимость}).
  Предположим обратное, т.\,е. $\abs{W} \neq = 0$, но $\rang M < n$.
  Тогда найдётся вектор $\ell \neq = 0$,
  что $\ell^\tran B = \dots = \ell^\tran A^{n - 1} B = 0$.
  Из теоремы Гамильтона--Кэли следует, что любая степень матрицы является
  линейной комбинацией её первых $n-1$ степеней.

  $\ell^\tran A^k B \equiv 0 \forall k \geqslant 0$,
  а следовательно $\ell^\tran e^{A(t_1 - \tau)} B \equiv 0$,
  и $\abs{W} = 0$, что противоречит предположению.
\end{proof}
	
\begin{stm}
  $\Im W = \Im M$, или, что то же, $\Ker W = \left\lbrace \ell \colon \ell \perp \Im M\right\rbrace$.
\end{stm}
%Слишком много кванторов
\begin{proof}
	Доказательство идёт по той же самой схеме, что и предыдущее: %поэтому мы опускаем большую его часть.
	
	$l\perp \Im M$ $\ \Rightarrow\ $ $l \perp B,AB,\dots A^{n-1}B$ $\ \stackrel{\text{нер. Гамильтона--Кели}}{\Rightarrow}\ $  $l^T e^{A(t-\tau)}B\equiv0$ $\ \Rightarrow\ $ $l \in \Ker W$.
	
	Обратно: %первому пункту предыдущего утверждения.
	%хотя, по-хорошему, стоит расписать...
	
	$ y \in \Ker W \Rightarrow l^Te^{A(t_1-\tau)} \perp B, AB,\dots,A^{n-1}B \Rightarrow l\perp \Im [B|AB|\dots|A^{n-1}B]. $
\end{proof}

Чем непрерывный случай принципиально отличается от дискретного? Тем, что в дискретном случае матрица $M$ выглядела как $[B|\dots|A^{min(k_1-k_0,n)}B]$, ибо у нас было лишь ограниченное число шагов, и их могло просто не хватить.

\begin{df}
Пара матриц $A,B$ называется \emph{[полностью(?)] управляемой}, если $\rang[B|AB|\dots|A^{n-1}B]=n$.
\end{df}

\begin{ex}%[полностью управляемой системы].
	Рассмотрим маятник: $\ddot{x} + \omega^2 x = u$. $x$ --- координата маятника, $u$ --- сила, которую мы к нему прилагаем.
	
	Управляема ли эта система? Сейчас узнаем.
	
	Обозначим $x_1=x, x_2 = \dot{x}$. Тогда
	\begin{equation} \label{PendulumSystem}
		\left\{
		\begin{aligned}
			\dot{x}_1 &=x_2, \\
			\dot{x}_2 &=-\omega^2 x_1 +u,
		\end{aligned}
		\right.
	\end{equation}
	и тогда наши матрицы имеют вид
	%array?
	\begin{equation*}
		A=\begin{bmatrix}
			0 & 1 \\ -\omega^2 & 0
		\end{bmatrix}, \ 
		B=\begin{bmatrix}
			0\\1
		\end{bmatrix}.
	\end{equation*}
	\begin{equation*}
		[B|AB] = \begin{bmatrix} 0&1\\1&0 \end{bmatrix}.
	\end{equation*}
	То есть наша система управляема, что в общем-то согласуется с бытовыми представлениями --- с помощью сколь угодно большой силы можно поставить маятник в любое положение и придать ему любую скорость за сколь угодно малое время
	%"можно перевести маятник в любое стстояние"?

%переформулировать нормально надо бы... Много слов...
Ещё стоит заметить, что, казалось бы, в \eqref{PendulumSystem} координата маятника не зависит от управления, поэтому на неё нельзя влиять управлением, но, на самом деле, это не так, ибо она зависит от скорости, которая в свою очередь зависит от управления очень даже.
\end{ex}
%Может быть, стоит привести пример не полностью управляемой системы? Например маятник в пространстве, на который мы можем влиять только по одной координате, но это нефизично, да и неинтересно - как будто кроме маятников ничего нету.
%В общем, требуется пример неуправляемой системы.

Если система не полностью управляема, то
%комментарий в тетрадке "Мы такое раньше не рассматривали". Почему? потому что начальное состояние - не точка, а множество точек. Надо включить как-то.
%вообще, к чему этот текст?... привязать жёстче надо
%multline?
%много символов. получается каша. Структуризовать.
\begin{multline*}
	\soa[t_1] = \soa(t,t_0,x^0) = \{x^1 \mid \exists\, u(\cdot), \exists\, x^0 \in \soa^0 : x(t_1,t_0,x^0|u(\cdot)) = x^1 \}  ={}\\{}= e^{A(t_1 - t_0)}\soa^0 + \Im [B|\dots|A^{n-1}B],
\end{multline*}
$x^1 \in \soa[t_1]$ $\Leftrightarrow$ $\exists\,\mu, \exists\, x^0 \in \soa^0: x^1-e^{A(t_1-t_0)}x^0 = c \in \soa_\mu[t_1]$, а $\mathop{\cup}\limits_\mu \soa_\mu = \Im[B|\dots|A^{n-1}B]$.
%непонятная фраза:
Так вот, если $\soa^0 = \{x^0\}$, то $\soa[t_1]$ --- есть линейное многообразие, и можно сказать, что $e^{A(t_1-t_0)}x^0$ --- есть центр сплющенного эллипсоида $\soa_\mu$, а в остальном всё (???) то же самое. (то же самое, как и что? наборщик недоумевает\ldots).

%нужно привязать этот абзац к тексту про то, что координата "как бы" не зависит от управления, или тот текст к этому абзацу.
Приведём следующую немаловажную теорему:
\begin{theorem}[о декомпозиции] %в лекциях оформлена как Утверждение...
Для любой линейной системы с постоянными коэффициентами \eqref{ConstKoefContSys}  найдётся такое [невырожденное] преобразование координат $y=Tx$ $T\not = 0$, что $y=(y^1,y^2)$, $y^1\in\real^k$, $y^2\in\real^{n-k}$, где $k=\rang[B|AB|\dots|A^{n-1}B]$, и система преобразуется к виду
\begin{equation*}
	\begin{array}{l}
		\dot{y}^1 = A_{11} y^1 + A_{12} y^2 + B_1 u, \\
		\dot{y}^2 = A_{22}y^2,
	\end{array}
\end{equation*}
причём $(A_{11},B)$ --- полностью управляема.
\end{theorem}
Собственно, это теорема (как следует из названия) о декомпозиции системы на полностью управляемую и неуправляемую части. $y^2$ \glqqq плывёт\grqqq{} сам по себе, а $y^1$ можно привести куда угодно (даже несмотря на добавку $A_{12}y^2$, ибо система для $y^1$ полностью управляема).

% Первый набросок лекции №8

Мы рассматриваем непрерывную систему с постоянными коэффициентами без свободного члена:

% Тут если нужно - вставить ещё раз систему.
% Вступление - будем формализовывать результат предыдущей лекции.

\begin{theorem}[(о декомпозиции состояния)]
Существует замена
\begin{equation*}
\begin{array}{l}
    \exists T \in \real ^{n \times n};  y = Tx \\
	y = (y^1; y^2)^T, y^1 \in \real ^k, y^2 \in \real ^{n-k}, \text{где} \\
	k = \rg [B|AB| \ldots |A^{n-1}B],
\end{array}
\end{equation*}

обладающая свойством:
\begin{multline}
	\dot{y} = T \dot{x} = TAT^{-1} y + TBu \\
	TAT^{-1} =  \\ %Тут матрица блочная TB = %И тут матрица блочная	
\end{multline}

То есть, систему можно записать в виде
%% Тут делается нормальная СЛАУ
\begin{multline}
	\dot{y}^1 = A_{11} y^1 + A_{12} y^2 + B_1 u \\ % Полностью управляемая часть
	\dot{y}^2 = A_{22} y^2 % Управление вообще отсутствует
\end{multline}
\end{theorem}

%%%%%%%%%%%%
\subsection{О декомпозиции состояния нелинейной системы (отступление)}
Рассмотрим более общую систему, линейную по управлению.
\begin{equation*}
	\dot{x}(t) = f(x(t)) + g_1 (x(t))u_1 + \ldots + g_m (x(t))u_m
\end{equation*}

В линейном случае
\begin{gather*}
	f(x) = Ax	\\
	g_j(x) = B^j % Это столбец!
\end{gather*}

\begin{df}
	Будем называть {\it скобкой Ли} двух векторных полей следующее ?поле?:
	\begin{equation*}
		[f_1(\cdot), f_2(\cdot)] = \frac {\partial f_1}{\partial x} f_2 - 
								   \frac {\partial f_2}{\partial x} f_1. 
	\end{equation*}
	Если $[f_1, f_2] = 0$, то говорят, что поля $f_1$ и $f_2$ коммутируют.

\end{df}

В линейном случае
\begin{gather*}
	[f, g_j] = AB_j \\
	[f, [f, g_j]] = A_2 B_j
\end{gather*}
и коммутируемость полей равносильна коммутируемости (перестановочности) матриц.

Прикладной смысл скобки Ли:
% Про графичек с последовательным применением и разность путей / эпсилон при стремлении к нулю

% Про возможность сдвинуться туда, сюда и смысл - определение ещё движений. Так, на машине можно парковаться (двигаться вбок) путем повторения чередующихся шагов различных управлений (вперёд-назад) и поворота.

%%%%%%%%%%%%
\subsection{Доказательство теоремы для случая постоянных коэффициентов}
Рассмотрим пространство $L = \Im (B|AB| \ldots A^{n-1}B)$. Докажем

\begin{lemma}
	L инвариантно относительно нашей линейной системы с постоянными коэффициентами ?Ссылка?
\end{lemma}
\begin{proof}
	Пусть $x^0 \in L$. Это верно, если $x^0$ имеет вид
	\begin{equation}
		x^0 = B v^0 + AB v^1 + \ldots + A^{n-1} B v^{n-1} \label{elemL}
	\end{equation}

	Выпишем формулу Коши и разложим матричную экспоненту в ряд.
	%Тут формула коши + разложение в ряд
	
	Из теоремы Гамильтона--Кэли следует, что любую степень матрицы можно представить в виде линейной комбинации
	%Тут линейная комбинация для A^k
	
	Подставим, соберём коэффициенты и получим, что $x(t)$ также представим в виде \eqref{elemL}, следовательно $x(t) \in L$, что и означает инвариантность L относительно нашей системы.
\end{proof}

Далее приведем достаточные условия управляемости для различных систем.

%%%%%%%%%%%%
\subsection{Случай непрерывных систем с переменными коэффициентами}
Важное отличие непрерывных систем с переменными коэффициентами от непрерывных с постоянными коэффициентами и рассмотренных дискретных в том, что в этом случае управляемость может зависеть от рассматриваемого промежутка времени. В то время, как в случае постоянных коэффициентов или дискретном, если система управляема на некотором промежутке, то она управляема и на любом меньшем. Только берите управление побольше.

Рассмотрим однородную систему с переменными коэффициентами
%Тут система + ограничение на гладкость A и B

Обозначим
%Наверное, тут не гадер а что-то выравниваемое...
\begin{gather*}
	L_1	= B(t)  \\
	L_k = A(t) L_{k-1}(t) - \frac{d L_{k-1}}{dt}, k = 2, \ldots, n.
\end{gather*}

\begin{theorem}[(достаточное условие управляемости для непр. системы)]
	Если $\exists \bar{t} \in [t_0, t_1]$ такой, что
	\begin{equation*} \rg [L_1(\bar{t})| \ldots | L_n(\bar{t})] = n, \end{equation*}
	то система полностью управляема на $[t_0, t_1]$ (и больших - ??очевидно??).
\end{theorem}
\begin{proof}
	Вспомним наш критерий управляемости. 
	%%% Вот это надо хорошенько связать с 4й и осмыслить...
\end{proof}

\subsection{Случай непрерывных систем с периодическими коэффициентами}
Мы рассматриваем ту же систему
% система,
но уже когда $A(t)$ и $B(t)$ периодические с периодом T.

Потребуем аналитичности $A(t)$ и $B(t)$. %Надо написать определение или сослаться куда-нить...

\begin{theorem}[(достаточное условие управляемости для непр. системы с периодич. коэффициентами)]
	Если 
	\begin{equation*} 
		\rg [B(t_0)| X(T,0)B(t_0) | \ldots | X^{n-1} (T,0)B(t_0)] = n, 
	\end{equation*}
	то система полностью управляема на ?любом? промежутке времени.
\end{theorem}

\begin{proof}
	%%%% как тождественное равенство делать??? \equiv
	Предположим противное. Пусть $\exists\ l \neq 0$ т.ч. $l^T X(t_1,\tau)B(\tau) = 0$
	
	Посмотрим на это равенство в точках	$t_0, t_0 - T, \ldots, t_0 - (n-1)T$
\end{proof}

